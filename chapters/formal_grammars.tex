\chapter{Error Checkers}

This chapter will discuss the detection of these errors using context-free grammars (CFGs), Greibach normal forms (CNF), and Chomsky normal forms (CNF).

\newpage
\section{Grammars for Tagalog Errors}

\subsection{Tagalog Grammar: R-D Alternation, Example 1}
\subsubsection{CFG}
\begin{center}
    \begin{tabular}{rcl}
        \text{Start} & $ \rightarrow $ & \text{S "daw"} \\
        \text{S} & $ \rightarrow $ & \text{"sabi"} \\
    \end{tabular}
\end{center}

\subsubsection{CFG}
\begin{center}
    \begin{tabular}{rcl}
        \text{Start} & $ \rightarrow $ & \text{S D} \\
        \text{S} & $ \rightarrow $ & \text{"sabi"} \\
        \text{D} & $ \rightarrow $ & \text{"daw"} \\
    \end{tabular}
\end{center}

\subsubsection{GNF}
\begin{center}
    \begin{tabular}{rcl}
        \text{Z1} & $ \rightarrow $ & \text{"sabi" Z3} \\
        \text{Z3} & $ \rightarrow $ & \text{"daw"} \\
    \end{tabular}
\end{center}

\newpage
\subsection{Tagalog Grammar: R-D Alternation, Example 2}
\subsubsection{CFG}
\begin{center}
    \begin{tabular}{rcl}
        \text{Start} & $ \rightarrow $ & \text{H "raw" S} \\
        \text{H} & $ \rightarrow $ & \text{"hahabol"} \\
        \text{S} & $ \rightarrow $ & \text{"sila"} \\
    \end{tabular}
\end{center}

\subsubsection{CNF}
\begin{center}
    \begin{tabular}{rcl}
        \text{Start} & $ \rightarrow $ & \text{HR S} \\
        \text{H} & $ \rightarrow $ & \text{"hahabol"} \\
        \text{S} & $ \rightarrow $ & \text{"sila"} \\
        \text{R} & $ \rightarrow $ & \text{"raw"} \\
        \text{HR} & $ \rightarrow $ & \text{H R} \\
    \end{tabular}
\end{center}

\subsubsection{GNF}
\begin{center}
    \begin{tabular}{rcl}
        \text{Z1} & $ \rightarrow $ & \text{"hahabol" Z4 Z3} \\
        \text{Z3} & $ \rightarrow $ & \text{"sila"} \\
        \text{Z4} & $ \rightarrow $ & \text{"raw"} \\
    \end{tabular}
\end{center}

\newpage
\subsection{Tagalog Grammar: R-D Alternation, Example 3}
\subsubsection{CFG}
\begin{center}
    \begin{tabular}{rcl}
        \text{Start} & $ \rightarrow $ & \text{A "din"} \\
        \text{A} & $ \rightarrow $ & \text{"ako"} \\
    \end{tabular}
\end{center}

\subsubsection{CNF}
\begin{center}
    \begin{tabular}{rcl}
        \text{Start} & $ \rightarrow $ & \text{A D} \\
        \text{A} & $ \rightarrow $ & \text{"ako"} \\
        \text{D} & $ \rightarrow $ & \text{"daw"} \\
    \end{tabular}
\end{center}

\subsubsection{GNF}
\begin{center}
    \begin{tabular}{rcl}
        \text{Z1} & $ \rightarrow $ & \text{"ako" Z3} \\
        \text{Z3} & $ \rightarrow $ & \text{"daw"} \\
    \end{tabular}
\end{center}

\newpage
\subsection{Tagalog Grammar: Ng Case 1, Example 1}
\subsubsection{CFG}
\begin{center}
    \begin{tabular}{rcl}
        \text{Start} & $ \rightarrow $ & \text{K "ng" M} \\
        \text{K} & $ \rightarrow $ & \text{"kumain"} \\
        \text{M} & $ \rightarrow $ & \text{"mabagal"} \\
    \end{tabular}
\end{center}

\subsubsection{CNF}
\begin{center}
    \begin{tabular}{rcl}
        \text{Start} & $ \rightarrow $ & \text{KN M} \\
        \text{K} & $ \rightarrow $ & \text{"kumain"} \\
        \text{M} & $ \rightarrow $ & \text{"mabagal"} \\
        \text{N} & $ \rightarrow $ & \text{"ng"} \\
        \text{KN} & $ \rightarrow $ & \text{K N} \\
    \end{tabular}
\end{center}

\subsubsection{GNF}
\begin{center}
    \begin{tabular}{rcl}
        \text{Z1} & $ \rightarrow $ & \text{"kumain" Z4 Z3} \\
        \text{Z3} & $ \rightarrow $ & \text{"mabagal"} \\
        \text{Z4} & $ \rightarrow $ & \text{"ng"} \\
    \end{tabular}
\end{center}

\newpage
\subsection{Tagalog Grammar: Ng Case 1, Example 2}
\subsubsection{CFG}
\begin{center}
    \begin{tabular}{rcl}
        \text{Start} & $ \rightarrow $ & \text{T S "ng" M} \\
        \text{T} & $ \rightarrow $ & \text{"tumakbo"} \\
        \text{S} & $ \rightarrow $ & \text{"siya"} \\
        \text{M} & $ \rightarrow $ & \text{"mabilis"}
    \end{tabular}
\end{center}

\subsubsection{CNF}
\begin{center}
    \begin{tabular}{rcl}
        \text{Start} & $ \rightarrow $ & \text{TS NM} \\
        \text{T} & $ \rightarrow $ & \text{"tumakbo"} \\
        \text{S} & $ \rightarrow $ & \text{"siya"} \\
        \text{M} & $ \rightarrow $ & \text{"mabilis"} \\
        \text{N} & $ \rightarrow $ & \text{"ng"} \\
        \text{TS} & $ \rightarrow $ & \text{T S} \\
        \text{NM} & $ \rightarrow $ & \text{N M}
    \end{tabular}
\end{center}

\subsubsection{GNF}
\begin{center}
    \begin{tabular}{rcl}
        \text{Z1} & $ \rightarrow $ & \text{"tumakbo" Z3 Z7} \\
        \text{Z3} & $ \rightarrow $ & \text{"siya"} \\
        \text{Z4} & $ \rightarrow $ & \text{"mabilis"} \\
        \text{Z7} & $ \rightarrow $ & \text{"ng" Z4} 
    \end{tabular}
\end{center}

\newpage
\subsection{Tagalog Grammar: Ng Case 1, Example 3}
\subsubsection{CFG}
\begin{center}
    \begin{tabular}{rcl}
        \text{Start} & $ \rightarrow $ & \text{S "ng" M} \\
        \text{S} & $ \rightarrow $ & \text{"sumigaw"} \\
        \text{M} & $ \rightarrow $ & \text{"malakas"}
    \end{tabular}
\end{center}

\subsubsection{CNF}
\begin{center}
    \begin{tabular}{rcl}
        \text{Start} & $ \rightarrow $ & \text{SN M} \\
        \text{S} & $ \rightarrow $ & \text{"sumigaw"} \\
        \text{M} & $ \rightarrow $ & \text{"malakas"} \\
        \text{N} & $ \rightarrow $ & \text{"ng"} \\
        \text{SN} & $ \rightarrow $ & \text{S N}
    \end{tabular}
\end{center}

\subsubsection{GNF}
\begin{center}
    \begin{tabular}{rcl}
        \text{Z1} & $ \rightarrow $ & \text{"sumigaw" Z4 Z3} \\
        \text{Z3} & $ \rightarrow $ & \text{"malakas"} \\
        \text{Z4} & $ \rightarrow $ & \text{"ng"}
    \end{tabular}
\end{center}

\newpage
\subsection{Tagalog Grammar: Ng Case 2, Example 1}
\subsubsection{CFG}
\begin{equation*}
    \begin{aligned}
        \text{Start}   & \rightarrow \text{“ng” P S SA M}   \\
        \text{P} & \rightarrow \text{“pumunta”} \\
        \text{S} & \rightarrow \text{“siya”} \\
        \text{SA} & \rightarrow \text{“sa”} \\
        \text{M} & \rightarrow \text{“mall”}
    \end{aligned}
\end{equation*}

\subsubsection{CNF}
\begin{equation*}
    \begin{aligned}
        \text{Start}   & \rightarrow \text{NPS SAM} \\
        \text{P} & \rightarrow \text{“pumunta”} \\
        \text{S} & \rightarrow \text{“siya”} \\
        \text{SA} & \rightarrow \text{“sa”} \\
        \text{M} & \rightarrow \text{“mall”} \\
        \text{N} & \rightarrow \text{“ng”} \\
        \text{NP} & \rightarrow \text{N P} \\
        \text{SAM} & \rightarrow \text{SA M} \\
        \text{NPS} & \rightarrow \text{NP S}
    \end{aligned}
\end{equation*}

\subsubsection{GNF}
\begin{equation*}
    \begin{aligned}
        \text{Z1}   & \rightarrow \text{“ng” Z2 Z3 Z8} \\
        \text{Z22} & \rightarrow \text{“pumunta”} \\
        \text{Z3} & \rightarrow \text{“siya”} \\
        \text{Z5} & \rightarrow \text{“mall”} \\
        \text{Z8} & \rightarrow \text{“sa” Z5}
    \end{aligned}
\end{equation*}

\newpage
\subsection{Tagalog Grammar: Ng Case 2, Example 2}
\subsubsection{CFG}
\begin{center}
    \begin{tabular}{rcl}
        \text{Start} & $ \rightarrow $ & \text{"ng" M NA A} \\
        \text{M} & $ \rightarrow $ & \text{"maaraw"} \\
        \text{NA} & $ \rightarrow $ & \text{"na"} \\
        \text{A} & $ \rightarrow $ & \text{"araw"} 
    \end{tabular}
\end{center}

\subsubsection{CNF}
\begin{center}
    \begin{tabular}{rcl}
        \text{Start} & $ \rightarrow $ & \text{NM NAA} \\
        \text{M} & $ \rightarrow $ & \text{"maaraw"} \\
        \text{NA} & $ \rightarrow $ & \text{"na"} \\
        \text{A} & $ \rightarrow $ & \text{"araw"} \\
        \text{N} & $ \rightarrow $ & \text{"ng"} \\
        \text{NM} & $ \rightarrow $ & \text{N M} \\
        \text{NAA} & $ \rightarrow $ & \text{NA A} 
    \end{tabular}
\end{center}

\subsubsection{GNF}
\begin{center}
    \begin{tabular}{rcl}
        \text{Z1} & $ \rightarrow $ & \text{"ng" Z2 Z7} \\
        \text{Z2} & $ \rightarrow $ & \text{"maaraw"} \\
        \text{Z4} & $ \rightarrow $ & \text{"araw"} \\
        \text{Z7} & $ \rightarrow $ & \text{"na" Z4} 
    \end{tabular}
\end{center}

\newpage
\subsection{Tagalog Grammar: Ng Case 3, Example 3}
\subsubsection{CFG}
\begin{equation*}
    \begin{aligned}
        \text{Start}   & \rightarrow \text{“ng” U S “ng” T}   \\
        \text{U} & \rightarrow \text{“uminom”} \\
        \text{S} & \rightarrow \text{“siya”} \\
        \text{T} & \rightarrow \text{“tubig”}
    \end{aligned}
\end{equation*}

\paragraph{CNF}
\begin{equation*}
    \begin{aligned}
        \text{Start}   & \rightarrow \text{NU SNT}   \\
        \text{U} & \rightarrow \text{“uminom”} \\
        \text{S} & \rightarrow \text{“siya”} \\
        \text{T} & \rightarrow \text{“tubig”}
        \text{N} & \rightarrow \text{“ng”} \\
        \text{NU} & \rightarrow \text{N U} \\
        \text{NT} & \rightarrow \text{N T} \\
        \text{SNT} & \rightarrow \text{S NT}
    \end{aligned}
\end{equation*}
 

\subsubsection{GNF}
\begin{equation*}
    \begin{aligned}
        \text{Z1}   & \rightarrow \text{“ng” Z2 Z8}   \\
        \text{Z2} & \rightarrow \text{“uminom”} \\
        \text{Z4} & \rightarrow \text{“siya”} \\
        \text{Z7} & \rightarrow \text{“ng” Z4}
        \text{Z8} & \rightarrow \text{“siya” Z7}
    \end{aligned}
\end{equation*}

\newpage
\subsection{Tagalog Grammar: Ng Case 3, Example 1}
\subsubsection{CFG}
\begin{center}
    \begin{tabular}{rcl}
        \text{Start} & $ \rightarrow $ & \text{A M "ng" D} \\
        \text{A} & $ \rightarrow $ & \text{"ang"} \\
        \text{M} & $ \rightarrow $ & \text{"mamatay"} \\
        \text{D} & $ \rightarrow $ & \text{"dahil"} 
    \end{tabular}
\end{center}

\subsubsection{CNF}
\begin{center}
    \begin{tabular}{rcl}
        \text{Start} & $ \rightarrow $ & \text{AM ND} \\
        \text{A} & $ \rightarrow $ & \text{"ang"} \\
        \text{M} & $ \rightarrow $ & \text{"mamatay"} \\
        \text{D} & $ \rightarrow $ & \text{"dahil"} \\
        \text{N} & $ \rightarrow $ & \text{"ng"} \\
        \text{AM} & $ \rightarrow $ & \text{A M} \\
        \text{ND} & $ \rightarrow $ & \text{N D} 
    \end{tabular}
\end{center}

\subsubsection{GNF}
\begin{center}
    \begin{tabular}{rcl}
        \text{Z1} & $ \rightarrow $ & \text{"ang" Z3 Z7} \\
        \text{Z3} & $ \rightarrow $ & \text{"mamatay"} \\
        \text{Z4} & $ \rightarrow $ & \text{"dahil"} \\
        \text{Z7} & $ \rightarrow $ & \text{"ng" Z4} 
    \end{tabular}
\end{center}

\newpage
\subsection{Tagalog Grammar: Ng Case 3, Example 2}
\subsubsection{CFG}
\begin{center}
    \begin{tabular}{rcl}
        \text{Start} & $ \rightarrow $ & \text{A M "ng" B A} \\
        \text{A} & $ \rightarrow $ & \text{"ang"} \\
        \text{M} & $ \rightarrow $ & \text{"magtulungan"} \\
        \text{B} & $ \rightarrow $ & \text{"buo"} 
    \end{tabular}
\end{center}

\subsubsection{CNF}
\begin{center}
    \begin{tabular}{rcl}
        \text{Start} & $ \rightarrow $ & \text{AM NBA} \\
        \text{A} & $ \rightarrow $ & \text{"ang"} \\
        \text{M} & $ \rightarrow $ & \text{"magtulungan"} \\
        \text{B} & $ \rightarrow $ & \text{"buo"} \\
        \text{N} & $ \rightarrow $ & \text{"ng"} \\
        \text{AM} & $ \rightarrow $ & \text{A M} \\
        \text{NBA} & $ \rightarrow $ & \text{N BA} 
    \end{tabular}
\end{center}

\subsubsection{GNF}
\begin{center}
    \begin{tabular}{rcl}
        \text{Z1} & $ \rightarrow $ & \text{"ang" Z3 Z8} \\
        \text{Z2} & $ \rightarrow $ & \text{"ang"} \\
        \text{Z3} & $ \rightarrow $ & \text{"magtulungan"} \\
        \text{Z7} & $ \rightarrow $ & \text{"buo" Z2} \\
        \text{Z8} & $ \rightarrow $ & \text{"ng" Z7} 
    \end{tabular}
\end{center}


\newpage
\subsection{Tagalog Grammar: Ng Case 3, Example 3}
\subsubsection{CFG}
\begin{center}
    \begin{tabular}{rcl}
        \text{Start} & $ \rightarrow $ & \text{A M "ng" MT AY} \\
        \text{A} & $ \rightarrow $ & \text{"ang"} \\
        \text{M} & $ \rightarrow $ & \text{"magkasama"} \\
        \text{MT} & $ \rightarrow $ & \text{"matagal"} \\
        \text{AY} & $ \rightarrow $ & \text{"ay"} 
    \end{tabular}
\end{center}

\subsubsection{CNF}
\begin{center}
    \begin{tabular}{rcl}
        \text{Start} & $ \rightarrow $ & \text{AMN MTAY} \\
        \text{A} & $ \rightarrow $ & \text{"ang"} \\
        \text{M} & $ \rightarrow $ & \text{"magkasama"} \\
        \text{MT} & $ \rightarrow $ & \text{"matagal"} \\
        \text{AY} & $ \rightarrow $ & \text{"ay"} \\
        \text{N} & $ \rightarrow $ & \text{"ng"} \\
        \text{AM} & $ \rightarrow $ & \text{A M} \\
        \text{MTAY} & $ \rightarrow $ & \text{MT AY} \\
        \text{AMN} & $ \rightarrow $ & \text{AM N} 
    \end{tabular}
\end{center}

\subsubsection{GNF}
\begin{center}
    \begin{tabular}{rcl}
        \text{Z1} & $ \rightarrow $ & \text{"ang" Z3 Z6 Z8} \\
        \text{Z3} & $ \rightarrow $ & \text{"magkasama"} \\
        \text{Z5} & $ \rightarrow $ & \text{"ay"} \\
        \text{Z6} & $ \rightarrow $ & \text{"ng"} \\
        \text{Z8} & $ \rightarrow $ & \text{"matagal" Z5} 
    \end{tabular}
\end{center}

\newpage
\subsection{Tagalog Grammar: Ng Case 4, Example 1}
\subsubsection{CFG}
\begin{center}
    \begin{tabular}{rcl}
        \text{Start} & $ \rightarrow $ & \text{K B S R "ng" M} \\
        \text{K} & $ \rightarrow $ & \text{"kailangang"} \\
        \text{B} & $ \rightarrow $ & \text{"bitayin"} \\
        \text{S} & $ \rightarrow $ & \text{"si"} \\
        \text{R} & $ \rightarrow $ & \text{"Rizal"} \\
        \text{M} & $ \rightarrow $ & \text{"matakot"} 
    \end{tabular}
\end{center}

\subsubsection{CNF}
\begin{center}
    \begin{tabular}{rcl}
        \text{Start} & $ \rightarrow $ & \text{KBS RNM} \\
        \text{K} & $ \rightarrow $ & \text{"kailangang"} \\
        \text{B} & $ \rightarrow $ & \text{"bitayin"} \\
        \text{S} & $ \rightarrow $ & \text{"si"} \\
        \text{R} & $ \rightarrow $ & \text{"Rizal"} \\
        \text{M} & $ \rightarrow $ & \text{"matakot"} \\
        \text{N} & $ \rightarrow $ & \text{"ng"} \\
        \text{KB} & $ \rightarrow $ & \text{K B} \\
        \text{NM} & $ \rightarrow $ & \text{N M} \\
        \text{KBS} & $ \rightarrow $ & \text{KB S} \\
        \text{RNM} & $ \rightarrow $ & \text{R NM}
    \end{tabular}
\end{center}

\subsubsection{GNF}
\begin{center}
    \begin{tabular}{rcl}
        \text{Z1} & $ \rightarrow $ & \text{"kailangang" Z3 Z4 Z11} \\
        \text{Z3} & $ \rightarrow $ & \text{"bitayin"} \\
        \text{Z4} & $ \rightarrow $ & \text{"si"} \\
        \text{Z6} & $ \rightarrow $ & \text{"matakot"} \\
        \text{Z9} & $ \rightarrow $ & \text{"ng" Z6} \\
        \text{Z1} & $ \rightarrow $ & \text{"Rizal" Z9} 
    \end{tabular}
\end{center}

\newpage
\subsection{Tagalog Grammar: Ng Case 4, Example 2}
\subsubsection{CFG}
\begin{center}
    \begin{tabular}{rcl}
        \text{Start} & $ \rightarrow $ & \text{D S P SA O "ng" M} \\
        \text{D} & $ \rightarrow $ & \text{"dinala"} \\
        \text{S} & $ \rightarrow $ & \text{"si"} \\
        \text{P} & $ \rightarrow $ & \text{"Pedro"} \\
        \text{SA} & $ \rightarrow $ & \text{"sa"} \\
        \text{O} & $ \rightarrow $ & \text{"ospital"} \\
        \text{M} & $ \rightarrow $ & \text{"magamot"} 
    \end{tabular}
\end{center}

\subsubsection{CNF}
\begin{center}
    \begin{tabular}{rcl}
        \text{Start} & $ \rightarrow $ & \text{DSPSA ONM} \\
        \text{D} & $ \rightarrow $ & \text{"dinala"} \\
        \text{S} & $ \rightarrow $ & \text{"si"} \\
        \text{P} & $ \rightarrow $ & \text{"Pedro"} \\
        \text{SA} & $ \rightarrow $ & \text{"sa"} \\
        \text{O} & $ \rightarrow $ & \text{"ospital"} \\
        \text{M} & $ \rightarrow $ & \text{"magamot"} \\
        \text{N} & $ \rightarrow $ & \text{"ng"} \\
        \text{DS} & $ \rightarrow $ & \text{D S} \\
        \text{NM} & $ \rightarrow $ & \text{N M} \\
        \text{DSP} & $ \rightarrow $ & \text{DS P} \\
        \text{ONM} & $ \rightarrow $ & \text{O NM} \\
        \text{DSPSA} & $ \rightarrow $ & \text{DSP SA} 
    \end{tabular}
\end{center}

\subsubsection{GNF} 
\begin{center} 
\begin{tabular}{rcl} 
    \text{Z1} & $ \rightarrow $ & \text{"dinala" Z3 Z4 Z5 Z12} \\ \text{Z3} & $ \rightarrow $ & \text{"si"} \\ 
    \text{Z4} & $ \rightarrow $ & \text{"Pedro"} \\ 
    \text{Z5} & $ \rightarrow $ & \text{"sa"} \\ 
    \text{Z7} & $ \rightarrow $ & \text{"magamot"} \\ 
    \text{Z10} & $ \rightarrow $ & \text{"ng" Z7} \\ 
    \text{Z12} & $ \rightarrow $ & \text{"ospital" Z10} 
\end{tabular} 
\end{center}

\newpage
\subsection{Tagalog Grammar: Ng Case 4, Example 3}
\subsubsection{CFG:}
\begin{equation*}
    \begin{aligned}
        \text{Start}   & \rightarrow \text{I S “ng” M}   \\
        \text{I} & \rightarrow \text{“nag-ipon”} \\
        \text{S} & \rightarrow \text{“siya”} \\
        \text{M} & \rightarrow \text{“makabili”}
    \end{aligned}
\end{equation*}

\subsubsection{CNF}
\begin{equation*}
    \begin{aligned}
        \text{Start}   & \rightarrow \text{IS NM}   \\
        \text{I} & \rightarrow \text{“nag-ipon”} \\
        \text{S} & \rightarrow \text{“siya”} \\
        \text{M} & \rightarrow \text{“makabili”} \\
        \text{N} & \rightarrow \text{“ng”} \\
        \text{IS} & \rightarrow \text{I S} \\
        \text{NM} & \rightarrow \text{N M}
    \end{aligned}
\end{equation*}

\subsubsection{GNF}
\begin{equation*}
    \begin{aligned}
        \text{Z1}   & \rightarrow \text{“nag-ipon” Z3 Z7}   \\
        \text{Z3} & \rightarrow \text{“siya”} \\
        \text{Z4} & \rightarrow \text{“makabili”} \\
        \text{Z7} & \rightarrow \text{“ng” Z4}
    \end{aligned}
\end{equation*}Removed U-O Reduplication

\newpage
\subsection{Tagalog Grammar: Nang Case for Noun, Example 1}
\subsubsection{CFG}
\begin{equation*}
    \begin{aligned}
        \text{Start}   & \rightarrow \text{BG “nang” B}   \\
        \text{BG} & \rightarrow \text{“bag”} \\
        \text{B} & \rightarrow \text{“bata”}
    \end{aligned}
\end{equation*}

\subsubsection{CNF}
\begin{equation*}
    \begin{aligned}
        \text{Start}   & \rightarrow \text{BG NB}   \\
        \text{BG} & \rightarrow \text{“bag”} \\
        \text{B} & \rightarrow \text{“bata”} \\
        \text{N} & \rightarrow \text{“nang”} \\
        \text{NB} & \rightarrow \text{N B}
    \end{aligned}
\end{equation*}

\subsubsection{GNF}
\begin{equation*}
    \begin{aligned}
        \text{Z1}   & \rightarrow \text{“bag” Z5}   \\
        \text{Z3} & \rightarrow \text{“bata”} \\
        \text{Z5} & \rightarrow \text{“nang” Z3}
    \end{aligned}
\end{equation*}

\newpage
\subsection{Tagalog Grammar: Nang Case for Noun, Example 2}
\subsubsection{CFG}
\begin{equation*}
    \begin{aligned}
        \text{Start}   & \rightarrow \text{P “nang” B}   \\
        \text{P} & \rightarrow \text{“pinto”} \\
        \text{B} & \rightarrow \text{“bahay”}
    \end{aligned}
\end{equation*}

\subsubsection{CNF}
\begin{equation*}
    \begin{aligned}
        \text{Start}   & \rightarrow \text{PN B}   \\
        \text{P} & \rightarrow \text{“pinto”} \\
        \text{B} & \rightarrow \text{“bahay”} \\
        \text{N} & \rightarrow \text{“nang”} \\
        \text{PN} & \rightarrow \text{P N}
    \end{aligned}
\end{equation*}

\subsubsection{GNF}
\begin{equation*}
    \begin{aligned}
        \text{Z1}   & \rightarrow \text{“pinto” Z4 Z3}   \\
        \text{Z3} & \rightarrow \text{“bahay”} \\
        \text{Z4} & \rightarrow \text{“nang”}
    \end{aligned}
\end{equation*}

\newpage
\subsection{Tagalog Grammar: Nang Case for Noun, Example 3}
\subsubsection{CFG}
\begin{equation*}
    \begin{aligned}
        \text{Start}   & \rightarrow \text{H “nang” P}   \\
        \text{H} & \rightarrow \text{“hawakan”} \\
        \text{P} & \rightarrow \text{“pinto”}
    \end{aligned}
\end{equation*}

\subsubsection{CNF}
\begin{equation*}
    \begin{aligned}
        \text{Start}   & \rightarrow \text{HN P}   \\
        \text{H} & \rightarrow \text{“hawakan”} \\
        \text{P} & \rightarrow \text{“pinto”} \\
        \text{N} & \rightarrow \text{“nang”} \\
        \text{HN} & \rightarrow \text{H N} \\
    \end{aligned}
\end{equation*}

\subsubsection{GNF}
\begin{equation*}
    \begin{aligned}
        \text{Z1}   & \rightarrow \text{“hawakan” Z4 Z3}   \\
        \text{Z3} & \rightarrow \text{“pinto”} \\
        \text{Z4} & \rightarrow \text{“nang”}
    \end{aligned}
\end{equation*}

\newpage
\subsection{Tagalog Grammar: Nang Case for Verbs, Example 1}
\subsubsection{CFG}
\begin{equation*}
    \begin{aligned}
        \text{Start}   & \rightarrow \text{BIN “nang” B A S}   \\
        \text{BIN} & \rightarrow \text{“binalat”} \\
        \text{B} & \rightarrow \text{“bata”} \\
        \text{A} & \rightarrow \text{“ang”} \\
        \text{S} & \rightarrow \text{“saging”}
    \end{aligned}
\end{equation*}

\subsubsection{CNF}
\begin{equation*}
    \begin{aligned}
        \text{Start}   & \rightarrow \text{BIN NBAS}   \\
        \text{BIN} & \rightarrow \text{“binalat”} \\
        \text{B} & \rightarrow \text{“bata”} \\
        \text{A} & \rightarrow \text{“ang”} \\
        \text{S} & \rightarrow \text{“saging”} \\
        \text{N} & \rightarrow \text{“nang”} \\
        \text{NB} & \rightarrow \text{N B} \\
        \text{AS} & \rightarrow \text{A S} \\
        \text{NBAS} & \rightarrow \text{NB AS}
    \end{aligned}
\end{equation*}

\subsubsection{GNF}
\begin{equation*}
    \begin{aligned}
        \text{Z1}   & \rightarrow \text{“binalat” Z9}   \\
        \text{Z3} & \rightarrow \text{“bata”} \\
        \text{Z5} & \rightarrow \text{“saging”} \\
        \text{Z8} & \rightarrow \text{“ang” Z5} \\
        \text{Z9} & \rightarrow \text{“nang” Z3 Z8}
    \end{aligned}
\end{equation*}

\newpage
\subsection{Tagalog Grammar: Nang Case for Verbs, Example 2}
\subsubsection{CFG}
\begin{equation*}
    \begin{aligned}
        \text{Start}   & \rightarrow \text{T “nang” B}   \\
        \text{T} & \rightarrow \text{“tinapon”} \\
        \text{B} & \rightarrow \text{“basurero”}
    \end{aligned}
\end{equation*}

\subsubsection{CNF}
\begin{equation*}
    \begin{aligned}
        \text{Start}   & \rightarrow \text{TN B}   \\
        \text{T} & \rightarrow \text{“tinapon”} \\
        \text{B} & \rightarrow \text{“basurero”} \\
        \text{N} & \rightarrow \text{“nang”} \\
        \text{TN} & \rightarrow \text{T N}
    \end{aligned}
\end{equation*}

\subsubsection{GNF}
\begin{equation*}
    \begin{aligned}
        \text{Z1}   & \rightarrow \text{“tinapon” Z4 Z3}   \\
        \text{Z3} & \rightarrow \text{“basurero”} \\
        \text{Z4} & \rightarrow \text{“nang”}
    \end{aligned}
\end{equation*}

\newpage
\subsection{Tagalog Grammar: Nang Case for Verbs, Example 3}
\subsubsection{CFG}
\begin{equation*}
    \begin{aligned}
        \text{Start}   & \rightarrow \text{I “nang” L}   \\
        \text{I} & \rightarrow \text{“inalis”} \\
        \text{L} & \rightarrow \text{“lola”}
    \end{aligned}
\end{equation*}

\subsubsection{CNF}
\begin{equation*}
    \begin{aligned}
        \text{Start}   & \rightarrow \text{IN L}   \\
        \text{I} & \rightarrow \text{“inalis”} \\
        \text{L} & \rightarrow \text{“lola”} \\
        \text{N} & \rightarrow \text{“nang”} \\
        \text{IN} & \rightarrow \text{I N}
    \end{aligned}
\end{equation*}

\subsubsection{GNF}
\begin{equation*}
    \begin{aligned}
        \text{Z1}   & \rightarrow \text{“inalis” Z4 Z3}   \\
        \text{Z3} & \rightarrow \text{“lola”} \\
        \text{Z4} & \rightarrow \text{“nang”}
    \end{aligned}
\end{equation*}

\newpage
\subsection{Tagalog Grammar: Gitling Usage for "Alas-"}
\subsubsection{CFG}
\begin{center}
    \begin{tabular}{rcl}
        \text{Start} & $ \rightarrow $ & \text{Error\_1  \textbar  Error\_2  \textbar  Error\_3} \\
        \text{O} & $ \rightarrow $ & 1  \textbar  \text{"una"} \\
        \text{H} & $ \rightarrow $ & 2  \textbar  \text{"dos"} \\
        \text{H} & $ \rightarrow $ & 3  \textbar  \text{"tres"} \\
        \text{H} & $ \rightarrow $ & 4  \textbar  \text{"kwatro"} \\
        \text{H} & $ \rightarrow $ & 5  \textbar  \text{"singko"} \\
        \text{H} & $ \rightarrow $ & 6  \textbar  \text{"sais"} \\
        \text{H} & $ \rightarrow $ & 7  \textbar  \text{"syete"} \\
        \text{H} & $ \rightarrow $ & 8  \textbar  \text{"otso"} \\
        \text{H} & $ \rightarrow $ & 9  \textbar  \text{"nwebe"} \\
        \text{H} & $ \rightarrow $ & 10  \textbar  \text{"dyes"} \\
        \text{H} & $ \rightarrow $ & 11  \textbar  \text{"onse"} \\
        \text{H} & $ \rightarrow $ & 12  \textbar  \text{"dose"} \\
        \text{P} & $ \rightarrow $ & \text{" "} \\
        \text{G} & $ \rightarrow $ & \text{"-"} \\
        \text{Error\_1} & $ \rightarrow $ & \text{"alas" O  \textbar  "alas" H} \\
        \text{Error\_1} & $ \rightarrow $ & \text{"ala" H  \textbar  "ala" O} \\
        \text{Error\_2} & $ \rightarrow $ & \text{"alas" P O  \textbar  "alas" P H} \\
        \text{Error\_2} & $ \rightarrow $ & \text{"ala" P H  \textbar  "ala" P O} \\
        \text{Error\_3} & $ \rightarrow $ & \text{"alas" G O  \textbar  "ala" G H} \\
    \end{tabular}
\end{center}
For ease, the non-terminal derived from H is "h" representing the hours following the first hour.

\newpage
\subsubsection{CNF}
\begin{center}
    \begin{tabular}{rcl}
        \text{Start} & $ \rightarrow $ & \text{S O \textbar\ S H} \\
        \text{Start} & $ \rightarrow $ & \text{A O \textbar\ A H} \\
        \text{Start} & $ \rightarrow $ & \text{S PO \textbar\ S PH} \\
        \text{Start} & $ \rightarrow $ & \text{A PO \textbar\ A PH} \\
        \text{Start} & $ \rightarrow $ & \text{S GO \textbar\ A GH} \\
        \text{S} & $ \rightarrow $ & \text{"alas"} \\
        \text{A} & $ \rightarrow $ & \text{"ala"} \\
        \text{O} & $ \rightarrow $ & \text{"1" \textbar\ "una"} \\
        \text{H} & $ \rightarrow $ & \text{h} \\
        \text{P} & $ \rightarrow $ & \text{" "} \\
        \text{G} & $ \rightarrow $ & \text{"-"} \\
        \text{PO} & $ \rightarrow $ & \text{P O} \\
        \text{PH} & $ \rightarrow $ & \text{P H} \\
        \text{GO} & $ \rightarrow $ & \text{G O} \\
        \text{GH} & $ \rightarrow $ & \text{G H} \\
    \end{tabular}
\end{center}

\subsubsection{GNF}
\begin{center}
    \begin{tabular}{rcl}
        \text{Start} & $ \rightarrow $ & \text{"alas" O \textbar\ "alas" H} \\
        \text{Start} & $ \rightarrow $ & \text{"ala" O \textbar\ "ala" H} \\
        \text{Start} & $ \rightarrow $ & \text{"alas" P O \textbar\ "alas" P H} \\
        \text{Start} & $ \rightarrow $ & \text{"ala" P O \textbar\ "ala" P H} \\
        \text{Start} & $ \rightarrow $ & \text{"alas" G O \textbar\ "alas" G H} \\
        \text{O} & $ \rightarrow $ & \text{"1" \textbar\ "una"} \\
        \text{H} & $ \rightarrow $ & \text{h} \\
        \text{P} & $ \rightarrow $ & \text{" "} \\
        \text{G} & $ \rightarrow $ & \text{"-"} \\
    \end{tabular}
\end{center}

\newpage
\subsection{Tagalog Grammar: Gitling Usage for "Di-"}
\subsubsection{CFG}
\begin{center}
    \begin{tabular}{rcl}
        \text{Start} & $ \rightarrow $ & \text{Error\_1 \textbar\ Error\_2 \textbar\ Error\_3} \\
        \text{P} & $ \rightarrow $ & \text{" "} \\
        \text{G} & $ \rightarrow $ & \text{"-"} \\
        \text{W} & $ \rightarrow $ & \text{doktor \textbar\ lalaki \textbar\ mabait \textbar\ tulog} \\
        \text{C} & $ \rightarrow $ & \text{Doktor \textbar\ Lalaki \textbar\ Mabait \textbar\ Tulog} \\
        \text{Error\_1} & $ \rightarrow $ & \text{"di" W \textbar\ "di" P W} \\
        \text{Error\_1} & $ \rightarrow $ & \text{"di" C \textbar\ "di" P C} \\
        \text{Error\_2} & $ \rightarrow $ & \text{"hindi" G W \textbar\ "hindi" G C} \\
        \text{Error\_3} & $ \rightarrow $ & \text{"di" G C} \\
    \end{tabular}
\end{center}

\subsubsection{CNF}
\begin{center}
    \begin{tabular}{rcl}
        \text{Start} & $ \rightarrow $ & \text{S W \textbar\ S PW} \\
        \text{Start} & $ \rightarrow $ & \text{S C \textbar\ S PC} \\
        \text{Start} & $ \rightarrow $ & \text{H GW \textbar\ H GC} \\
        \text{Start} & $ \rightarrow $ & \text{S GC} \\
        \text{S} & $ \rightarrow $ & \text{"di"} \\
        \text{H} & $ \rightarrow $ & \text{"hindi"} \\
        \text{W} & $ \rightarrow $ & \text{doktor \textbar\ lalaki \textbar\ mabait \textbar\ tulog} \\
        \text{C} & $ \rightarrow $ & \text{Doktor \textbar\ Lalaki \textbar\ Mabait \textbar\ Tulog} \\
        \text{P} & $ \rightarrow $ & \text{" "} \\
        \text{G} & $ \rightarrow $ & \text{"-"} \\
        \text{PW} & $ \rightarrow $ & \text{P W} \\
        \text{PC} & $ \rightarrow $ & \text{P C} \\
        \text{GW} & $ \rightarrow $ & \text{G W} \\
        \text{GC} & $ \rightarrow $ & \text{G C} \\
    \end{tabular}
\end{center}

\subsubsection{GNF}
\begin{center}
    \begin{tabular}{rcl}
        \text{Start} & $ \rightarrow $ & \text{"di" W \textbar\ "di" P W} \\
        \text{Start} & $ \rightarrow $ & \text{"di" C \textbar\ "di" PC} \\
        \text{Start} & $ \rightarrow $ & \text{"hindi" GW \textbar\ "hindi" GC} \\
        \text{Start} & $ \rightarrow $ & \text{"di" GC} \\
        \text{W} & $ \rightarrow $ & \text{doktor \textbar\ lalaki \textbar\ mabait \textbar\ tulog} \\
        \text{C} & $ \rightarrow $ & \text{Doktor \textbar\ Lalaki \textbar\ Mabait \textbar\ Tulog} \\
        \text{P} & $ \rightarrow $ & \text{" "} \\
        \text{G} & $ \rightarrow $ & \text{"-"} \\
    \end{tabular}
\end{center}

\newpage
\subsection{Tagalog Grammar: Gitling Usage for Proper Nouns}
\subsubsection{CFG}
\begin{center}
    \begin{tabular}{rcl}
        \text{Start} & $ \rightarrow $ & \text{Error\_1 \textbar\ Error\_2} \\
        \text{P} & $ \rightarrow $ & \text{" "} \\
        \text{G} & $ \rightarrow $ & \text{"-"} \\
        \text{S} & $ \rightarrow $ & \text{"taga" \textbar\ "pa" \textbar\ "maka"} \\
        \text{PN} & $ \rightarrow $ & \text{"Tondo" \textbar\ "Davao" \textbar\ "Rizal" \textbar\ "DLSU"} \\
        \text{PV} & $ \rightarrow $ & \text{"tondo" \textbar\ "davao" \textbar\ "rizal" \textbar\ "dlsu"} \\
        \text{Error\_1} & $ \rightarrow $ & \text{S PN \textbar\ S P PN} \\
        \text{Error\_1} & $ \rightarrow $ & \text{S PV \textbar\ S P PV} \\
        \text{Error\_2} & $ \rightarrow $ & \text{S G PV} \\
    \end{tabular}
\end{center}

\subsubsection{CNF}
\begin{center}
    \begin{tabular}{rcl}
        \text{Start} & $ \rightarrow $ & \text{S PN \textbar\ S PPN} \\
        \text{Start} & $ \rightarrow $ & \text{S PV \textbar\ S PPV} \\
        \text{Start} & $ \rightarrow $ & \text{S GPV} \\
        \text{P} & $ \rightarrow $ & \text{" "} \\
        \text{G} & $ \rightarrow $ & \text{"-"} \\
        \text{S} & $ \rightarrow $ & \text{"taga" \textbar\ "pa" \textbar\ "maka"} \\
        \text{PN} & $ \rightarrow $ & \text{"Tondo" \textbar\ "Davao" \textbar\ "Rizal" \textbar\ "DLSU"} \\
        \text{PV} & $ \rightarrow $ & \text{"tondo" \textbar\ "davao" \textbar\ "rizal" \textbar\ "dlsu"} \\
        \text{PPN} & $ \rightarrow $ & \text{P PN} \\
        \text{PPV} & $ \rightarrow $ & \text{P PV} \\
        \text{GPV} & $ \rightarrow $ & \text{G PV} \\
    \end{tabular}
\end{center}

\subsubsection{GNF}
\begin{center}
    \begin{tabular}{rcl}
        \text{Start} & $ \rightarrow $ & \text{"taga" PN \textbar\ "taga" P PN} \\
        \text{Start} & $ \rightarrow $ & \text{"taga" PV \textbar\ "taga" P PV} \\
        \text{Start} & $ \rightarrow $ & \text{"taga" G PV} \\
        \text{Start} & $ \rightarrow $ & \text{"pa" PN \textbar\ "pa" P PN} \\
        \text{Start} & $ \rightarrow $ & \text{"pa" PV \textbar\ "pa" P PV} \\
        \text{Start} & $ \rightarrow $ & \text{"pa" GPV} \\
        \text{Start} & $ \rightarrow $ & \text{"maka" PN \textbar\ "maka" P PN} \\
        \text{Start} & $ \rightarrow $ & \text{"maka" PV \textbar\ "maka" P PV} \\
        \text{Start} & $ \rightarrow $ & \text{"maka" G PV} \\
        \text{P} & $ \rightarrow $ & \text{" "} \\
        \text{G} & $ \rightarrow $ & \text{"-"} \\
        \text{PN} & $ \rightarrow $ & \text{"Tondo" \textbar\ "Davao" \textbar\ "Rizal" \textbar\ "DLSU"} \\
        \text{PV} & $ \rightarrow $ & \text{"tondo" \textbar\ "davao" \textbar\ "rizal" \textbar\ "dlsu"} \\
    \end{tabular}
\end{center}


\newpage
\subsection{Tagalog Grammar: Gitling Usage for Foreign Words}
\subsubsection{CFG}
\begin{center}
    \begin{tabular}{rcl}
        \text{Start} & $ \rightarrow $ & \text{Error\_1 \textbar\ Error\_2 \textbar\ Edge\_case} \\
        \text{P} & $ \rightarrow $ & \text{" "} \\
        \text{G} & $ \rightarrow $ & \text{"-"} \\
        \text{S} & $ \rightarrow $ & \text{"pa" \textbar\ "ipa" \textbar\ "maki"} \\
        \text{W} & $ \rightarrow $ & \text{"message" \textbar\ "anime" \textbar\ "game"} \\
        \text{C} & $ \rightarrow $ & \text{"Message" \textbar\ "Anime" \textbar\ "Game"} \\
        \text{Z} & $ \rightarrow $ & \text{"Kpop"} \\
        \text{E} & $ \rightarrow $ & \text{"kpop" \textbar\ "K-pop" \textbar\ "k-pop"} \\
        \text{Error\_1} & $ \rightarrow $ & \text{S W \textbar\ S P W} \\
        \text{Error\_1} & $ \rightarrow $ & \text{S C \textbar\ S P C} \\
        \text{Error\_2} & $ \rightarrow $ & \text{S G C} \\
        \text{Edge\_case} & $ \rightarrow $ & \text{S G E \textbar\ S P E \textbar\ S E} \\
    \end{tabular}
\end{center}

\subsubsection{CNF}
\begin{center}
    \begin{tabular}{rcl}
        \text{Start} & $ \rightarrow $ & \text{S W \textbar\ S PW} \\
        \text{Start} & $ \rightarrow $ & \text{S C \textbar\ S PC} \\
        \text{Start} & $ \rightarrow $ & \text{S GC} \\
        \text{Start} & $ \rightarrow $ & \text{S GE \textbar\ S PE \textbar\ S E} \\
        \text{Start} & $ \rightarrow $ & \text{S PZ \textbar\ S Z} \\
        \text{P} & $ \rightarrow $ & \text{" "} \\
        \text{G} & $ \rightarrow $ & \text{"-"} \\
        \text{S} & $ \rightarrow $ & \text{"pa" \textbar\ "ipa" \textbar\ "maki"} \\
        \text{W} & $ \rightarrow $ & \text{"message" \textbar\ "anime" \textbar\ "game"} \\
        \text{C} & $ \rightarrow $ & \text{"Message" \textbar\ "Anime" \textbar\ "Game"} \\
        \text{Z} & $ \rightarrow $ & \text{"Kpop"} \\
        \text{E} & $ \rightarrow $ & \text{"kpop" \textbar\ "K-pop" \textbar\ "k-pop"} \\
        \text{PW} & $ \rightarrow $ & \text{P W} \\
        \text{PC} & $ \rightarrow $ & \text{P C} \\
        \text{GC} & $ \rightarrow $ & \text{G C} \\
        \text{PE} & $ \rightarrow $ & \text{P E} \\
        \text{GE} & $ \rightarrow $ & \text{G E} \\
        \text{PZ} & $ \rightarrow $ & \text{P Z} \\
    \end{tabular}
\end{center}

\newpage
\subsubsection{GNF}
\begin{center}
    \begin{tabular}{rcl}
        \text{Start} & $ \rightarrow $ & \text{"pa" W \textbar\ "pa" P W} \\
        \text{Start} & $ \rightarrow $ & \text{"pa" C \textbar\ "pa" P C} \\
        \text{Start} & $ \rightarrow $ & \text{"pa" G C} \\
        \text{Start} & $ \rightarrow $ & \text{"pa" G E \textbar\ "pa" P E \textbar\ "pa" E} \\
        \text{Start} & $ \rightarrow $ & \text{"pa" P Z \textbar\ "pa" Z} \\
        \text{Start} & $ \rightarrow $ & \text{"ipa" W \textbar\ "ipa" P W} \\
        \text{Start} & $ \rightarrow $ & \text{"ipa" C \textbar\ "ipa" P C} \\
        \text{Start} & $ \rightarrow $ & \text{"ipa" G C} \\
        \text{Start} & $ \rightarrow $ & \text{"ipa" G E \textbar\ "ipa" P E \textbar\ "ipa" E} \\
        \text{Start} & $ \rightarrow $ & \text{"ipa" P Z \textbar\ "ipa" Z} \\
        \text{Start} & $ \rightarrow $ & \text{"maki" W \textbar\ "maki" P W} \\
        \text{Start} & $ \rightarrow $ & \text{"maki" C \textbar\ "maki" P C} \\
        \text{Start} & $ \rightarrow $ & \text{"maki" G C} \\
        \text{Start} & $ \rightarrow $ & \text{"maki" G E \textbar\ "maki" P E \textbar\ "maki" E} \\
        \text{Start} & $ \rightarrow $ & \text{"maki" P Z \textbar\ "maki" Z} \\
        \text{P} & $ \rightarrow $ & \text{" "} \\
        \text{G} & $ \rightarrow $ & \text{"-"} \\
        \text{W} & $ \rightarrow $ & \text{"message" \textbar\ "anime" \textbar\ "game"} \\
        \text{C} & $ \rightarrow $ & \text{"Message" \textbar\ "Anime" \textbar\ "Game"} \\
        \text{Z} & $ \rightarrow $ & \text{"Kpop"} \\
        \text{E} & $ \rightarrow $ & \text{"kpop" \textbar\ "K-pop" \textbar\ "k-pop"} \\
    \end{tabular}
\end{center}

\newpage
\section{Grammars for Bikol Language Errors}
\subsection{Bikol Language Grammar: Object-Focused Future Tense, Example 1}
\subsubsection{CFG}
\begin{center}
    \begin{tabular}{rcl}
        \text{Start} & $ \rightarrow $ & \text{B "on"} \\
        \text{B} & $ \rightarrow $ & \text{"biho"} \\
    \end{tabular}
\end{center}

\subsubsection{CNF}
\begin{center}
    \begin{tabular}{rcl}
        \text{Start} & $ \rightarrow $ & \text{B O} \\
        \text{B} & $ \rightarrow $ & \text{"biho"} \\
        \text{O} & $ \rightarrow $ & \text{"on"} \\
    \end{tabular}
\end{center}

\subsubsection{GNF}
\begin{center}
    \begin{tabular}{rcl}
        \text{Z1} & $ \rightarrow $ & \text{"biho" Z2} \\
        \text{Z2} & $ \rightarrow $ & \text{"on"} \\
    \end{tabular}
\end{center}

\newpage
\subsection{Bikol Language Grammar: Object-Focused Future Tense, Example 2}
\subsubsection{CFG}
\begin{center}
    \begin{tabular}{rcl}
        \text{Start} & $ \rightarrow $ & \text{B "on"} \\
        \text{B} & $ \rightarrow $ & \text{"babasa"} \\
    \end{tabular}
\end{center}

\subsubsection{CNF}
\begin{center}
    \begin{tabular}{rcl}
        \text{Start} & $ \rightarrow $ & \text{B O} \\
        \text{B} & $ \rightarrow $ & \text{"babasa"} \\
        \text{O} & $ \rightarrow $ & \text{"on"} \\
    \end{tabular}
\end{center}

\subsubsection{GNF}
\begin{center}
    \begin{tabular}{rcl}
        \text{Z1} & $ \rightarrow $ & \text{"babasa" Z2} \\
        \text{Z2} & $ \rightarrow $ & \text{"on"} \\
    \end{tabular}
\end{center}

\newpage
\subsection{Bikol Language Grammar: Object-Focused Future Tense, Example 3}
\subsubsection{CFG}
\begin{center}
    \begin{tabular}{rcl}
        \text{Start} & $ \rightarrow $ & \text{M "hon"} \\
        \text{M} & $ \rightarrow $ & \text{"masasabot"} \\
    \end{tabular}
\end{center}

\subsubsection{CNF}
\begin{center}
    \begin{tabular}{rcl}
        \text{Start} & $ \rightarrow $ & \text{M H} \\
        \text{M} & $ \rightarrow $ & \text{"masasabot"} \\
        \text{H} & $ \rightarrow $ & \text{"hon"} \\
    \end{tabular}
\end{center}

\subsubsection{GNF}
\begin{center}
    \begin{tabular}{rcl}
        \text{Z1} & $ \rightarrow $ & \text{"masasabot" Z2} \\
        \text{Z2} & $ \rightarrow $ & \text{"hon"} \\
    \end{tabular}
\end{center}

\newpage
\subsection{Bikol Language Grammar: U-O Distribution for the Simple Case, Example 1}
\subsubsection{CFG}
\begin{center}
    \begin{tabular}{rcl}
        \text{Start} & $ \rightarrow $ & \text{T "o" K "o"} \\
        \text{T} & $ \rightarrow $ & \text{"T"} \\
        \text{K} & $ \rightarrow $ & \text{"kd"} \\
    \end{tabular}
\end{center}

\subsubsection{CNF}
\begin{center}
    \begin{tabular}{rcl}
        \text{Start} & $ \rightarrow $ & \text{TO KO} \\
        \text{T} & $ \rightarrow $ & \text{"T"} \\
        \text{K} & $ \rightarrow $ & \text{"kd"} \\
        \text{O} & $ \rightarrow $ & \text{"o"} \\
        \text{TO} & $ \rightarrow $ & \text{T O} \\
        \text{KO} & $ \rightarrow $ & \text{K O} \\
    \end{tabular}
\end{center}

\subsubsection{GNF}
\begin{center}
    \begin{tabular}{rcl}
        \text{Z1} & $ \rightarrow $ & \text{"T" Z2 Z3} \\
        \text{Z2} & $ \rightarrow $ & \text{"o"} \\
        \text{Z3} & $ \rightarrow $ & \text{"kd" Z2} \\
    \end{tabular}
\end{center}

\newpage
\subsection{Bikol Language Grammar: U-O Distribution for the Simple Case, Example 2}
\subsubsection{CFG}
\begin{center}
    \begin{tabular}{rcl}
        \text{Start} & $ \rightarrow $ & \text{B "o" L} \\
        \text{B} & $ \rightarrow $ & \text{"B"} \\
        \text{L} & $ \rightarrow $ & \text{"lan"} \\
    \end{tabular}
\end{center}

\subsubsection{CNF}
\begin{center}
    \begin{tabular}{rcl}
        \text{Start} & $ \rightarrow $ & \text{BO L} \\
        \text{B} & $ \rightarrow $ & \text{"B"} \\
        \text{L} & $ \rightarrow $ & \text{"lan"} \\
        \text{O} & $ \rightarrow $ & \text{"o"} \\
        \text{BO} & $ \rightarrow $ & \text{B O} \\
    \end{tabular}
\end{center}

\subsubsection{GNF}
\begin{center}
    \begin{tabular}{rcl}
        \text{Z1} & $ \rightarrow $ & \text{"B" Z3 Z2} \\
        \text{Z2} & $ \rightarrow $ & \text{"lan"} \\
        \text{Z3} & $ \rightarrow $ & \text{"o"} \\
    \end{tabular}
\end{center}

\newpage
\subsection{Bikol Language Grammar: U-O Distribution for the Simple Case, Example 3}
\subsubsection{CFG}
\begin{center}
    \begin{tabular}{rcl}
        \text{Start} & $ \rightarrow $ & \text{A "u"} \\
        \text{A} & $ \rightarrow $ & \text{"ak"} \\
    \end{tabular}
\end{center}

\subsubsection{CNF}
\begin{center}
    \begin{tabular}{rcl}
        \text{Start} & $ \rightarrow $ & \text{A U} \\
        \text{A} & $ \rightarrow $ & \text{"ak"} \\
        \text{U} & $ \rightarrow $ & \text{"u"} \\
    \end{tabular}
\end{center}

\subsubsection{GNF}
\begin{center}
    \begin{tabular}{rcl}
        \text{Z1} & $ \rightarrow $ & \text{"ak" Z2} \\
        \text{Z2} & $ \rightarrow $ & \text{"u"} \\
    \end{tabular}
\end{center}


\newpage
\subsection{Bikol Language Grammar: U-O Distribution for Verbs in Object-Focused Future Tense, Example 1}
\subsubsection{CFG}
\begin{center}
    \begin{tabular}{rcl}
        \text{Start} & $ \rightarrow $ & \text{M "u" T} \\
        \text{M} & $ \rightarrow $ & \text{"masasab"} \\
        \text{T} & $ \rightarrow $ & \text{"ton"} \\
    \end{tabular}
\end{center}

\subsubsection{CNF}
\begin{center}
    \begin{tabular}{rcl}
        \text{Start} & $ \rightarrow $ & \text{M U T} \\
        \text{M} & $ \rightarrow $ & \text{"masasab"} \\
        \text{T} & $ \rightarrow $ & \text{"ton"} \\
        \text{U} & $ \rightarrow $ & \text{"u"} \\
    \end{tabular}
\end{center}

\subsubsection{CNF}
\begin{center}
    \begin{tabular}{rcl}
        \text{Z1} & $ \rightarrow $ & \text{"masasab" Z3 Z2} \\
        \text{Z2} & $ \rightarrow $ & \text{"ton"} \\
        \text{Z3} & $ \rightarrow $ & \text{"u"} \\
    \end{tabular}
\end{center}

\newpage
\subsection{Bikol Language Grammar: U-O Distribution for Verbs in Object-Focused Future Tense, Example 2}
\subsubsection{CFG}
\begin{center}
    \begin{tabular}{rcl}
        \text{Start} & $ \rightarrow $ & \text{L "u" H} \\
        \text{L} & $ \rightarrow $ & \text{"lut"} \\
        \text{H} & $ \rightarrow $ & \text{"hon"} \\
    \end{tabular}
\end{center}

\subsubsection{CNF}
\begin{center}
    \begin{tabular}{rcl}
        \text{Start} & $ \rightarrow $ & \text{L U H} \\
        \text{L} & $ \rightarrow $ & \text{"lut"} \\
        \text{H} & $ \rightarrow $ & \text{"hon"} \\
        \text{U} & $ \rightarrow $ & \text{"u"} \\
    \end{tabular}
\end{center}

\subsubsection{GNF}
\begin{center}
    \begin{tabular}{rcl}
        \text{Z1} & $ \rightarrow $ & \text{"lut" Z3 Z2} \\
        \text{Z2} & $ \rightarrow $ & \text{"hon"} \\
        \text{Z3} & $ \rightarrow $ & \text{"u"} \\
    \end{tabular}
\end{center}

\newpage
\subsection{Bikol Language Grammar: U-O Distribution for Verbs in Object-Focused Future Tense, Example 3}
\subsubsection{CFG}
\begin{center}
    \begin{tabular}{rcl}
        \text{Start} & $ \rightarrow $ & \text{HR "u" H} \\
        \text{HR} & $ \rightarrow $ & \text{"her"} \\
        \text{H} & $ \rightarrow $ & \text{"hon"} \\
    \end{tabular}
\end{center}

\subsubsection{CNF}
\begin{center}
    \begin{tabular}{rcl}
        \text{Start} & $ \rightarrow $ & \text{HR UH} \\
        \text{HR} & $ \rightarrow $ & \text{"her"} \\
        \text{H} & $ \rightarrow $ & \text{"hon"} \\
        \text{U} & $ \rightarrow $ & \text{"u"} \\
        \text{UH} & $ \rightarrow $ & \text{U H} \\
    \end{tabular}
\end{center}

\subsubsection{GNF}
\begin{center}
    \begin{tabular}{rcl}
        \text{Z1} & $ \rightarrow $ & \text{"her" "u" Z2} \\
        \text{Z2} & $ \rightarrow $ & \text{"hon"} \\
    \end{tabular}
\end{center}
