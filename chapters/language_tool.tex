\chapter{LanguageTool}
\label{language_tool}
\section{Introduction to LanguageTool}

LanguageTool is an open-source (\url{https://github.com/languagetool-org/languagetool}) grammar and spell checker, originally by Daniel Naber \url{https://github.com/danielnaber}, that utilizes a combination of XML-based pattern rules, regular expressions, and Java code to detect and correct linguistic errors across multiple languages \cite{about_languagetool}. The core of its functionality lies in the \texttt{grammar.xml} file, where most rules are defined. These XML rules consist of patterns that match specific sequences of words or part-of-speech tags, allowing LanguageTool to identify common grammatical mistakes. Regular expressions enhance this capability by enabling the creation of flexible and complex matching patterns within the XML rules. For instance, a rule might use a regular expression to match variations of a verb to ensure subject-verb agreement. When certain grammatical issues cannot be effectively captured through XML patterns and regular expressions, LanguageTool allows for the implementation of custom Java rules \cite{LanguageToolGitHub}. 

Developers can extend LanguageTool's \texttt{Rule} class and implement matching function of that class, the \texttt{match(AnalyzedSentence)} method. Doing so allows the user to create more sophisticated checks that are not possible using regular expressions alone. This multi-faceted approach enables LanguageTool to provide comprehensive grammar checking by leveraging the strengths of regular expressions for flexibility, and Java for complex rule implementation \cite{LanguageToolGitHub}.

The developer documentation of LanguageTool discusses in detail the implementation of rules \cite{LanguageToolDevDocs}. In summary, the grammar checking of LanguageTool works as follows:

\paragraph{Boilerplate Structure of a Simple Rule} Each rule is defined by a \texttt{<rule>} element containing a \texttt{<patter>} to match the text and a \texttt{<message>} to display when the pattern is found.

\begin{lstlisting}[language=XML, caption=Boilerplate Code for a Simple Rule]
<rule id="RULE_ID" name="Rule Name">
    <pattern>
        <!-- Tokens and regular expressions go here -->
    </pattern>
    <message>Your message explaining the issue.</message>
    <example correction="corrected text">
        Incorrect text to be highlighted.
    </example>
</rule>    
\end{lstlisting}

\paragraph{Defining a Token} Each \texttt{<token>} represents a word or punctuation mark in the text, and we can restrict the matching for the \texttt{token} by using regular expressions or the \texttt{regexp} attribute.

For instance, we want to create a grammar rule in \texttt{grammar.xml} that defines and checks for the "alot" and "a lot" error.

\begin{lstlisting}[language= XML, caption=Example Grammar Rule for "alot" v. "a lot"]
<rule id="ALOT_RULE" name="Alot Correction">
    <pattern>
        <token regexp="yes">[aA]lot</token>
    </pattern>
    <message>
        'Alot' is a common misspelling. 
        Consider using 'a lot' instead.
    </message>
  <example correction="a lot">
    I have <marker>alot</marker> of work to do.
    </example>
</rule>
\end{lstlisting}


As for creating sophisticated grammar and style rules, these are discussed at length in Chapter \ref{future_work}.

\section{LanguageTool Implementation}

The complete source code for the LanguageTool implementation for the errors discussed in the previous chapters is available in the preface of the paper. The sections below will take a closer look at one of the grammars implemented for Tagalog and Bicol language.


\subsection{LanguageTool Implementation for [[TODO]]}

\subsection{LanguageTool Implementation for U-O Distribution}