\chapter{Preliminaries}
This section discusses some preliminaries for the case study. In particular, it gives a brief introduction to the Tagalog and Bicol language, and the notation the case study will use.

\section{Notation}
The following notation and rules for each notation, if any, will be used for the entirety of the paper to remove redundancy and notational ambiguity:

\paragraph{Languages and Alphabets}
All \emph{languages} are represented by script capital letters, while all \emph{alphabets} are represented by blackboard bold capital letters. For instance, the Filipino language is given by \(\mathcal{F}\) while the Filipino alphabet is given by \(\mathbb{F}\).

\paragraph{String Operations}
The dot operator (\(\cdot\)) denotes a \emph{concatenation} of two strings, the plus operator as a superscript (\(+\)) denotes the \emph{Kleene Plus}, the star operator as a superscript denotes the \emph{Kleene Star}, and the question mark as a superscript denotes the \emph{Kleene Question Mark} or the optional quantifier.

\begin{itemize}
    \item $s\cdot k$ is the concatenation of $s$ and $k$.
    \item $s^+$ is the Kleene star of $s$.
    \item $s^+$ is the Kleene plus of $s$.
    \item $s^?$ is the Kleene question mark of $s$.
\end{itemize}

\paragraph{Regular Expressions}
Regular expressions will use \emph{formal notation} and to say that some string \(s\) matches the regular expression modeled by \(\mathcal{R}\) then we denote this by \(s \in \mathcal{R}\).

Although the notation for regular expressions uses formal notation, for simplicity, syntax from the \texttt{IEEE POSIX} is occasionally used.

\paragraph{Special Characters in Regular Expressions} The following are special symbols that are used frequently in the paper:

\begin{center}
\begin{tabular}{l l l}
        \textbf{Symbol} & \textbf{Representation} & \textbf{Notes}                    \\
        $\psi$          & " "                     & the blank space symbol $|\psi|=1$ \\
        $\lambda$       & ""                      & the empty string $|\lambda| = 0$
    \end{tabular}
\end{center}

\paragraph{Remarks on IEEE POSIX notation} For cases where wildcard characters are used, such as \texttt{.} or the $\backslash$\texttt{w} will accept only valid characters in the alphabet $\mathbb{F}$, $\mathbb{B}$, or both. For instance, the regular expression \texttt{.} on the language $\mathcal{B}$ will accept $\forall s\in \mathbb{B}$ and will not accept $\forall j\notin \mathbb{B}$ like \texttt{=, +, $\backslash$, @}. Similarly, the regular expression \texttt{.$^*$} on $\mathcal{B}$ will accept all strings from $\mathbb{B}^*$.

\paragraph{Remarks on Case Sensitivity} The regular expressions for this paper are case insensitive, excluding Chapter \ref{proper_nouns_chapter}. That means, when we have the regular expression $\mathbb{B}^*$ this will accept "kumain", "KUMAIN", and other strings of the same letters but different cases. Furthermore, we say that "KUMAIN" and "kumain" are the same strings.

\section{Tagalog}
\subsection{The Filipino Alphabet}
Let \(\mathcal{F}\) be the Filipino language and \(\mathbb{F}\) be the Alphabet of \(\mathcal{F}\), this alphabet is
composed of 56 scripts and 11 punctuation marks \cite{OOP}. The 56 scripts are divided into
two, the first half being the capital letters of the modern Latin script with
the addition of \texttt{Ñ} and \texttt{Ng}; while the other half is the lowercase variants
of each letter.

The 11 punctuation marks in the Filipino language are the: \textit{tuldok} (.),
\textit{tandang pananong} (?), \textit{tandang padamdam} (!), \textit{kuwit} (,),
\textit{kudlit} ('), \textit{\textit{gitling}} (-), \textit{tutuldok} (:),
\textit{tuldok-kuwit} (;), \textit{panipi} ("), \textit{pambukas na panaklong}
((), \textit{pampasarang panaklong} ()), at ang \textit{tutuldok-tuldok} (...)

In mathematical notation, we can represent \(\mathbb{F}\) as the set:
\[
    \mathbb{F} = \{\texttt{a},\texttt{b},\dots,\texttt{z},\texttt{ñ},\texttt{ng},   \
    \texttt{A},\texttt{B},\texttt{C},\dots,\texttt{Z},\texttt{Ñ},\texttt{Ng}\}         \
    \cup \{\texttt{.},\texttt{?},\texttt{!},\texttt{,},\texttt{'},\texttt{-},\texttt{:}, \
    \texttt{;},\texttt{"},\texttt{(},\texttt{)}, \texttt{...}\}
\]

and the size of \(\mathbb{F}\), \(|\mathbb{F}| = 67\) and the set of all possible Filipino words is $\mathcal{F}$. Furthermore, from this, we can introduce the following sets:
\begin{enumerate}
    \item \(\mathbb{M}\) = \{\texttt{.}, \texttt{?}, \texttt{!}, \texttt{,}, \texttt{,}, \texttt{'}, \texttt{-}, \texttt{:}, \texttt{;}, \texttt{"}, \texttt{(}, \texttt{)}, \texttt{...}\}, the set of punctuation marks.
    \item \(\mathbb{V}_\mathbb{F}\) = \{\texttt{a}, \texttt{e}, \texttt{i}, \texttt{o}, \texttt{u}, \texttt{A}, \texttt{E}, \texttt{I}, \texttt{O}, \texttt{U}\}, the set of upper and lower case vowels.
    \item \(\mathbb{C}_\mathbb{F}\) = \(\mathbb{F} - (\mathbb{M} \cup \mathbb{V}_\mathbb{F})\),
          the set of upper and lower case consonants.
    \item \(\Sigma_\mathbb{F}\) = \(\mathbb{F} - \mathbb{M}_\mathbb{F}\), the set of consonants and vowels.
    \item The subscript "lower" on \(\mathbb{C}_\mathbb{F}, \mathbb{V}_\mathbb{F}, \Sigma_\mathbb{F}\) denotes the set of lowercase letters.
    \item The subscript "upper" on \(\mathbb{C}_\mathbb{F}, \mathbb{V}_\mathbb{F}, \Sigma_\mathbb{F}\) denotes the set of uppercase letters.
\end{enumerate}

\paragraph{Remark on the Digraph: \texttt{ng}/\texttt{Ng} or "\textit{en dyi}"}
\label{digraph_remark}

Although the letter "Ng" or "ng" is a concatenation of two separate graphemes or
symbols in \(\mathbb{F}\) (since \(\texttt{Ng} = \texttt{N}\cdot\texttt{g}\) and
\(\texttt{ng} = \texttt{n}\cdot\texttt{g}\)), the letter \texttt{Ng} is officially recognized
as a grapheme in \(\mathbb{F}\) since it represents a distinct Filipino sound.
In particular, it represents the \textit{voiced velar nasal sound}, or in the International
Phonetic Alphabet (IPA), the \textipa{N} sound \cite{Malabonga_2009}.

For instance, the word "hangin" has 5 letters namely: \texttt{h}, \texttt{a}, \texttt{g}, \texttt{i}, \texttt{n},
since \texttt{ng} is pronounced as a velar nasal sound, not as two separate sounds
\textipa{n-g}. So, "hangin" is pronounced as \textipa{haNin} ("ha-ngin").
Take for instance the English  word "manger" where "ng" is a substring
but is not pronounced as the velar nasal sound. Instead, its pronunciation is
\textipa{\textprimstress meI ndZ @r} ("meyn-jer"); not
\textipa{\textprimstress m\ae N Z @r} ("mang-jer"),
\textipa{\textprimstress m\ae N @r} ("mang-er") or
\textipa{\textprimstress m\ae N@r} ("manger").

With this in mind, the digraph \texttt{ng} will be treated as a single letter if the word pronunciation is the voiced velar nasal sound; otherwise it is the two separate characters \texttt{n} and \texttt{g}.

\section{Bicol Language}
\subsection{The Bicol Language and Alphabet}
Let $\mathcal{B}$ be the Bicol language and $\mathbb{B}$ be the alphabet of $\mathcal{B}$. The 64 scripts, the Filipino characters excluding the character \texttt{ñ} and \texttt{Ñ}, and including the vowel sounds with the glottal stop \texttt{'a}, \texttt{'e}, \texttt{'i}, \texttt{'o},  \texttt{'u}, and their upper-case equivalents. The punctuation marks of $\mathcal{F}$ are the same in $\mathcal{B}$.

Therefore, $|\mathbb{B}|=75$ and we can introduce the following subsets:
\begin{enumerate}
    \item \(\mathbb{V}_\mathbb{B}\) = \{\texttt{a}, \texttt{e}, \texttt{i}, \texttt{o}, \texttt{u}, \texttt{A}, \texttt{E}, \texttt{I}, \texttt{O}, \texttt{U}\}, the set of upper and lower case vowels.
    \item \(\mathbb{C}_\mathbb{B}\) = \(\mathbb{B} - (\mathbb{M} \cup \mathbb{V}_\mathbb{B})\),
          the set of upper and lower case consonants.
    \item \(\Sigma_\mathbb{B}\) = \(\mathbb{B} - \mathbb{M}_\mathbb{B}\), the set of consonants and vowels.
    \item The subscript "lower" on \(\mathbb{C}_\mathbb{B}, \mathbb{V}_\mathbb{B}, \Sigma_\mathbb{B}\) denotes the set of lowercase letters.
    \item The subscript "upper" on \(\mathbb{C}_\mathbb{B}, \mathbb{V}_\mathbb{B}, \Sigma_\mathbb{B}\) denotes the set of uppercase letters.
\end{enumerate}

For clarity, words that contain a vowel with a glottal stop and words that contain the same vowel sound without a glottal stop are not the same:

\[
    \forall v\in\mathbb{V}_\mathbb{B}, v^\prime\notin v
\]

And,

\[
    \Sigma_\mathbb{B}^*v\Sigma_\mathbb{B}^*\notin \Sigma_\mathbb{B}^*v^\prime\Sigma_\mathbb{B}^*
\]

Furthermore, since the digraph \texttt{ng} and \texttt{Ng} is in the Bicol language, the same rules discussed \ref{digraph_remark}.