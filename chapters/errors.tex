\chapter{Selected Grammatical Errors}
\label{errors}

This chapter discusses selected grammatical errors, the regular expression that can check for each error, and sample sentences.

\section{Selected Errors in Filipino}
\subsection{R-D Alternation}
\label{rd_alternation}

R-D Alternation refers to a word starting with \texttt{r} if the preceding word ends in a vowel or \textit{patinig}, and \texttt{d} if the preceding word ends in a consonant \textit{katinig} \cite{KWF, OOP}. 

Here we look at three cases where R-D Alternation occurs:

\paragraph{Enclitic Particles} When we have the sentence structure \(S = \alpha \cdot p \cdot x\) where $x$ is the enclitic particle \texttt{raw} or \texttt{daw}, and \(\alpha\) is any pangngalan, pang-uri, pandiwa, or pang-abay \cite{KWF}. 

\paragraph{Discourse particles} When we have the sentence structure $S = \omega \cdot p \cdot x$ where $x$ is the discourse particle \texttt{rin} or \texttt{din}, and $\omega$ is a pangngalan, pandiwa, pang-uri, or panghalip.

We can formally say that if the word before $x$ is matched by the regular expression:

\begin{example}
\end{example}

\begin{enumerate}
    \item Sabi \textbf{raw}... (correct)
        \\ Sabi \textit{daw}... (incorrect)
    \item Hahabol \textbf{daw} sila... (correct)
        \\ Hahabol \textit{raw} sila... (incorrect)
    \item Ako \textbf{rin}... (correct)
        \\ Ako \textit{din}... (incorrect)
    \item Subukan \textbf{din} natin... (correct)
        \\ Subukan \textit{rin} natin... (incorrect)
\end{enumerate}

\subsection{Ng v. Nang}

\paragraph{First Case of \texttt{Nang}} It can be used before an pang-abay when the sentence structure $S$ is given by \cite{KWF}:

\[
      S_1 = \text{v} \cdot p \cdot \text{pron}^? \cdot p \cdot \texttt{nang} \cdot p \cdot \text{adv}
\]

Where v is a pandiwa, pron is a panghalip, and adv is an pang-abay.

\begin{example}
\end{example}

\begin{enumerate}
      \item Kumain \textbf{nang} mabagal. (correct)
            \\ Kumain \textit{ng} mabagal. (incorrect)
      \item Tumakbo siya \textbf{nang} mabilis. (correct)
            \\ Tumakbo siya \textit{ng} mabilis. (incorrect)
      \item Sumigaw \textbf{nang} malakas. (correct)
            \\ Sumigaw \textit{ng} malakas. (incorrect)
\end{enumerate}

\paragraph{Second Case of \texttt{Nang}} Replacement for the pang-ugnay "noong" \cite{KWF}.

\[
      S_2 = \texttt{noong} \cdot p \cdot (\text{v} \vert \text{adv})
\]

\begin{example}

\end{example}

\begin{enumerate}
      \item \textbf{Nang} pumunta siya sa mall… (correct)
        \\ \textit{Ng} pumunta siya sa mall… (incorrect)
      \item \textbf{Nang} maaraw na araw… (correct)
            \\ \textit{Ng} maaraw na araw… (incorrect)
            \item \textbf{Nang} uminom siya ng tubig… (correct)
            \\ \textit{Ng} uminom siya ng tubig… (incorrect)
\end{enumerate}

\paragraph{Third Case of \texttt{Nang}} Replacement for the combined enclitic particle \texttt{na} $\cdot$ \texttt{-ng} \cite{KWF}.

\[
      S_3 = \text{v} \cdot p \cdot \text{"na"} \cdot \text{"-ng"}
\]

\begin{example}

\end{example}

\begin{enumerate}
      \item Ang mamatay \textbf{nang} dahil... (correct)
            \\ Ang mamatay \textit{ng} dahil... (incorrect)
      \item Ang magtulungan \textbf{nang} buo ang... (correct)
            \\ Ang magtulungan \textit{ng} buo ang... (incorrect)
      \item Ang magkasama \textbf{nang} matagal ay... (correct)
            \\ Ang magkasama \textit{ng} matagal ay... (incorrect)
\end{enumerate}

\paragraph{Fourth Case of \texttt{Nang}} Replacement for "upang" or "para". If we have a sentence that matches $S_4$, we can transform it to $S_4^\prime$ \cite{KWF}.

\[
      S_4 = (\texttt{upang}| \texttt{para}) \cdot p \cdot \text{v}
\]

\[
    S_4^\prime = \texttt{nang} \cdot p \cdot \text{v}
\]

\begin{example}

\end{example}

\begin{enumerate}
    \item \textbf{Para} kumain (original) 
        \\ \textbf{Nang} kumain (alternative) 
        \\ \textit{Ng} kumain (incorrect)
    \item \textbf{Upang} matutunan (original)
        \\ \textbf{Nang} matutunan (alternative)
        \\ \textit{Ng} matutunan (incorrect)
    \item \textbf{Upang} magtanong (original)
        \\ \textbf{Nang} magtanong (incorrect)
        \\ \textit{Ng} magtanong (incorrect)
\end{enumerate}

\noindent There are five ways to use "ng" in Filipino.

\paragraph{First Case of \texttt{Ng}} The string "ng" is used to denote possession similar to the word "of" in English when the structure of any sentence S is given by \cite{KWF}:

\[
      S_1 = \text{n}_1 \cdot p \cdot \texttt{ng} \cdot p \cdot \text{n}_1
\]

Where n$_1$ and n$_1$ are in the set of pangngalan,  n$_1$ is the possessor, and n$_2$ is the possessed.

\begin{example}
\end{example}

\begin{enumerate}
      \item Bag \textbf{ng} bata (correct)
            \\ Bag \textit{nang} bata (incorrect)
      \item Pinto \textbf{ng} bahay (correct)
            \\ Pinto \textit{nang} bahay (incorrect)
      \item Hawakan \textbf{ng} pinto (correct)
            \\ Hawakan \textit{nang} pinto (incorrect)
\end{enumerate}

\paragraph{Second Case of \texttt{Ng}} Marking the direct object of an action in a sentence \cite{KWF}.

\[
      S_2 = \text{v} \cdot p \cdot \texttt{ng} \cdot p \cdot \text{n}
\]

Where v is a pandiwa and n is a pangngalan.

\begin{example}
\end{example}

\begin{enumerate}
      \item Binalat \textbf{ng} bata ang saging. (correct)
            \\ Binalat \textit{nang} bata ang saging. (incorrect)
      \item Tinapon \textbf{ng} basurero... (correct)
            \\ Tinapon \textit{nang} basurero... (incorrect)
      \item Inalis \textbf{ng} lola... (correct)
            \\ Inalis \textit{nang} lola... (incorrect)
\end{enumerate}

\subsection{Gitling Usage}
\label{gitling_usage}
For this section, let $k$ be any Filipino word (\(k\in\mathcal{F}\)) and $s$ be any Spanish word (\(s\in\mathcal{S}\)) \cite{Balarila}.

\subsubsection{"Alas-"}

Let \(h\) be an hour in the 12-hour clock (numeric or non-numeric) excluding the first hour.
\[
      k = (\texttt{ala})\texttt{-}(\texttt{una} | 1)
\]

If $h$ is an hour that refers to one o'clock, then we use \texttt{ala-}.

\[
      k = (\texttt{alas})\texttt{-}h
\]

We use \texttt{alas}, otherwise.

\begin{example}
\end{example}
\begin{enumerate}
      \item \textbf{ala-1} ng tanghali. (correct) \\
            \textit{ala1} ng tanghali. (incorrect)
      \item \textbf{ala-una} ng umaga. (correct) \\
            \textit{ala una} ng umaga. (incorrect) \\
            \textit{alas-una} ng tanghali. (incorrect)
      \item \textbf{alas-tres} ng tanghali. (correct) \\
            \textit{alas tres} ng tanghali. (incorrect)
      \item \textbf{alas-2} ng tanghali. (correct) \\
            \textit{alas2} ng tanghali. (incorrect)
\end{enumerate}

\subsubsection{"Di-"}

Let $w\in\mathcal{F}$ be any pang-uri or pangngalan.

\[
      k = (\texttt{di-})w
\]

The prefix \texttt{di-} is a shortened format for the word \texttt{hindi}, and it acts as a negation to the root word $w$ \cite{Balarila}.

\begin{example}
\end{example}
\begin{enumerate}
      \item \textbf{di-doktor} (correct) \\
            \textit{di doktor} (incorrect)
      \item \textbf{di-tulog} (correct) \\
            \textit{di Tulog} (incorrect)
      \item \textbf{di-mainit} (correct) \\
            \textit{di mainit} (incorrect)
      \item \textbf{di-lalaki} (correct) \\
            \textit{di-lalaki} (incorrect)
      \item \textbf{di-mabait} (correct) \\
            \textit{di-Mabait} (incorrect)
\end{enumerate}

\subsubsection{Pangngalang Pambalana}
\label{proper_nouns_chapter}
Let $j \in \mathcal{F}$ be any pangngalang pambalana and $s$ is a derivational prefix that modifies the root word to create a root word \cite{rafael2018}.

\[
      k = s\texttt{-}j
\]

\begin{example}
\end{example}
\begin{enumerate}
      \item \textbf{taga-Tondo} (correct) \\
            \textit{taga Tondo} (incorrect)
      \item \textbf{maka-Rizal} (correct) \\
            \textit{makaRizal} (incorrect)
      \item \textbf{taga-Palawan} (correct) \\
            \textit{taga-palawan} (incorrect)
      \item \textbf{ka-Pasig} (correct) \\
            \textit{ka-pasig} (incorrect)
      \item \textbf{ka-Davao} (correct) \\
            \textit{ka davao} (incorrect)
\end{enumerate}

In this case, we have four derivational prefixes:
\begin{itemize}
    \item \texttt{taga-} which denotes the place of origin.
    \item \texttt{maka-} which denotes an inclination to something.
    \item \texttt{ka-} which denotes kinship.
\end{itemize}

\subsubsection{Foreign Words}
Let \(w \notin \mathcal{F}\) or it is any foreign word and $s$ is a derivational prefix.

\[
      k = s\texttt{-}w
\]

\begin{example}
\end{example}
\begin{enumerate}
      \item \textbf{pa-message} (correct) \\
            \textit{pa Message} (incorrect)
      \item \textbf{pa-cuenta} (correct) \\
            \textit{pa-Cuenta} (incorrect)
      \item \textbf{maki-Kpop} (correct) \\
            \textit{maki-K-pop} (incorrect)
      \item \textbf{maki-anime} (correct) \\
            \textit{maki Anime} (incorrect)
      \item \textbf{ipa-kanji} (correct) \\
            \textit{ipa kanji} (incorrect)
\end{enumerate}

In this case, we have three derivational prefixes:
\begin{itemize}
    \item \texttt{pa-} which denotes causation.
    \item \texttt{maki-} which denotes shared action or involvement.
    \item \texttt{ipa-} which denotes directing an action towards something.
\end{itemize}

\section{Selected Errors in the Bikol Language}
\subsection{Object-Focused Future Tense}
Any pandiwa in the Bikol Language can be transformed into its Object-Focused Future Tense by adding the suffix \cite{bikol_dictionary}:

\[
      \texttt{h}^?\texttt{on}
\]

In particular, the Kleene question mark is 1 if and only if the root word $r$ follows:

\[
      \backslash\texttt{b.}^*\texttt{[aeiou]}\backslash\texttt{b}
\]

\begin{example}
\end{example}

\begin{enumerate}
      \item \textbf{gibohon} (correct) \\
            \textit{bihoon} (incorrect)
      \item \textbf{babasahon} (correct) \\
            \textit{babasaon} (incorrect)
      \item \textbf{masasaboton} (correct) \\
            \textit{masasabothon} (incorrect)
\end{enumerate}

\subsection{U-O Distribution}
\label{uo_distribution}
\subsubsection{U-O Distribution: Simple Case}
Any word in the Bikolano language that has an \texttt{o} sound that is not the last syllable is pronounced as \texttt{u}. Only if the \texttt{o} sound is the last syllable of the word, that sound is pronounced as \texttt{o} \cite{bikol_dictionary}.

\begin{example}
\end{example}

\begin{enumerate}
      \item \textbf{Tukdo} (correct) \\
            \textit{Tokdo} or \textit{Tukdu} (incorrect) 
      \item \textbf{Bulan} (correct) \\
            \textit{Bolan} (incorrect) 
      \item \textbf{Ako} (correct) \\
            \textit{Aku} (incorrect) 
      \item \textbf{Du'ot} (correct) \\
            \textit{Du'ut} or \textit{Do'ot} (incorrect) 
\end{enumerate}

\subsubsection{U-O Distribution for Pandiwa in Object-Focused Future Tense}

Pandiwa with the $\texttt{-h}^?\texttt{on}$ suffix or pandiwa in the object-focused future tense form maintain the same root word $r$ spelling. Specifically, when we add the $\texttt{-h}^?\texttt{on}$ suffix to a root word that already has an \texttt{o} sound at its last syllable. We do not change the \texttt{o} sound of the root word. In other words, it maintains its spelling even after adding the suffix \cite{bikol_dictionary}.

\begin{example}
\end{example}

\begin{enumerate}
      \item \textbf{Masasaboton} (correct) \\
            \textit{Masasabuton} (incorrect)
      \item \textbf{Lutohon} (correct) \\
            \textit{Lutuhon} (incorrect)
      \item \textbf{Herohon} (correct) \\
            \textit{Heruhon} (incorrect)
\end{enumerate}
