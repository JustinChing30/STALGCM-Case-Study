\chapter{Selected Grammatical Errors}
This chapter discusses selected grammatical errors, the regular expression that can check for each error and sample sentences.

      % NOTES
      % COMMENT THIS OUT, once the chapter is done
      {\color{blue}
            This section will contain the preliminaries for the case study:

            \begin{itemize}
                  \item Errors in Tagalog.
                        \begin{itemize}
                              \item R-D Allophones
                              \item Ng v. Nang
                              \item Gitling Usage
                                    \begin{itemize}
                                          \item "Alas-"
                                          \item "Di-'
                                          \item Foreign Words and Proper Nouns
                                    \end{itemize}
                        \end{itemize}
                  \item Errors in the Bicol Language.
                        \begin{itemize}
                              \item Object-Focused Future Tense
                              \item U-O Distribution
                                    \begin{itemize}
                                          \item Simple Case
                                          \item Verbs in Object-Focused Future Tense
                                          \item Reduplications
                                    \end{itemize}
                        \end{itemize}

                        The format will be used for \textbf{each error}.
                        \begin{itemize}
                              \item Discussion of error
                              \item Regular Expression of error
                              \item Sample correct sentences
                              \item Sample incorrect sentences
                        \end{itemize}
            \end{itemize}
      }

\section{Selected Errors in Tagalog}
\subsection{R-D Allophones}
\subsection{Ng v. Nang}

There are five ways to use the word "nang" in Filipino.

\paragraph{First Case} It can be used before an adverb when the sentence structure $S$ is given by:

\[
      S_1 = \text{v} \cdot p \cdot \text{pron}^* \cdot p \cdot \text{"nang"} \cdot p \cdot \text{adv}
\]

Where v is a verb, pron is a pronoun, and adv is an adverb.

\begin{example}

\end{example}

\begin{enumerate}
      \item Kumain \textbf{nang} mabagal. (correct)
            \\ Kumain \textit{ng} mabagal. (incorrect)
      \item Tumakbo siya \textbf{nang} mabilis. (correct)
            \\ Tumakbo siya \textit{ng} mabilis. (incorrect)
      \item Sumigaw \textbf{nang} malakas. (correct)
            \\ Sumigaw \textit{ng} malakas. (incorrect)
\end{enumerate}

\paragraph{Second Case} Replacement for the conjunction "noong".

\[
      S_2 = \text{"noong"} \cdot p \cdot \text{v\textpipe adv}
\]

\begin{example}

\end{example}

\begin{enumerate}
      \item \textbf{Nang} pumunta siya sa mall… (correct)
            \\ \textit{Ng} pumunta siya sa mall… (incorrect)
      \item \textbf{Nang} maaraw na araw… (correct)
            \\ \textit{Ng} maaraw na araw… (incorrect)
\end{enumerate}

\paragraph{Third Case} Replacement for the combined Enclictic Particle "na" + "-ng".

\[
      S_3 = \text{v} \cdot p \cdot \text{"na"} \cdot \text{"-ng"}
\]

\begin{example}

\end{example}

\begin{enumerate}
      \item Ang mamatay \textbf{nang} dahil sa iyo. (correct)
            \\ Ang mamatay \textit{ng} dahil sa iyo. (incorrect)
\end{enumerate}

\paragraph{Fourth Case} Replacement for "upang" or "para".

\[
      S_4 = \text{"upang"\textpipe "para"} \cdot p \cdot \text{v}
\]

\begin{example}

\end{example}

\begin{enumerate}
      \item …kailangang bitayin si Rizal \textbf{nang} matakot ang mga Filipino. (correct)
            \\ …kailangang bitayin si Rizal \textit{ng} matakot ang mga Filipino. (correct)
      \item Dinala si Pedro sa ospital \textbf{nang} magamot. (correct)
            \\ Dinala si Pedro sa ospital \textit{ng} magamot. (incorrect)
\end{enumerate}

\paragraph{Fifth Case} Used inbetween repeated verbs.

\[
      S_5 = v_1 \cdot p \cdot \text{"nang"} \cdot p \cdot v_1
\]

\begin{example}

\end{example}

\begin{enumerate}
      \item Kain \textbf{nang} kain... (correct)
            \\ Kain \textit{ng} kain... (incorrect)
      \item Takbo \textbf{nang} takbo... (correct)
            \\ Takbo \textit{ng} takbo... (incorrect)
\end{enumerate}

The string "ng" is used to denote possession similar to
the word "of" in English when the structure of any sentence S is given by:

\[
      S_1 = \text{n} \cdot p \cdot \text{"ng"} \cdot p \cdot \text{n}
\]

Where n is a noun.

\begin{example}

\end{example}

\begin{enumerate}
      \item Bag \textbf{ng} bata (correct)
            \\ Bag \textit{nang} bata (incorrect)
      \item Pinto \textbf{ng} bahay (correct)
            \\ Pinto \textit{nang} bahay (incorrect)
      \item Hawakan \textbf{ng} pinto (correct)
            \\ Hawakan \textit{nang} pinto (incorrect)
\end{enumerate}

"Ng" can be used to mark the direct object of an action in a sentence.

\[
      S_2 = \text{v} \cdot p \cdot \text{"ng"} \cdot p \cdot \text{n}
\]

\begin{example}

\end{example}

\begin{enumerate}
      \item Binalat \textbf{ng} bata ang saging. (correct)
            \\ Binalat \textit{nang} bata ang saging. (incorrect)
      \item Tinapon \textbf{ng} basurero... (correct)
            \\ Tinapon \textit{nang} basurero... (incorrect)
      \item Inalis \textbf{ng} lola... (correct)
            \\ Inalis \textit{nang} lola... (incorrect)
\end{enumerate}

\begin{example}
      Here are examples of misused "nang"s and "ng"s in Filipino:
\end{example}

\begin{enumerate}
      \item Inalis \textbf{ng} lola... (correct) \\
            Inalis \textit{nang} lola... (incorrect) \\
            "Nang" is not used to denote possession.
      \item Bag \textbf{ng} bata. (correct) \\
            Bag \textit{nang} bata. (incorrect) \\
            "Nang" is not used to denote possession.
      \item Ang mamatay \textbf{nang} dahil sa iyo. (correct) \\
            Ang mamatay \textit{ng} dahil sa iyo. (incorrect) \\
            "Ng" is not used for combining Enclictic Particles "na" + "-ng".
      \item Sinapak \textbf{nang} masakit... (correct) \\
            Sinapak \textit{ng} masakit... (incorrect) \\
            "Ng" is not used before an adverb.
      \item Kain \textbf{nang} kain... (correct) \\
            Kain \textit{ng} kain... (incorrect) \\
            "Ng" is not used inbetween repeating verbs.
\end{enumerate}

\begin{example}
      Here are some examples of literature and webpages where "nang" or "ng" is
      misused:
\end{example}

\begin{enumerate}
      \item https://brainly.ph/question/21370957 \\
      \textbf{(Error: "takbo ng takbo")}
      \item https://www.facebook.com/teamagustintvvlogs/videos/kain-ng-kain-pero-bakit-hindi-tumataba-ano-daw-ang-aking-ginagawa/502889525820314/ \\
      \textbf{(Error: "kain ng kain")}
      \item https://www.facebook.com/groups/280224753328936/posts/1212892863395449/ \\
      \textbf{(Errors: "pa customized nang bed", "bed nang mga anak")}
      \item https://harrypotter.fandom.com/f/p/3008625405222454882 \\
      \textbf{(Error: "ng pumunta siya")}
\end{enumerate}

\subsection{Gitling Usage}
For this section, Let \(k\in\mathcal{F}\) and \(s\in\mathbb{S}\).
\subsubsection{"Alas-"}
Let \(h\) be an hour in the 12-hour clock (numeric or non-numeric) excluding the first hour.
\[
      k = (\text{"ala"}) \cdot "\text{-}" \cdot (\text{"una"} \ {\textpipe} \ 1)
\]
\begin{center}
"ala" for the following hours.
\end{center}
\[
      k = (\text{"alas"}) \cdot "\text{-}" \cdot h
\]
\begin{center}
"alas" for the following hours.
\end{center}

\begin{example}
\end{example}
\begin{enumerate}
      \item \textbf{ala-1} ng tanghali. (correct) \\
            \textit{ala1} ng tanghali. (incorrect)
      \item \textbf{ala-una} ng umaga. (correct) \\
            \textit{ala una} ng umaga. (incorrect)
      \item \textbf{ala-una} ng tanghali. (correct) \\
            \textit{alas-una} ng tanghali. (incorrect)
      \item \textbf{alas-tres} ng tanghali. (correct) \\
            \textit{alas tres} ng tanghali. (incorrect)
      \item \textbf{alas-2} ng tanghali. (correct) \\
            \textit{alas2} ng tanghali. (incorrect)
\end{enumerate}

\subsubsection{"Di-"}

Let \(w \in \mathbb{A} \ \cup \ \mathbb{N}\) be an element of the set of adjectives or nouns.
\[
      k = (\text{"di"}) \cdot "\text{-}" \cdot w
\]
The syllable "di" is a shortened format for the word "hindi", and it acts as a negation to the root word.
\begin{example}
\end{example}
\begin{enumerate}
      \item \textbf{di-doktor} (correct) \\
            \textit{didoktor} (incorrect)
      \item \textbf{di-tulog} (correct) \\
            \textit{hindi-tulog} (incorrect)
      \item \textbf{di-mainit} (correct) \\
            \textit{di mainit} (incorrect)
      \item \textbf{di-lalaki} (correct) \\
            \textit{dilalaki} (incorrect)
      \item \textbf{di-mabait} (correct) \\
            \textit{hindi-mabait} (incorrect)
\end{enumerate}

\subsubsection{Proper Nouns}
\label{proper_nouns_chapter}
Let \(j \in \mathbb{PN}\), where \(\mathbb{PN}\) is the set of proper nouns.
\[
      k = s \cdot "\text{-}" \cdot j
\]
\begin{example}
\end{example}
\begin{enumerate}
      \item \textbf{taga-Tondo} (correct) \\
            \textit{taga Tondo} (incorrect)
      \item \textbf{maka-Rizal} (correct) \\
            \textit{makaRizal} (incorrect)
      \item \textbf{pa-Palawan} (correct) \\
            \textit{papunta-palawan} (incorrect)
      \item \textbf{pa-DLSU} (correct) \\
            \textit{pa-dlsu} (incorrect)
      \item \textbf{ka-Davao} (correct) \\
            \textit{kadavao} (incorrect)
\end{enumerate}

\subsubsection{Foreign Words}
Let \(w \in \mathbb{FW}\), where \(\mathbb{FW}\) is the set of foreign words.
\[
      k = s \cdot "\text{-}" \cdot w
\]
\begin{example}
\end{example}
\begin{enumerate}
      \item \textbf{pa-message} (correct) \\
            \textit{pa message} (incorrect)
      \item \textbf{pa-cuenta} (correct) \\
            \textit{pa-Cuenta} (incorrect)
      \item \textbf{maki-Kpop} (correct) \\
            \textit{maki-K-pop} (incorrect)
      \item \textbf{maki-anime} (correct) \\
            \textit{makiAnime} (incorrect)
      \item \textbf{ipa-kanji} (correct) \\
            \textit{ipa kanji} (incorrect)
\end{enumerate}

\section{Selected Errors in the Bicol Language}
\subsection{Object-Focused Future Tense}
Any verb in the Bicol Language can be transformed into its Object-Focused Future Tense by adding the suffix:

\[
      \texttt{h}^?\texttt{on}
\]

In particular, the Kleene question mark is 1 if and only if the root word $r$ follows:

\[
      \backslash\texttt{b.}^*\texttt{[aeiou]}\backslash\texttt{b}
\]

\begin{example}
\end{example}

\begin{enumerate}
      \item \textbf{gibohon} (correct) \\
            \textit{bihoon} (incorrect)
      \item \textbf{babasahon} (correct) \\
            \textit{babasaon} (incorrect)
      \item \textbf{masasaboton} (correct) \\
            \textit{masasabothon} (incorrect)
      \item \textbf{DO THIS} (correct) \\
            \textit{DO THIS} (incorrect)
      \item \textbf{DO THIS} (correct) \\
            \textit{DO THIS} (incorrect)
\end{enumerate}

\subsection{U-O Distribution}

\subsubsection{U-O Distribution: Simple Case}
Any word in the Bicolano language that has an \texttt{o} sound that is not the last syllable is pronounced as \texttt{u}. Only if the \texttt{o} sound is the last syllable of the word, that sound is pronounced as \texttt{o}.

Formally, we can say that any Bicolano language word that contains an ‘o’ sound can be represented by the regular expression:

\[
      \left(\texttt{C}^?\texttt{uC}^?\right)^* \left(\texttt{C}^?\texttt{oC}^?\right)
\]

\begin{example}
\end{example}

\begin{enumerate}
      \item \textbf{Tukdo} (correct) \\
            \textit{Tokdo} or \textit{Tukdu} (incorrect) \\
      \item \textbf{Bulan} (correct) \\
            \textit{Bolan} (incorrect) \\
      \item \textbf{Ako} (correct) \\
            \textit{Aku} (incorrect) \\
      \item \textbf{Tukdo} (correct) \\
            \textit{Tokdo} or \textit{Tukdu} (incorrect) \\
      \item \textbf{Du'ot} (correct) \\
            \textit{Du'ut} or \textit{Do'ot} (incorrect) \\
\end{enumerate}

\subsubsection{U-O Distribution for Verbs in Object-Focused Future Tense}

Verbs with the $\texttt{h}^?on$ suffix or verbs in the object-focused future tense form, or any word of the form:

\[
      r \texttt{h}^? \texttt{on} \quad \text{where } r \text{ is the root word}
\]

The spelling of the root word $r$ will stay the same. That is, verbs at their object-focused future form may have an \texttt{o} sound at the second to the last suffix. Take for instance the word "gibohon" which has three syllables, notice that there is an \texttt{o} sound at the second syllable.

\begin{example}
\end{example}

\begin{enumerate}
      \item \textbf{Masasaboton} (correct) \\
            \textit{Masasabuton} (incorrect)
      \item \textbf{Lutohon} (correct) \\
            \textit{Lutuhon} (incorrect)
      \item \textbf{Herohon} (correct) \\
            \textit{Heruhon} (incorrect)
\end{enumerate}

\subsubsection{U-O Distribution for Reduplications}

Words that are a reduplication, or a joining of two syllables, can have the \texttt{u} sound at the end if and only if the reduplicated syllable does indeed have the \texttt{u} sound.

Formally, any word given by:
\[
      \left( \Sigma_\mathbb{B}^*\texttt{u}\Sigma_\mathbb{B}^* \right)\backslash1
\]

is a valid word in $\mathcal{B}$.

\begin{example}

\end{example}

\begin{enumerate}
      \item \textbf{Tugtug} (correct) \\
            \textit{Tugtog} (incorrect)
      \item \textbf{Gu'gu'} (correct)
            \\ \textit{Gu'go'} (incorrect)
      \item \textbf{Kutkut} (correct)
            \\ \textit{Kutkot} (incorrect)
\end{enumerate}


