\section{Preliminaries}
\subsection{The Filipino Alphabet}

Let \(\mathcal{F}\) be the Alphabet for the Filipino language, this alphabet is
composed of 56 scripts and 11 punctuation marks \cite{OOP}. The 56 scripts are divided into
two, the first half being the capital letters of the modern Latin script with
the addition of "Ñ" and "Ng"; while the other half is the lower case variants
of each letter.

The 11 punctuation marks in the Filipino language are the: \textit{tuldok} (.),
\textit{tandang pananong} (?), \textit{tandang padamdam} (!), \textit{kuwit} (,),
\textit{kudlit} ('), \textit{gitling} (-), \textit{tutuldok} (:),
\textit{tuldok-kuwit} (;), \textit{panipi} ("), \textit{pambukas na panaklong}
((), \textit{pampasarang panaklong} ()), at ang \textit{tutuldok-tuldok} (...)

In mathematical notation, we can represent \(\mathcal{F}\) as the set:

\[
    \mathcal{F} = \{\text{a},\text{b},\dots,\text{z},\text{ñ},\text{ng},   \
    \text{A},\text{B},\text{C},\dots,\text{Z},\text{Ñ},\text{Ng}\}         \
    \cup \{\text{.},\text{?},\text{!},\text{,},\text{'},\text{-},\text{:}, \
    \text{;},\text{"},\text{(},\text{)}, \ldots\}
\]

and the size of \(\mathcal{F}\), \(\left|\mathcal{F}\right| = 67\).

We can also introduce subsets the following which are subsets of \(\mathcal{F}\).
\begin{enumerate}
    \item \(\mathbb{M}\) = \{.,?,!,,,',-,:,;,",(,),...\}, the set of punctuation marks
    \item \(\mathbb{V}\) = \{a,e,i,o,u,A,E,I,O,I\}, the set of upper and lower case vowels
    \item \(\mathbb{C}\) = \(\mathcal{F} - (\mathbb{M}\cup\mathbb{V})\),
          the set of upper and lower case consonants
    \item \(\mathbb{V}_\text{upper}\) is the set of upper case vowels
    \item \(\mathbb{V}_\text{lower}\) is the set of lower case vowels
    \item \(\mathbb{C}_\text{upper}\) is the set of upper case consonants
    \item \(\mathbb{C}_\text{lower}\) is the set of lower case consonants
    \item \(\mathbb{L}\) = \(\mathcal{F} - \mathbb{M}\), the set of consonants and vowels
\end{enumerate}

\subsubsection{Remarks on the Digraph: \text{ng/Ng} or "\textit{en dyi}"}

Although the letter "Ng" or "ng" is a concatenation of two separate graphemes or
symbols in \(\mathcal{F}\) (since \(\text{Ng} = \text{N}\cdot\text{g}\) and
\(\text{ng} = \text{n}\cdot\text{g}\)), the letter "Ng" is officially recognized
as a symbol in \(\mathcal{F}\) since it represents a distinct Filipino sound.
In particular, it represents the voiced velar nasal sound, or in the International
Phonetic Alphabet (IPA), the \textipa{N} sound \cite{Malabonga_2009}.

For instance, the word "hangin" has 5 letters namely: "h", "a", "ng", "i", "n",
since "ng" is pronounced as a velar nasal sound, not as two separate sounds
\textipa{n-g}. So, "hangin" is pronounced as \textipa{haNin} ("ha-ngin").
Take for instance the English  word "manger" where "ng" is a substring
but is not pronounced as the velar nasal sound. Instead, its pronunciation is
\textipa{\textprimstress meI ndZ @r} ("meyn-jer"); not
\textipa{\textprimstress m\ae N Z @r} ("mang-jer"),
\textipa{\textprimstress m\ae N @r} ("mang-er") or
\textipa{\textprimstress m\ae N@r} ("manger").

\subsection{Common Errors}

\subsubsection{"es"-words and "is"-words}

Given the Spanish and English roots of Filipino, some \textit{loan} words have
rules for Filipino spelling. Let \(s\) be any string, the English language
\(\mathcal{E}\), the Spanish language \(\mathcal{S}\), and \(\mathcal{F}(s)\) is
the translation of \(s\) in \(\mathcal{F}\)

\begin{enumerate}
    \item \(\forall s\in \mathcal{S} \rightarrow \mathcal{F}(s) \models \text{es}\cdot
          (\mathbb{M}|\mathbb{C})* \)
    \item \(\forall s\in \mathcal{E} \rightarrow \mathcal{F}(s) \models \text{is}\cdot
          (\mathbb{M}|\mathbb{C})* \)
\end{enumerate}

Rule (1) denotes that if \(s\) is a Spanish word, translating \(s\) to a Filipino
word would use "es" as the prefix to the word to denote that \(\mathcal{F}(s)\)
is a word of Spanish origin. On the other hand, for rule (2), if \(s\) is an
English word, then \(\mathcal{F}(s)\) would use "is" as the prefix to the word
to denote that it is of English origin.

\begin{example}
    Here are examples of English and Spanish loan words in Filipino:
\end{example}
\begin{enumerate}
    \item Ako ay papasok sa \textbf{eskwelahan}. (correct) \\
          Ako ay papasok sa \textit{iskwelahan}. (incorrect) \\
          eskwelahan (\(\mathcal{F}\)) from escuela (\(\mathcal{S}\))
    \item Aba! Malaki pala ang \textbf{espasyo} rito. (correct) \\
          Aba! Malaki pala ang \textit{ispasyo} rito. (incorrect) \\
          espasyo (\(\mathcal{F}\)) from espacio (\(\mathcal{S}\))
    \item Marami kaming \textbf{estudyante} sa Computer Science. (correct) \\
          Marami kaming \textit{istudyante} sa Computer Science. (incorrect)\\
          estudyante (\(\mathcal{F}\)) from estudiante (\(\mathcal{S}\))
    \item Mahilig sila Turing at Sipser maglaro ng \textbf{eskrima}. (correct) \\
          Mahilig sila Turing at Sipser maglaro ng \textit{iskrima}. (incorrect) \\
          eskrima  (\(\mathcal{F}\)) from esgrima (\(\mathcal{S}\))
    \item Marami raw \textbf{espiritu} rito. (correct) \\
          Marami raw \textit{ispiritu} rito. (incorrect) \\
          espiritu (\(\mathcal{F}\)) from espiritu (\(\mathcal{S}\))
    \item Kinausap mo na ba iyung \textbf{ispiker}. (correct) \\
          Kinausap mo na ba iyung \textit{espiker}. (incorrect) \\
          ispiker (\(\mathcal{F}\)) from speaker (\(\mathcal{E}\))
    \item Marami naman daw \textbf{isports} na puwedeng pagpilian. (correct) \\
          Marami naman daw \textit{esports} na puwedeng pagpilian. (incorrect) \\
          isports (\(\mathcal{F}\)) from sports (\(\mathcal{E}\))
    \item Si Dijkstra ay hindi raw \textbf{iskolar}. (correct) \\
          Si Dijkstra ay hindi raw \textit{eskolar}. (incorrect) \\
          iskolar (\(\mathcal{F}\)) from scholar (\(\mathcal{E}\))
\end{enumerate}

\begin{example}
    Here are some examples of literature and webpages where the "es-" and "is-"
    prefixes are misused:
\end{example}

\begin{enumerate}
    \item \textbf{TODO}
\end{enumerate}

\subsubsection{"kump"-words and "kumb"-words}

If \(s\in \mathcal{S}\) and the the prefix of \(s\) is given by the
regular expression \((\text{C}|\text{c})\text{on}(\text{f}|\text{v})\). Then,
\(\mathcal{F}(s)\) is prefixed with
\((\text{C}|\text{c})\text{um}(\text{p}|\text{b})\).

In particular:
\begin{itemize}
    \item \(s \models (\text{conf})\mathbb{L}* \rightarrow \mathcal{F}(s) \models (\text{kump})\mathbb{L}*\)
    \item \(s \models (\text{conv})\mathbb{L}* \rightarrow \mathcal{F}(s) \models (\text{kumb})\mathbb{L}*\)
\end{itemize}

\begin{example}
    Here are examples of Spanish words in Filipino with the "kump-" and "kumb-"
    prefixes:
\end{example}
\begin{enumerate}
    \item Pumasok si Cormen sa \textbf{kumbento}. (correct) \\
          Pumasok si Cormen sa \textit{konbento}. (incorrect) \\
          kumbento (\(\mathcal{F}\)) from convento (\(\mathcal{S}\))
    \item Tinanong ko si Leiserson bilang \textbf{kumpirmasyon}. (correct) \\
          Tinanong ko si Leiserson bilang \textit{kunpirmasyon}. (incorrect) \\
          kumpirmasyon (\(\mathcal{F}\)) from confirmacion (\(\mathcal{S}\))
    \item Ang \textbf{kumpetisyon} sa agham ay ginanap sa paaralan. (correct) \\
          Ang \textit{kunpetisyon} sa agham ay ginanap sa paaralan. (incorrect) \\
          kumpetisyon (\(\mathcal{F}\)) from competición (\(\mathcal{S}\))
    \item Nagbigay siya ng \textbf{kumpisal} sa pari noong Linggo. (correct) \\
          Nagbigay siya ng \textit{kunpisal} sa pari noong Linggo. (incorrect) \\
          kumpisal (\(\mathcal{F}\)) from confesión (\(\mathcal{S}\))
    \item Sumali ako sa isang \textbf{kumbento} sa simbahan natin. (correct) \\
          Sumali ako sa isang \textit{kunbento} sa simbahan natin. (incorrect) \\
          kumbento (\(\mathcal{F}\)) from convento (\(\mathcal{S}\))
    \item Hindi ko alam ang \textbf{kumbinasyon} sa Graph Coloring. (correct) \\
          Hindi ko alam ang \textit{kunbinasyon} sa Graph Coloring. (correct) \\
          kumbinasyon (\(\mathcal{F}\)) from combinación (\(\mathcal{S}\))
    \item Ipinahayag niya ang kanyang \textbf{kumbersasyon} sa kanila. (correct) \\
          Ipinahayag niya ang kanyang \textit{kunbersasyon} sa . (incorrect) \\
          kumbersasyon (\(\mathcal{F}\)) from conversación (\(\mathcal{S}\))
    \item Ang patakaran ay dumaan sa \textbf{kumpigurasyon} bago ipatupad. (correct) \\
          Ang patakaran ay dumaan sa \textit{kunpigurasyon} bago ipatupad. (incorrect) \\
          kumpigurasyon (\(\mathcal{F}\)) from configuración (\(\mathcal{S}\))
\end{enumerate}

\begin{example}
    Here are some examples of literature and webpages where the "kump-" and "kumb-"
    prefixes are misused:
\end{example}
\begin{enumerate}
    \item \textbf{TODO}
\end{enumerate}

\subsubsection{Morphophonemic Alteration with Suffixes}

For a \(k \in \mathcal{F}\) that is constructed with  the concatenation
\(k = \alpha \cdot \text{root}_\text{allomorph} \cdot \omega \)
for \(\alpha\) is the prefix of \(k\), \(\omega\) is the suffix of \(k\),
and \(\text{root}_\text{allomorph}\) is the root allomorph word of \(k\).

The morphophonemic alteration of the root word is given by:
\begin{enumerate}
    \item \(\text{root} \models \mathbb{L}+e \rightarrow
          \text{root}_\text{allomorph} \models \mathbb{L}+i \)
    \item \(\text{root} \models \mathbb{L}+o \rightarrow
          \text{root}_\text{allomorph} \models \mathbb{L}+u \)
\end{enumerate}

That is to say, that if a Filipino word is the concatenation of a root word that
ends in "e" or "o" and a suffix. Then, "e" will change to "i" and "o" will change
to "u" \cite{Yap_1967}.

\begin{example}
    Here are examples of morphophonemic alterations in Filipino:
\end{example}
\begin{enumerate}
    \item \textbf{Tinakbuhan} niya ang kanyang problema. \\
          takbo \(\rightarrow\) takbu \(\cdot\) han \(\rightarrow\) tinakbuhan
    \item \textbf{Pinasukan} niya ang silid-aralan. \\
          pasok \(\rightarrow\) pasuk \(\cdot\) an \(\rightarrow\) pinasukan
    \item \textbf{Pinayuhan} siya ng kanyang guro. \\
          payo \(\rightarrow\) payu \(\cdot\) han \(\rightarrow\) pinayuhan
    \item \textbf{Babaihan} ang tawag sa lugar kung saan maraming babae. \\
          babae \(\rightarrow\) babai \(\cdot\) han \(\rightarrow\) babaíhan
    \item \textbf{Biniruan} niya ang kanyang kaibigan. \\
          biro \(\rightarrow\) biru \(\cdot\) an \(\rightarrow\) biniruan
    \item Gusto naming \textbf{laruin} ni Knuth iyan! \\
          laro  \(\rightarrow\) laru \(\cdot\) in \(\rightarrow\) laruin
    \item Tinanong ko siya kung hahaluin iyong pagkain... \\
          halo  \(\rightarrow\) halu \(\cdot\) in \(\rightarrow\) haluin
    \item Kakayanin kitang \textbf{talunin} sa Catan! \\
          talo  \(\rightarrow\) talu \(\cdot\) nin  \(\rightarrow\) talunin
\end{enumerate}


\begin{example}
    Here are some examples of literature and webpages where that have the incorrect
    root allomorph:
\end{example}
\begin{enumerate}
    \item \textbf{TODO}
\end{enumerate}

\subsubsection{Hyphenated Reduplications}

For a \(k \in \mathcal{F} \text{ such that } k\models b\text{\textbackslash-} r\)

for \(b\) is the base word of \(k\) and \(r\) is the reduplicant of \(b\). The
reduplicant \(r\) is given by:

\begin{enumerate}
    \item \(b\models \mathbb{L}+e \rightarrow r\models\mathbb{L}+i\)
    \item \(b\models \mathbb{L}+o \rightarrow r\models\mathbb{L}+u\)
\end{enumerate}

Here are some examples of literature and webpages where the wrong reduplicant is
used:
\begin{enumerate}
    \item \textbf{TODO}
\end{enumerate}

\subsubsection{Raw v. Daw (Enclitic Particles); Rin v. Din}

Let \(p = "\,"\) the blank symbol or the symbol containing space. And, we have the
sentence structure \(S = \alpha + p + EP\). The \textit{enclitic particles}
(\(\mathbb{EP}\)) is the set \(\mathbb{EP} = \{"\text{raw}", "\text{daw}"\}\) and
\(\alpha\) is any noun, adjective, verb, or adverb. The proper usage of the enclitic
particles are given by:

\[
    \left(\alpha \models \left[\text{a-zñ(ng)A-ZÑ(Ng)}\right]+\left[\text{aeiou}\right] \equiv (\mathbb{V}\cup\mathbb{C}) + \mathbb{V}_\text{lower}\right)
    \longrightarrow EP = \text{"raw"}
\]

Otherwise,

\[
    EP = \text{"daw"}
\]

In other words, if the preceeding word to the enclitic particle ends in vowel,
then the enclitic particle is "raw". If it is a consonant, then it is "daw".

This is the same idea with the adverbs "rin" and "din". If the sentence structure
\(S = \omega + p + (\text{"rin"}|\text{"din"})\) and \(\omega\) ends in a vowel
then the adverb used is "rin". Otherwise, the adverb is "din".

Here are some examples of literature and webpages where the enclitic particles,
and the adverbs "raw" and "daw" are misused:
\begin{enumerate}
    \item \textbf{TODO}
\end{enumerate}

\subsubsection{Ng v. Nang}

\subsubsection{\textit{Gitling} Usage}