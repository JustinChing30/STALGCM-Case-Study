\section{Preliminaries}
\subsection{The Filipino Alphabet}

Let \(\mathcal{F}\) the Filipino language
and \(\mathbb{F}\) be the Alphabet of \(\mathcal{F}\), this alphabet is
composed of 56 scripts and 11 punctuation marks \cite{OOP}. The 56 scripts are divided into
two, the first half being the capital letters of the modern Latin script with
the addition of "Ñ" and "Ng"; while the other half is the lower case variants
of each letter.

The 11 punctuation marks in the Filipino language are the: \textit{tuldok} (.),
\textit{tandang pananong} (?), \textit{tandang padamdam} (!), \textit{kuwit} (,),
\textit{kudlit} ('), \textit{gitling} (-), \textit{tutuldok} (:),
\textit{tuldok-kuwit} (;), \textit{panipi} ("), \textit{pambukas na panaklong}
((), \textit{pampasarang panaklong} ()), at ang \textit{tutuldok-tuldok} (...)

In mathematical notation, we can represent \(\mathbb{F}\) as the set:

\[
      \mathbb{F} = \{\text{a},\text{b},\dots,\text{z},\text{ñ},\text{ng},   \
      \text{A},\text{B},\text{C},\dots,\text{Z},\text{Ñ},\text{Ng}\}         \
      \cup \{\text{.},\text{?},\text{!},\text{,},\text{'},\text{-},\text{:}, \
      \text{;},\text{"},\text{(},\text{)}, \ldots\}
\]

and the size of \(\mathbb{F}\), \(\left|\mathbb{F}\right| = 67\).

From this we can introduce the following sets:
\begin{enumerate}
      \item \(\mathbb{M}\) = \{.,?,!,,,',-,:,;,",(,),...\}, the set of punctuation marks
      \item \(\mathbb{V}\) = \{a,e,i,o,u,A,E,I,O,I\}, the set of upper and lower case vowels
      \item \(\mathbb{C}\) = \(\mathcal{F} - (\mathbb{M}\cup\mathbb{V})\),
            the set of upper and lower case consonants
      \item \(\mathbb{V}_\text{upper}\) is the set of upper case vowels
      \item \(\mathbb{V}_\text{lower}\) is the set of lower case vowels
      \item \(\mathbb{C}_\text{upper}\) is the set of upper case consonants
      \item \(\mathbb{C}_\text{lower}\) is the set of lower case consonants
      \item \(\mathbb{L}\) = \(\mathbb{F} - \mathbb{M}\), is the set of consonants and vowels
      \item \(\mathbb{L}*\), is the set of words in \(\mathcal{F}\)
\end{enumerate}

\paragraph{Remark on the Digraph: \text{ng/Ng} or "\textit{en dyi}"}

Although the letter "Ng" or "ng" is a concatenation of two separate graphemes or
symbols in \(\mathbb{F}\) (since \(\text{Ng} = \text{N}\cdot\text{g}\) and
\(\text{ng} = \text{n}\cdot\text{g}\)), the letter "Ng" is officially recognized
as a grapheme in \(\mathbb{F}\) since it represents a distinct Filipino sound.
In particular, it represents the \textit{voiced velar nasal sound}, or in the International
Phonetic Alphabet (IPA), the \textipa{N} sound \cite{Malabonga_2009}.

For instance, the word "hangin" has 5 letters namely: "h", "a", "ng", "i", "n",
since "ng" is pronounced as a velar nasal sound, not as two separate sounds
\textipa{n-g}. So, "hangin" is pronounced as \textipa{haNin} ("ha-ngin").
Take for instance the English  word "manger" where "ng" is a substring
but is not pronounced as the velar nasal sound. Instead, its pronunciation is
\textipa{\textprimstress meI ndZ @r} ("meyn-jer"); not
\textipa{\textprimstress m\ae N Z @r} ("mang-jer"),
\textipa{\textprimstress m\ae N @r} ("mang-er") or
\textipa{\textprimstress m\ae N@r} ("manger").

\subsection{Common Errors}

\subsubsection{"es"-words and "is"-words}

Given the Spanish and English roots of Filipino, some \textit{loan} words have
rules for Filipino spelling. Let \(s\) be any string, the English language
\(\mathcal{E}\), the Spanish language \(\mathcal{S}\), and \(\mathcal{F}(s)\) is
the translation of \(s\) in \(\mathcal{F}\). Furthermore, let \(\mathbb{S}*\) be
the set of words in \(\mathcal{S}\) and \(\mathbb{E}*\) be the set of words in
\(\mathcal{E}\).

\begin{enumerate}
      \item \(\left(\forall s\in \mathbb{S}*,s \matches \text{es}(\mathbb{S}*) \right) \rightarrow \left(\mathcal{F}(s) \matches \text{es}
                  (\mathbb{M}|\mathbb{C})*\right) \)
      \item \(\left(\forall s\in \mathbb{E}*,s \matches \text{s}(\mathbb{E}*)\right) \rightarrow  \left(\mathcal{F}(s) \matches \text{is}
                  (\mathbb{M}|\mathbb{C})* \right)\)
\end{enumerate}

Rule (1) denotes that if \(s\) is a Spanish word, translating \(s\) to a Filipino
word would use "es" as the prefix to the word to denote that \(\mathcal{F}(s)\)
is a word of Spanish origin. On the other hand, for rule (2), if \(s\) is an
English word, then \(\mathcal{F}(s)\) would use "is" as the prefix to the word
to denote that it is of English origin.

\begin{example}
      Here are examples of English and Spanish loan words in Filipino:
\end{example}
\begin{enumerate}
      \item Ako ay papasok sa \textbf{eskwelahan}. (correct) \\
            Ako ay papasok sa \textit{iskwelahan}. (incorrect) \\
            eskwelahan (\(\mathcal{F}\)) from escuela (\(\mathcal{S}\))
      \item Aba! Malaki pala ang \textbf{espasyo} rito. (correct) \\
            Aba! Malaki pala ang \textit{ispasyo} rito. (incorrect) \\
            espasyo (\(\mathcal{F}\)) from espacio (\(\mathcal{S}\))
      \item Marami kaming \textbf{estudyante} sa Computer Science. (correct) \\
            Marami kaming \textit{istudyante} sa Computer Science. (incorrect)\\
            estudyante (\(\mathcal{F}\)) from estudiante (\(\mathcal{S}\))
      \item Mahilig sila Turing at Sipser maglaro ng \textbf{eskrima}. (correct) \\
            Mahilig sila Turing at Sipser maglaro ng \textit{iskrima}. (incorrect) \\
            eskrima  (\(\mathcal{F}\)) from esgrima (\(\mathcal{S}\))
      \item Marami raw \textbf{espiritu} rito. (correct) \\
            Marami raw \textit{ispiritu} rito. (incorrect) \\
            espiritu (\(\mathcal{F}\)) from espiritu (\(\mathcal{S}\))
      \item Kinausap mo na ba iyung \textbf{ispiker}. (correct) \\
            Kinausap mo na ba iyung \textit{espiker}. (incorrect) \\
            ispiker (\(\mathcal{F}\)) from speaker (\(\mathcal{E}\))
      \item Marami naman daw \textbf{isports} na puwedeng pagpilian. (correct) \\
            Marami naman daw \textit{esports} na puwedeng pagpilian. (incorrect) \\
            isports (\(\mathcal{F}\)) from sports (\(\mathcal{E}\))
      \item Si Dijkstra ay hindi raw \textbf{iskolar}. (correct) \\
            Si Dijkstra ay hindi raw \textit{eskolar}. (incorrect) \\
            iskolar (\(\mathcal{F}\)) from scholar (\(\mathcal{E}\))
\end{enumerate}

\begin{example}
      Here are some examples of literature and webpages where the "es-" and "is-"
      prefixes are misused:
\end{example}

\begin{enumerate}
      \item \textbf{TODO}
\end{enumerate}

\subsubsection{"kump"-words and "kumb"-words}

If \(s\in \mathbb{S}*\) and the the prefix of \(s\) is given by the
regular expression \((\text{C}|\text{c})\text{on}(\text{f}|\text{v})\). Then,
\(\mathcal{F}(s)\) is prefixed with
\((\text{C}|\text{c})\text{um}(\text{p}|\text{b})\).

In particular:
\begin{itemize}
      \item \(s \matches (\text{conf})\mathbb{S}* \rightarrow \mathcal{F}(s) \matches (\text{kump})\mathbb{L}*\)
      \item \(s \matches (\text{conv})\mathbb{S}* \rightarrow \mathcal{F}(s) \matches (\text{kumb})\mathbb{L}*\)
\end{itemize}

\begin{example}
      Here are examples of Spanish words in Filipino with the "kump" and "kumb"
      prefixes:
\end{example}
\begin{enumerate}
      \item Pumasok si Cormen sa \textbf{kumbento}. (correct) \\
            Pumasok si Cormen sa \textit{konbento}. (incorrect) \\
            kumbento (\(\mathcal{F}\)) from convento (\(\mathcal{S}\))
      \item Tinanong ko si Leiserson bilang \textbf{kumpirmasyon}. (correct) \\
            Tinanong ko si Leiserson bilang \textit{kunpirmasyon}. (incorrect) \\
            kumpirmasyon (\(\mathcal{F}\)) from confirmacion (\(\mathcal{S}\))
      \item Ang \textbf{kumpetisyon} sa agham ay ginanap sa paaralan. (correct) \\
            Ang \textit{kunpetisyon} sa agham ay ginanap sa paaralan. (incorrect) \\
            kumpetisyon (\(\mathcal{F}\)) from competición (\(\mathcal{S}\))
      \item Nagbigay siya ng \textbf{kumpisal} sa pari noong Linggo. (correct) \\
            Nagbigay siya ng \textit{kunpisal} sa pari noong Linggo. (incorrect) \\
            kumpisal (\(\mathcal{F}\)) from confesión (\(\mathcal{S}\))
      \item Sumali ako sa isang \textbf{kumbento} sa simbahan natin. (correct) \\
            Sumali ako sa isang \textit{kunbento} sa simbahan natin. (incorrect) \\
            kumbento (\(\mathcal{F}\)) from convento (\(\mathcal{S}\))
      \item Hindi ko alam ang \textbf{kumbinasyon} sa Graph Coloring. (correct) \\
            Hindi ko alam ang \textit{kunbinasyon} sa Graph Coloring. (correct) \\
            kumbinasyon (\(\mathcal{F}\)) from combinación (\(\mathcal{S}\))
      \item Ipinahayag niya ang kanyang \textbf{kumbersasyon} sa kanila. (correct) \\
            Ipinahayag niya ang kanyang \textit{kunbersasyon} sa . (incorrect) \\
            kumbersasyon (\(\mathcal{F}\)) from conversación (\(\mathcal{S}\))
      \item Ang patakaran ay dumaan sa \textbf{kumpigurasyon} bago ipatupad. (correct) \\
            Ang patakaran ay dumaan sa \textit{kunpigurasyon} bago ipatupad. (incorrect) \\
            kumpigurasyon (\(\mathcal{F}\)) from configuración (\(\mathcal{S}\))
\end{enumerate}

\begin{example}
      Here are some examples of literature and webpages where the "kump" and "kumb"
      prefixes are misused:
\end{example}
\begin{enumerate}
      \item \textbf{TODO}
\end{enumerate}

\subsubsection{Morphophonemic Alteration with Suffixes}

Let \(k \in \mathbb{L}*\) be a word formed by the concatenation as follows
\(k = \alpha \cdot \text{root}_\text{allomorph} \cdot \omega \)
where  \(\alpha\) is the prefix of \(k\), \(\omega\) is the suffix of \(k\),
and \(\text{root}_\text{allomorph}\) is the root allomorph of \(k\).

The morphophonemic alteration of the root word is given by:
\begin{center}
      If the root ends in "e", then \(\text{root}_\text{allomorph}\) ends in "i". \\
      If the root ends in "o", then the \(\text{root}_\text{allomorph}\) ends in "u".
\end{center}

That is to say, that if a Filipino word is the concatenation of a root word that
ends in "e" or "o" and a suffix. Then, "e" will change to "i" and "o" will change
to "u" \cite{Yap_1967}.

\begin{example}
      Here are examples of morphophonemic alterations in Filipino:
\end{example}
\begin{enumerate}
      \item \textbf{Tinakbuhan} niya ang kanyang problema. \\
            takbo \(\rightarrow\) takbu \(\cdot\) han \(\rightarrow\) tinakbuhan
      \item \textbf{Pinasukan} niya ang silid-aralan. \\
            pasok \(\rightarrow\) pasuk \(\cdot\) an \(\rightarrow\) pinasukan
      \item \textbf{Pinayuhan} siya ng kanyang guro. \\
            payo \(\rightarrow\) payu \(\cdot\) han \(\rightarrow\) pinayuhan
      \item \textbf{Babaihan} ang tawag sa lugar kung saan maraming babae. \\
            babae \(\rightarrow\) babai \(\cdot\) han \(\rightarrow\) babaíhan
      \item \textbf{Biniruan} niya ang kanyang kaibigan. \\
            biro \(\rightarrow\) biru \(\cdot\) an \(\rightarrow\) biniruan
      \item Gusto naming \textbf{laruin} ni Knuth iyan! \\
            laro  \(\rightarrow\) laru \(\cdot\) in \(\rightarrow\) laruin
      \item Tinanong ko siya kung hahaluin iyong pagkain... \\
            halo  \(\rightarrow\) halu \(\cdot\) in \(\rightarrow\) haluin
      \item Kakayanin kitang \textbf{talunin} sa Catan! \\
            talo  \(\rightarrow\) talu \(\cdot\) nin  \(\rightarrow\) talunin
\end{enumerate}


\begin{example}
      Here are some examples of literature and webpages where that have the incorrect
      root allomorph:
\end{example}
\begin{enumerate}
      \item \textbf{TODO}
\end{enumerate}

\subsubsection{Reduplications}

A full reduplication is a word \(k \in \mathbb{L}*\) and is defined by the concatenation
\[
      k = "\text{b}" \cdot "\text{-}" \cdot "\text{b}"
\]
or it is a repetition of a root word \(b\) joined by the \textit{gitling} "-".
In a full reduplication, the repetition of the \(b\) has no variation from the first \(b\).

On the otherhand, an ablaut reduplication is a word \(j \in \mathbb{L}*\) and is defined by the
concatenation

\[
      j = "\text{x}" \cdot "\text{r}"
\]
or it is the repetition of the root word \(x\) and the root word with a vowel alteration \(r\). Notice
that for ablaut repetitions, there is no \textit{gitling} joining the repetitions. Furthermore,
the ablaut repetition of a root word creates a word with a different meaning. The ablaut or vowel
alteration is defined by:

\begin{center}
      If the root ends in "e", then the ablaut ends in "i". \\
      If the root ends in "o", then the ablaut ends in "u".
\end{center}

\begin{example}
      Here are examples of reduplications in Filipino:
\end{example}
\begin{enumerate}
      \item Kung \textbf{ano-ano} ginawa ni Hopcroft kanina. (correct) \\
            Kung \textit{anu-ano} ginawa ni Hopcroft kanina. (incorrect) \\
            ano-ano is a full reduplication
      \item Pupunta kami mamaya sa \textbf{sari-sari} store. (correct) \\
            Pupunta kami mamaya sa \textit{sarisari} store. (incorrect) \\
            sari-sari is a full reduplication
      \item Gagawin natin iyong \textbf{kanya-kanyang} gawain mamaya. (correct) \\
            Gagawin natin iyong \textit{kanyakanyang} gawain mamaya. (incorrect) \\
            kanya-kanya is a full reduplication
      \item Paulit-ulit ka nang tanong! (correct) \\
            Paulet-ulit ka nang tanong! (incorrect) \\
            ulit-ulit is a full reduplication
      \item Bumili kami ng \textbf{haluhalo} kaninang umaga. (correct) \\
            Bumili kami ng \textit{halo-halo} kaninang umaga. (incorrect) \\
            haluhalo is an ablaut reduplication
\end{enumerate}

\begin{example}
      Here are some examples of literature and webpages where the wrong reduplicant is
      used:
\end{example}
\begin{enumerate}
      \item \textbf{TODO}
\end{enumerate}

\subsubsection{Raw v. Daw (Enclitic Particles); Rin v. Din}

Let \(p = "\,"\) the blank symbol or the symbol containing space. And, we have the
sentence structure \(S = \alpha + p + EP\). The \textit{enclitic particles}
(\(\mathbb{EP}\)) is the set \(\mathbb{EP} = \{"\text{raw}", "\text{daw}"\}\) and
\(\alpha\) is any noun, adjective, verb, or adverb. The proper usage of the enclitic
particles are given by:

\[
      \left(\alpha \matches \left[\text{a-zñ(ng)A-ZÑ(Ng)}\right]+\left[\text{aeiou}\right] \equiv (\mathbb{V}\cup\mathbb{C}) + \mathbb{V}_\text{lower}\right)
      \longrightarrow EP = \text{"raw"}
\]

Otherwise,

\[
      EP = \text{"daw"}
\]

In other words, if the preceeding word to the enclitic particle ends in vowel,
then the enclitic particle is "raw". If it is a consonant, then it is "daw".

This is the same idea with the adverbs "rin" and "din". If the sentence structure
\(S = \omega + p + (\text{"rin"}|\text{"din"})\) and \(\omega\) ends in a vowel
then the adverb used is "rin". Otherwise, the adverb is "din".

Here are some examples of literature and webpages where the enclitic particles,
and the adverbs "raw" and "daw" are misused:
\begin{enumerate}
      \item \textbf{TODO}
\end{enumerate}

\subsubsection{Ng v. Nang}

There are five way to use "nang", for the sentence structures below
let \(p = "\,"\) the blank symbol or the symbol containing space.

\paragraph{First case:} the string "nang" is used before an adverb when the
structure of any sentence S is given by:

\begin{center}
      (S\textsubscript{1} = v \(\cdot\) p \(\cdot\) pron* \(\cdot\) p \(\cdot\)
      "nang" \(\cdot\) p \(\cdot\) adv)
\end{center}

Where v is a verb, adv is an adverb, and pron is a pronoun.

\begin{example}
      Example sentences for the first case:
\end{example}
\begin{itemize}
      \item Kumain nang mabagal.
      \item             Kumain siya nang mabagal.

\end{itemize}

\paragraph{Second case:} "Nang" can be used to replace the conjunction "noon" +
ligature "-ng".

\begin{center}
      (S\textsubscript{2} = "noong" \(\cdot\) (v {\textpipe} adv))
\end{center}

\begin{example}
      Example sentences for the second case:
\end{example}

\begin{itemize}
      \item (Noong {\textpipe} Nang) pumunta siya sa mall…
      \item (Noong {\textpipe} Nang) maaraw na araw…
\end{itemize}

\paragraph{Third case:} "Nang" can be used to replace a combined Enclictic Particle
"na" + ligature "-ng".

\begin{center}
      (S\textsubscript{3} = v \(\cdot\) p \(\cdot\) "na" \(\cdot\) p \(\cdot\)
      "-ng")
\end{center}

\begin{example}
      Example sentences for the third case:
\end{example}
\begin{itemize}
      \item Ang mamatay na ng dahil sa iyo.
      \item Ang mamatay nang dahil sa iyo.
\end{itemize}

\paragraph{Fourth case:} "Nang" can be used to replace "upang" or "para".
\begin{center}
      (S\textsubscript{4} = "nang" \(\cdot\) p \(\cdot\) v)
\end{center}

\begin{example}
      Examples sentences for the fourth case:
\end{example}
\begin{itemize}
      \item (from \cite{OOP}). …kailangang bitayin si Rizal (para {\textpipe} nang)
            matakot ang mga Filipino.
      \item (from \cite{OOP}). Dinala si Pedro sa ospital (para {\textpipe} nang)
            magamot.
\end{itemize}

\paragraph{Fifth case:} "Nang" can be used in between repeated verbs.
\begin{center}
      (S\textsubscript{5} = v\textsubscript{1} \(\cdot\) p \(\cdot\) "nang"
      \(\cdot\) p \(\cdot\) v\textsubscript{1})
\end{center}

\begin{example}
      Example sentence for the fifth case:
\end{example}
\begin{itemize}
      \item Kain nang kain...
\end{itemize}

Literature on Filipino grammar and style highlights that "nang" should not be
mistaken as the contraction of the words "na" + "ang" and that the correct
contraction for them is "na'ng" \cite{OOP}.

\paragraph{"Ng" and nouns.} The string "ng" is used to denote possession similar to
the word "of" in English when the structure of any sentence S is given by:

\begin{center}
      (S\textsubscript{1} = n \(\cdot\) p \(\cdot\) "ng" \(\cdot\) p \(\cdot\) n)
\end{center}

Where n is a noun. \\

\begin{example}
      Examples for "ng" as a linker to indicate possession:
\end{example}
\begin{itemize}
      \item Bag ng bata
      \item Pinto ng bahay
      \item Hawakan ng pinto
\end{itemize}

\paragraph{"Ng" and object of actions.}"Ng" can be used to mark the direct object
of an action in a sentence.
\begin{center}
      (S\textsubscript{2} = v \(\cdot\) p \(\cdot\) "ng" \(\cdot\) p \(\cdot\) n)
\end{center}

\begin{example}
      Examples for "ng" to indicate the object of the verb.
\end{example}
\begin{itemize}
      \item Binalat at kinain ng bata ang saging.
      \item Tinapon ng basurero...
      \item Inalis ng lola...
\end{itemize}

\subsubsection{\textit{Gitling} Usage}
The usage of a \textit{gitling} (hyphen in English) in the Filipino language is quite common and can often lead to its improper use. One of the proper ways to use the \textit{gitling} was discussed earlier in 1.2.4. This section will primarily focus on the proper usage of the \textit{gitling}.

\begin{enumerate}
    \item 
        Let \(k \in \mathbb{L}^*\) be a reduplication. And the root word structure
        \\ \(b = s_0\cdots s_n\), where \(s_n\) are syllables.
        
        \[
            k = b \cdot "\text{-}" \cdot b
        \]

        or

        \[
            k = (s_0) \cdot "\text{-}" \cdot (s_0)
        \]
        
        For reduplications with $|k|_{\text{count}}>$ 2 (syllable count). Then it follows the form:
        
        \[
            k = (s_0 \cdot s_1) \cdot "\text{-}" \cdot (s_0 \cdots s_n)
        \]
        


    \item
        Let \( k \in \mathbb{L}^* \) be an onomatopoeic word, where \( s_n \in \mathbb{O} \) represents an \textit{onomatopoeic syllable} and \( \mathbb{O} \) is the set of all such syllables. 
    
        The general form for an \textbf{onomatopoeic word with reduplication} is given by:
        \[
            k = s_0 \cdot "\text{-}" \cdot s_1 \cdot "\text{-}" \cdot s_n \cdots
        \]
        where \( n \geq 1 \), and each \( s_n \) is a single-syllable onomatopoeic unit. The word consists of at least two such syllables, separated by a \textit{gitling} (hyphen).

\end{enumerate}