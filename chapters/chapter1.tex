\section{Preliminaries}
\subsection{The Filipino Alphabet}

Let \(\mathcal{F}\) the Filipino language
and \(\mathbb{F}\) be the Alphabet of \(\mathcal{F}\), this alphabet is
composed of 56 scripts and 11 punctuation marks \cite{OOP}. The 56 scripts are divided into
two, the first half being the capital letters of the modern Latin script with
the addition of \texttt{Ñ} and \texttt{Ng}; while the other half is the lower case variants
of each letter.

The 11 punctuation marks in the Filipino language are the: \textit{tuldok} (.),
\textit{tandang pananong} (?), \textit{tandang padamdam} (!), \textit{kuwit} (,),
\textit{kudlit} ('), \textit{\textit{gitling}} (-), \textit{tutuldok} (:),
\textit{tuldok-kuwit} (;), \textit{panipi} ("), \textit{pambukas na panaklong}
((), \textit{pampasarang panaklong} ()), at ang \textit{tutuldok-tuldok} (...)

In mathematical notation, we can represent \(\mathbb{F}\) as the set:

\[
      \mathbb{F} = \{\texttt{a},\texttt{b},\dots,\texttt{z},\texttt{ñ},\texttt{ng},   \
      \texttt{A},\texttt{B},\texttt{C},\dots,\texttt{Z},\texttt{Ñ},\texttt{Ng}\}         \
      \cup \{\texttt{.},\texttt{?},\texttt{!},\texttt{,},\texttt{'},\texttt{-},\texttt{:}, \
      \texttt{;},\texttt{"},\texttt{(},\texttt{)}, \texttt{...}\}
\]

and the size of \(\mathbb{F}\), \(\mid\mathbb{F}\mid = 67\) and the set of all possible Filipino words is $\mathcal{F}$. Furthermore, from this, we can introduce the following sets:
\begin{enumerate}
      \item \(\mathbb{M}\) = \{\texttt{.}, \texttt{?}, \texttt{!}, \texttt{,}, \texttt{,}, \texttt{'}, \texttt{-}, \texttt{:}, \texttt{;}, \texttt{"}, \texttt{(}, \texttt{)}, \texttt{...}\}, the set of punctuation marks.
      \item \(\mathbb{V}\) = \{\texttt{a}, \texttt{e}, \texttt{i}, \texttt{o}, \texttt{u}, \texttt{A}, \texttt{E}, \texttt{I}, \texttt{O}, \texttt{U}\}, the set of upper and lower case vowels.
      \item \(\mathbb{C}\) = \(\mathbb{F} - (\mathbb{M} \cup \mathbb{V})\),
            the set of upper and lower case consonants.
      \item \(\Sigma\) = \(\mathbb{F} - \mathbb{M}\), the set of consonants and vowels.
      \item The subscript "lower" on \(\mathbb{C}, \mathbb{V}, \Sigma\) denotes the set of lowercase letters.
      \item The subscript "upper" on \(\mathbb{C}, \mathbb{V}, \Sigma\) denotes the set of uppercase letters.
\end{enumerate}


\paragraph{Remark on the Digraph: \texttt{ng}/\texttt{Ng} or "\textit{en dyi}"}

Although the letter "Ng" or "ng" is a concatenation of two separate graphemes or
symbols in \(\mathbb{F}\) (since \(\texttt{Ng} = \texttt{N}\cdot\texttt{g}\) and
\(\texttt{ng} = \texttt{n}\cdot\texttt{g}\)), the letter \texttt{Ng} is officially recognized
as a grapheme in \(\mathbb{F}\) since it represents a distinct Filipino sound.
In particular, it represents the \textit{voiced velar nasal sound}, or in the International
Phonetic Alphabet (IPA), the \textipa{N} sound \cite{Malabonga_2009}.

For instance, the word "hangin" has 5 letters namely: \texttt{h}, \texttt{a}, \texttt{g}, \texttt{i}, \texttt{n},
since \texttt{ng} is pronounced as a velar nasal sound, not as two separate sounds
\textipa{n-g}. So, "hangin" is pronounced as \textipa{haNin} ("ha-ngin").
Take for instance the English  word "manger" where "ng" is a substring
but is not pronounced as the velar nasal sound. Instead, its pronunciation is
\textipa{\textprimstress meI ndZ @r} ("meyn-jer"); not
\textipa{\textprimstress m\ae N Z @r} ("mang-jer"),
\textipa{\textprimstress m\ae N @r} ("mang-er") or
\textipa{\textprimstress m\ae N@r} ("manger").

\subsection{Common Errors}

\subsubsection{"es"-words and "is"-words}

Given the Spanish and English roots of Filipino, some \textit{loan} words have
rules for Filipino spelling. Let \(s\) be any string, the English language
\(\mathcal{E}\), the Spanish language \(\mathcal{S}\), and \(\mathcal{F}(s)\) is
the translation of \(s\) in \(\mathcal{F}\).

\begin{enumerate}
      \item \(\left(\forall s\in \mathcal{S},s \matches \texttt{es}\mathbb{S}^*\right) \Longrightarrow \left(\mathcal{F}(s) \matches \texttt{es}\Sigma_\text{lower}^*\right) \)
      \item \(\left(\forall s\in \mathcal{E},s \matches \texttt{s}\mathbb{E}^*\right) \Longrightarrow  \left(\mathcal{F}(s) \matches \texttt{is} \Sigma_\text{lower}^* \right)\)
\end{enumerate}

Rule (1) denotes that if \(s\) is a Spanish word, translating \(s\) to a Filipino
word would use \texttt{es} as the prefix to the word to denote that \(\mathcal{F}(s)\)
is a word of Spanish origin. On the other hand, for rule (2), if \(s\) is an
English word, then \(\mathcal{F}(s)\) would use \texttt{is} as the prefix to the word
to denote that it is of English origin.

\begin{example}
      Here are examples of English and Spanish loanwords in Filipino:
\end{example}
\begin{enumerate}
      \item Ako ay papasok sa \textbf{eskwelahan}. (correct) \\
            Ako ay papasok sa \textit{iskwelahan}. (incorrect) \\
            eskwelahan (\(\mathcal{F}\)) from escuela (\(\mathcal{S}\))
      \item Aba! Malaki pala ang \textbf{espasyo} rito. (correct) \\
            Aba! Malaki pala ang \textit{ispasyo} rito. (incorrect) \\
            espasyo (\(\mathcal{F}\)) from espacio (\(\mathcal{S}\))
      \item Marami kaming \textbf{estudyante} sa Computer Science. (correct) \\
            Marami kaming \textit{istudyante} sa Computer Science. (incorrect)\\
            estudyante (\(\mathcal{F}\)) from estudiante (\(\mathcal{S}\))
      \item Mahilig sila Turing at Sipser maglaro ng \textbf{eskrima}. (correct) \\
            Mahilig sila Turing at Sipser maglaro ng \textit{iskrima}. (incorrect) \\
            eskrima  (\(\mathcal{F}\)) from esgrima (\(\mathcal{S}\))
      \item Marami raw \textbf{espiritu} rito. (correct) \\
            Marami raw \textit{ispiritu} rito. (incorrect) \\
            espiritu (\(\mathcal{F}\)) from espiritu (\(\mathcal{S}\))
      \item Kinausap mo na ba iyung \textbf{ispiker}. (correct) \\
            Kinausap mo na ba iyung \textit{espiker}. (incorrect) \\
            ispiker (\(\mathcal{F}\)) from speaker (\(\mathcal{E}\))
      \item Marami naman daw \textbf{isports} na puwedeng pagpilian. (correct) \\
            Marami naman daw \textit{esports} na puwedeng pagpilian. (incorrect) \\
            isports (\(\mathcal{F}\)) from sports (\(\mathcal{E}\))
      \item Si Dijkstra ay hindi raw \textbf{iskolar}. (correct) \\
            Si Dijkstra ay hindi raw \textit{eskolar}. (incorrect) \\
            iskolar (\(\mathcal{F}\)) from scholar (\(\mathcal{E}\))
\end{enumerate}

\begin{example}
      Here are some examples of literature and webpages where the "es-" and "is-"
      prefixes are misused:
\end{example}

\begin{enumerate}
      \item \textbf{TODO}
\end{enumerate}

\subsubsection{"kump"-words and "kumb"-words}

If \(s\in \mathcal{S}\), and the the prefix of \(s\) is given by the
regular expression \((\texttt{C}\mid\texttt{c})\texttt{on}(\texttt{f}\mid\texttt{v})\). Then,
\(\mathcal{F}(s)\) is prefixed with
\((\texttt{C}\mid\texttt{c})\texttt{um}(\texttt{p}\mid\texttt{b})\).

In particular:
\begin{itemize}
      \item \(\left(s \matches (\texttt{conf})\mathbb{S}^*\right) \Longrightarrow \left(\mathcal{F}(s) \matches (\texttt{kump})\Sigma^*\right)\)
      \item \(\left(s \matches (\texttt{conv})\mathbb{S}^*\right) \Longrightarrow \left(\mathcal{F}(s) \matches (\texttt{kumb})\Sigma^*\right)\)
\end{itemize}

\begin{example}
      Here are examples of Spanish words in Filipino with the "kump" and "kumb"
      prefixes:
\end{example}
\begin{enumerate}
      \item Pumasok si Cormen sa \textbf{kumbento}. (correct) \\
            Pumasok si Cormen sa \textit{konbento}. (incorrect) \\
            kumbento (\(\mathcal{F}\)) from convento (\(\mathcal{S}\))
      \item Tinanong ko si Leiserson bilang \textbf{kumpirmasyon}. (correct) \\
            Tinanong ko si Leiserson bilang \textit{kunpirmasyon}. (incorrect) \\
            kumpirmasyon (\(\mathcal{F}\)) from confirmacion (\(\mathcal{S}\))
      \item Ang \textbf{kumpetisyon} sa agham ay ginanap sa paaralan. (correct) \\
            Ang \textit{kunpetisyon} sa agham ay ginanap sa paaralan. (incorrect) \\
            kumpetisyon (\(\mathcal{F}\)) from competición (\(\mathcal{S}\))
      \item Nagbigay siya ng \textbf{kumpisal} sa pari noong Linggo. (correct) \\
            Nagbigay siya ng \textit{kunpisal} sa pari noong Linggo. (incorrect) \\
            kumpisal (\(\mathcal{F}\)) from confesión (\(\mathcal{S}\))
      \item Sumali ako sa isang \textbf{kumbento} sa simbahan natin. (correct) \\
            Sumali ako sa isang \textit{kunbento} sa simbahan natin. (incorrect) \\
            kumbento (\(\mathcal{F}\)) from convento (\(\mathcal{S}\))
      \item Hindi ko alam ang \textbf{kumbinasyon} sa Graph Coloring. (correct) \\
            Hindi ko alam ang \textit{kunbinasyon} sa Graph Coloring. (correct) \\
            kumbinasyon (\(\mathcal{F}\)) from combinación (\(\mathcal{S}\))
      \item Ipinahayag niya ang kanyang \textbf{kumbersasyon} sa kanila. (correct) \\
            Ipinahayag niya ang kanyang \textit{kunbersasyon} sa . (incorrect) \\
            kumbersasyon (\(\mathcal{F}\)) from conversación (\(\mathcal{S}\))
      \item Ang patakaran ay dumaan sa \textbf{kumpigurasyon} bago ipatupad. (correct) \\
            Ang patakaran ay dumaan sa \textit{kunpigurasyon} bago ipatupad. (incorrect) \\
            kumpigurasyon (\(\mathcal{F}\)) from configuración (\(\mathcal{S}\))
\end{enumerate}

\begin{example}
      Here are some examples of literature and webpages where the "kump" and "kumb"
      prefixes are misused:
\end{example}
\begin{enumerate}
      \item \textbf{TODO}
\end{enumerate}

\subsubsection{Morphophonemic Alteration with Suffixes}

Let \( k \in \mathcal{F} \) be a word formed by concatenating the following components:
\[
      k = \alpha \cdot \text{root}_\text{allomorph} \cdot \omega
\]
where \( \alpha \) is the prefix of \( k \), \( \omega \) is the suffix of \( k \), and \( \text{root}_\text{allomorph} \) is the allomorphic form of the root.

The morphophonemic alteration of the root word follows these rules:
\begin{enumerate}
      \item If the root ends in \texttt{e}, its allomorphic form, \( \text{root}_\text{allomorph} \), ends in \texttt{i}.
      \item If the root ends in \texttt{o}, its allomorphic form ends in \texttt{u}.
\end{enumerate}

In other words, if a Filipino word consists of a root ending in \texttt{e} or \texttt{o} followed by a suffix, \texttt{e} changes to \texttt{i} and \texttt{o} changes to \texttt{u} \cite{Yap_1967}.


\begin{example}
      Here are examples of morphophonemic alterations in Filipino:
\end{example}
\begin{enumerate}
      \item \textbf{Tinakbuhan} niya ang kanyang problema. \\
            takbo \(\rightarrow\) takbu \(\cdot\) han \(\rightarrow\) tinakbuhan
      \item \textbf{Pinasukan} niya ang silid-aralan. \\
            pasok \(\rightarrow\) pasuk \(\cdot\) an \(\rightarrow\) pinasukan
      \item \textbf{Pinayuhan} siya ng kanyang guro. \\
            payo \(\rightarrow\) payu \(\cdot\) han \(\rightarrow\) pinayuhan
      \item \textbf{Babaihan} ang tawag sa lugar kung saan maraming babae. \\
            babae \(\rightarrow\) babai \(\cdot\) han \(\rightarrow\) babaíhan
      \item \textbf{Biniruan} niya ang kanyang kaibigan. \\
            biro \(\rightarrow\) biru \(\cdot\) an \(\rightarrow\) biniruan
      \item Gusto naming \textbf{laruin} ni Knuth iyan! \\
            laro  \(\rightarrow\) laru \(\cdot\) in \(\rightarrow\) laruin
      \item Tinanong ko siya kung hahaluin iyong pagkain... \\
            halo  \(\rightarrow\) halu \(\cdot\) in \(\rightarrow\) haluin
      \item Kakayanin kitang \textbf{talunin} sa Catan! \\
            talo  \(\rightarrow\) talu \(\cdot\) nin  \(\rightarrow\) talunin
\end{enumerate}


\begin{example}
      Here are some examples of literature and webpages where that have the incorrect
      root allomorph:
\end{example}
\begin{enumerate}
      \item \textbf{TODO}
\end{enumerate}

\subsubsection{Reduplications}

A full reduplication is a word \(k \in \mathcal{F}\) and is defined by the concatenation
\[
      k = \text{r} \cdot \text{-} \cdot \text{r}
\]
or it is a repetition of a root word \(r\) joined by the \textit{\textit{gitling}} "-".
In a full reduplication, the repetition of the \(r\) has no variation from the first \(r\).

On the other hand, an ablaut reduplication is a word \(j \in \mathcal{F}\) and is defined by the
concatenation

\[
      j = \text{r} \cdot \text{r$^\prime$}
\]
or it is the repetition of the root word \(r\) and the root word with a vowel alteration \(r^\prime\). Notice
that for ablaut repetitions, there is no \textit{\textit{gitling}} joining the repetitions. Furthermore,
the ablaut repetition of a root word creates a word with a different meaning. The ablaut or vowel
alteration is defined by:

\begin{enumerate}
      \item If the root ends in \texttt{e}, then the ablaut ends in \texttt{i}.
      \item If the root ends in \texttt{o}, then the ablaut ends in \texttt{u}.
\end{enumerate}

\begin{example}
      Here are examples of reduplications in Filipino:
\end{example}
\begin{enumerate}
      \item Kung \textbf{ano-ano} ginawa ni Hopcroft kanina. (correct) \\
            Kung \textit{anu-ano} ginawa ni Hopcroft kanina. (incorrect) \\
            ano-ano is a full reduplication
      \item Pupunta kami mamaya sa \textbf{sari-sari} store. (correct) \\
            Pupunta kami mamaya sa \textit{sarisari} store. (incorrect) \\
            sari-sari is a full reduplication
      \item Gagawin natin iyong \textbf{kanya-kanyang} gawain mamaya. (correct) \\
            Gagawin natin iyong \textit{kanyakanyang} gawain mamaya. (incorrect) \\
            kanya-kanya is a full reduplication
      \item \textbf{Paulit-ulit} ka nang tanong! (correct) \\
            \textit{Paulet-ulit} ka nang tanong! (incorrect) \\
            ulit-ulit is a full reduplication
      \item Bumili kami ng \textbf{haluhalo} kaninang umaga. (correct) \\
            Bumili kami ng \textit{halo-halo} kaninang umaga. (incorrect) \\
            haluhalo is an ablaut reduplication
\end{enumerate}

\begin{example}
      Here are some examples of literature and webpages where the wrong reduplicant is
      used:
\end{example}
\begin{enumerate}
      \item \textbf{TODO}
\end{enumerate}

\subsubsection{Allophonic Variation}

Let \(p = "\,"\) the blank symbol or the symbol containing space. And, we have the
sentence structure \(S = \alpha + p + EP\). The \textit{enclitic particles}
(\(\mathbb{EP}\)) is the set \(\mathbb{EP} = \{\texttt{raw}, \texttt{daw}\}\) and
\(\alpha\) is any noun, adjective, verb, or adverb. The proper usage of the enclitic
particles are given by:

If
\[
      \left( S = \alpha + p + EP \land \alpha \matches (\mathbb{L}_\text{upper}\mathbb{L}_\text{lower}^*\mathbb{V}) \right)
      \Longrightarrow
      \left( EP = \texttt{raw} \right)
\]
Otherwise,
\[
      EP = \texttt{daw}
\]

In other words, if the preceding word to the enclitic particle ends in a vowel,
then the enclitic particle is \texttt{raw}. If it is a consonant, then it is \texttt{daw}.

This is the same idea with the adverbs \texttt{rin} and \texttt{din}. If the sentence structure
\(S = \omega + p + (\texttt{rin}\mid\texttt{din})\) and \(\omega\) ends in a vowel
then the adverb used is \texttt{rin}. Otherwise, the adverb is \texttt{din}.

Here are some examples of literature and webpages where the enclitic particles,
and the adverbs "raw" and "daw" are misused:
\begin{enumerate}
      \item \textbf{TODO}
\end{enumerate}

\subsubsection{Ng v. Nang}

There are five way to use "nang", for the sentence structures below

\paragraph{First case:} the string "nang" is used before an adverb when the
structure of any sentence S is given by:

\begin{center}
      (S\textsubscript{1} = v \(\cdot\) p \(\cdot\) pron* \(\cdot\) p \(\cdot\)
      "nang" \(\cdot\) p \(\cdot\) adv)
\end{center}

Where v is a verb, adv is an adverb, and pron is a pronoun.

\begin{example}
      Example sentences for the first case:
\end{example}

\begin{itemize}
      \item Kumain nang mabagal.
      \item Kumain siya nang mabagal.
\end{itemize}

\begin{example}
      Incorrect examples using "ng" instead:
\end{example}

\begin{itemize}
      \item Kumain ng mabagal.
      \item Kumain siya ng mabagal.
\end{itemize}

\paragraph{Second case:} "Nang" can be used to replace the conjunction "noon" +
ligature "-ng". This becomes an error if "ng" is used to replace "noong".

\begin{center}
      (S\textsubscript{2} = "noong" \(\cdot\) (v {\textpipe} adv))
\end{center}

\begin{example}
      Example sentences for the second case:
\end{example}

\begin{itemize}
      \item (Noong {\textpipe} Nang) pumunta siya sa mall…
      \item (Noong {\textpipe} Nang) maaraw na araw…
\end{itemize}

\begin{example}
      Incorrect examples using "ng" instead:
\end{example}

\begin{itemize}
      \item (Noong {\textpipe} Ng) pumunta siya sa mall…
      \item (Noong {\textpipe} Ng) maaraw na araw…
\end{itemize}

\paragraph{Third case:} "Nang" is used to replace a combined Enclictic Particle
"na" + ligature "-ng".

\begin{center}
      (S\textsubscript{3} = v \(\cdot\) p \(\cdot\) "na" \(\cdot\) "-ng")
\end{center}

\begin{example}
      Example sentence for the third case:
\end{example}
\begin{itemize}
      \item Ang mamatay na-ng dahil sa iyo.
      \item Ang mamatay nang dahil sa iyo.
\end{itemize}

\begin{example}
      Incorrect example using "ng" instead:
\end{example}
\begin{itemize}
      \item Ang mamatay na-ng dahil sa iyo.
      \item Ang mamatay ng dahil sa iyo.
\end{itemize}

\paragraph{Fourth case:} "Nang" can be used to replace "upang" or "para".
\begin{center}
      (S\textsubscript{4} = ("upang" {\textpipe} "para") \(\cdot\) p \(\cdot\) v)
\end{center}

\begin{example}
      Example sentences for the fourth case:
\end{example}
\begin{itemize}
      \item (from \cite{OOP}). …kailangang bitayin si Rizal (para {\textpipe} nang)
            matakot ang mga Filipino.
      \item (from \cite{OOP}). Dinala si Pedro sa ospital (para {\textpipe} nang)
            magamot.
\end{itemize}

\begin{example}
      Incorrect examples using "ng" instead:
\end{example}
\begin{itemize}
      \item (from \cite{OOP}). …kailangang bitayin si Rizal (para {\textpipe} ng)
            matakot ang mga Filipino.
      \item (from \cite{OOP}). Dinala si Pedro sa ospital (para {\textpipe} ng)
            magamot.
\end{itemize}

\paragraph{Fifth case:} "Nang" can be used in between repeated verbs.
\begin{center}
      (S\textsubscript{5} = v\textsubscript{1} \(\cdot\) p \(\cdot\) "nang"
      \(\cdot\) p \(\cdot\) v\textsubscript{1})
\end{center}

\begin{example}
      Example sentence for the fifth case:
\end{example}
\begin{itemize}
      \item Kain nang kain...
\end{itemize}

\begin{example}
      Incorrect example using "ng" instead:
\end{example}
\begin{itemize}
      \item Kain ng kain...
\end{itemize}

Literature on Filipino grammar and style highlights that "nang" should not be
mistaken as the contraction of the words "na" + "ang" and that the correct
contraction for them is "na'ng" \cite{OOP}.

\paragraph{"Ng" and nouns.} The string "ng" is used to denote possession similar to
the word "of" in English when the structure of any sentence S is given by:

\begin{center}
      (S\textsubscript{1} = n \(\cdot\) p \(\cdot\) "ng" \(\cdot\) p \(\cdot\) n)
\end{center}

Where n is a noun. \\

\begin{example}
      Examples for "ng" as a linker to indicate possession:
\end{example}
\begin{itemize}
      \item Bag ng bata
      \item Pinto ng bahay
      \item Hawakan ng pinto
\end{itemize}

\begin{example}
      Incorrect examples using "nang" instead:
\end{example}
\begin{itemize}
      \item Bag nang bata
      \item Pinto nang bahay
      \item Hawakan nang pinto
\end{itemize}

\paragraph{"Ng" and object of actions.}"Ng" can be used to mark the direct object
of an action in a sentence.
\begin{center}
      (S\textsubscript{2} = v \(\cdot\) p \(\cdot\) "ng" \(\cdot\) p \(\cdot\) n)
\end{center}

\begin{example}
      Examples for "ng" to indicate the object of the verb.
\end{example}
\begin{itemize}
      \item Binalat ng bata ang saging.
      \item Tinapon ng basurero...
      \item Inalis ng lola...
\end{itemize}

\begin{example}
      Incorrect examples using "nang" instead:
\end{example}
\begin{itemize}
      \item Binalat nang bata ang saging.
      \item Tinapon nang basurero...
      \item Inalis nang lola...
\end{itemize}

\begin{example}
      Here are examples of misused "nang"s and "ng"s in Filipino:
\end{example}
\begin{enumerate}
      \item Inalis \textbf{ng} lola... (correct) \\
            Inalis \textit{nang} lola... (incorrect) \\
            "Nang" is not used to denote possession.
      \item Bag \textbf{ng} bata. (correct) \\
            Bag \textit{nang} bata. (incorrect) \\
            "Nang" is not used to denote possession.
      \item Ang mamatay \textbf{nang} dahil sa iyo. (correct) \\
            Ang mamatay \textit{ng} dahil sa iyo. (incorrect) \\
            "Ng" is not used for combining Enclictic Particles "na" + "-ng".
      \item Sinapak \textbf{nang} masakit... (correct) \\
            Sinapak \textit{ng} masakit... (incorrect) \\
            "Ng" is not used before an adverb.
      \item Kain \textbf{nang} kain... (correct) \\
            Kain \textit{ng} kain... (incorrect) \\
            "Ng" is not used inbetween repeating verbs.
\end{enumerate}

\begin{example}
      Here are some examples of literature and webpages where "nang" or "ng" is
      misused:
\end{example}
\begin{enumerate}
      \item \textbf{TODO}
\end{enumerate}

\subsubsection{\textit{Gitling} Usage}
The usage of a \textit{\textit{gitling}} (hyphen in English) in the Filipino language is quite common and can often lead to its improper use. This section will primarily focus on some of the proper ways to use \textit{\textit{gitling}}. For this section, Let \(k\in\mathcal{F}\) and \(s\in\mathbb{S}\), where \(\mathbb{S}\) is the set of syllables.

\paragraph{1st Case:} Used with the syllable \textbf{"di"}\\\\
Let \(w \in \mathbb{F}*\) be a Filipino root word.
\[
      k = (\text{"di"}) \cdot "\text{-}" \cdot w
\]
The syllable "di" is a shortened format for the word "hindi", and it acts as a negation to the root word.
\begin{example}
\end{example}
\begin{enumerate}
      \item \textbf{di-kumain} (correct) \\
            \textit{dikumain} (incorrect)
      \item \textbf{di-umawit} (correct) \\
            \textit{hindi-umawit} (incorrect)
      \item \textbf{di-ako} (correct) \\
            \textit{di ako} (incorrect)
      \item \textbf{di-tulog} (correct) \\
            \textit{ditulog} (incorrect)
      \item \textbf{di-magnanakaw} (correct) \\
            \textit{hindi-Magnanakaw} (incorrect)
\end{enumerate}

\paragraph{2nd Case:} Used with the syllable \textbf{"alas"}\\\\
Let \(h\) be the hour of day ranging from 1-12 (numeric or non-numeric).
\[
      k = (\text{"alas"}) \cdot "\text{-}" \cdot h
\]
\begin{example}
\end{example}
\begin{enumerate}
      \item \textbf{ala-1} ng tanghali. (correct) \\
            \textit{ala1} ng tanghali. (incorrect)
      \item \textbf{ala-una} ng umaga. (correct) \\
            \textit{ala una} ng umaga. (incorrect)
      \item \textbf{ala-una} ng tanghali. (correct) \\
            \textit{alas-una} ng tanghali. (incorrect)
      \item \textbf{alas-tres} ng tanghali. (correct) \\
            \textit{alas tres} ng tanghali. (incorrect)
      \item \textbf{alas-2} ng tanghali. (correct) \\
            \textit{alas2} ng tanghali. (incorrect)
\end{enumerate}

\paragraph{3rd Case:} With proper nouns\\\\
Let \(j \in \mathcal{F} \wedge \mathbb{PN}\), where \(\mathbb{PN}\) is the set of proper nouns.
\[
      k = s \cdot "\text{-}" \cdot j
\]
\begin{example}
\end{example}
\begin{enumerate}
      \item \textbf{taga-Tondo} (correct) \\
            \textit{taga Tondo} (incorrect)
      \item \textbf{maka-Rizal} (correct) \\
            \textit{makaRizal} (incorrect)
      \item \textbf{pa-Palawan} (correct) \\
            \textit{papunta-palawan} (incorrect)
      \item \textbf{pa-China} (correct) \\
            \textit{pa-china} (incorrect)
      \item \textbf{ka-Davao} (correct) \\
            \textit{kaDavao} (incorrect)
\end{enumerate}

\paragraph{4th Case:} With foreign words\\\\
Let \(w \in \mathbb{FW}\), where \(\mathbb{FW}\) is the set of foreign words.
\[
      k = s \cdot "\text{-}" \cdot w
\]
\begin{example}
\end{example}
\begin{enumerate}
      \item \textbf{pa-message} (correct) \\
            \textit{pa message} (incorrect)
      \item \textbf{pa-call} (correct) \\
            \textit{pa-Call} (incorrect)
      \item \textbf{maki-computer} (correct) \\
            \textit{makicomputer} (incorrect)
      \item \textbf{maki-ride} (correct) \\
            \textit{makiRide} (incorrect)
      \item \textbf{ipa-scan} (correct) \\
            \textit{ipa scan} (incorrect)
\end{enumerate}