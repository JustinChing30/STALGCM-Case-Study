\documentclass[12pt]{book}
\usepackage[utf8]{inputenc}
\usepackage{amsmath}
\usepackage{graphicx}
\usepackage{amsmath}
\usepackage{amssymb}
\usepackage{amsthm}
\usepackage{xcolor}
\usepackage{hyperref}
\usepackage{listings}
\usepackage[a4paper,margin=1in]{geometry} % removing assymetric margins  
\usepackage[style=numeric,backend=biber]{biblatex}
\usepackage{tipx} % for Chapter 1.1.1
\usepackage{fancyhdr} % for headers

% HEADER
\newcommand{\headerfont}{\fontsize{9}{11}}
\pagestyle{fancy}
\fancyhead{}
\fancyhf{} 
\fancyhead[C]{\headerfont{\textbf{\textit{S. Borja, C. Ching, and Z. Ganituen.}}}}
\fancyfoot[R]{\thepage}

% BOOk SETTINGS
\let\cleardoublepage\clearpage

% REFERENCES
\addbibresource{bibliography/references.bib}

% FUNCTIONS
\newtheorem{example}{Example}
\newtheorem{thm_rule}{Rule}
% \newcommand{\matches}{\overset{\text{regex}}{\longrightarrow}}
\newcommand{\regex}[1]{\(\texttt{#1}\)}
\newcommand{\blankspace}{\(\psi\)}
\newcommand{\emptystring}{\(\lambda\)}

% LISTINGS
\definecolor{tagcolor}{rgb}{0.5,0,0.5}
\definecolor{attrcolor}{rgb}{0,0.5,0}
\lstdefinelanguage{XML}{
  basicstyle=\ttfamily\footnotesize,
  morestring=[s]{"}{"},
  moredelim=[s][\color{tagcolor}]{<}{>},
  moredelim=[l][\color{attrcolor}]{\ },
  stringstyle=\color{blue},
  identifierstyle=\color{black}
}

\lstset{
    language=Java,
    backgroundcolor=\color{white},   
    basicstyle=\footnotesize\ttfamily, 
    keywordstyle=\color{blue}\bfseries, 
    commentstyle=\color{gray}\upshape,       
    stringstyle=\color{red},          
    breaklines=true,                  
    showspaces=false,                 
    showstringspaces=false,           
    showtabs=false,                   
    tabsize=2,                        
    breakatwhitespace=true            
}

\title{Formalizing Tagalog and Bikolano}
\author{Stephen Borja, Justin Ching, Zhean Ganituen}
\date{\today}

\begin{document}

\maketitle

This is a case study created as a requirement for the course Advanced Algorithms and Complexities (STALGCM) at De La Salle University, Manila. The source code of this case study document was typeset in \LaTeX\, and is available here:

\begin{center}
  \url{https://github.com/zrygan/Filipino-Case-Study}
\end{center}

This case study also uses LanguageTool version 6.5, the latest as of writing this paper. 

\begin{center}
    \url{https://github.com/languagetool-org/languagetool} 
\end{center}

Additional information on LanguageTool is available in Chapter \ref{language_tool}. The Extensible Markup Language  (\texttt{XML}) and supplementary files created for LanguageTool for this case study are available here:

\label{languageTool_implementation}
\begin{center}
  \url{https://github.com/zrygan/LanguageTool-Case-Study}
\end{center}

\vspace*{\fill}

\begin{tabular}{l l}
  Stephen Borja        & \href{mailto:stephen_borja_a@dlsu.edu.ph}{\texttt{stephen\_borja\_a@dlsu.edu.ph}}            \\
  Justin Ethan Ching   & \href{mailto:justin_ethan_l_ching@dlsu.edu.ph}{\texttt{justin\_ethan\_l\_ching@dlsu.edu.ph}} \\
  Zhean Robby Ganituen & \href{mailto:zhean_robby_ganituen@dlsu.edu.ph}{\texttt{zhean\_robby\_ganituen@dlsu.edu.ph}}  \\
\end{tabular}

\tableofcontents
\lstlistoflistings

\chapter{Preliminaries}
This section discusses some preliminaries for the case study. In particular, it gives a brief introduction to the Filipino and Bikol language, and the notation the case study will use.

\section{Notation}
The following notation and rules for each notation, if any, will be used for the entirety of the paper to remove redundancy and notational ambiguity:

\paragraph{Languages and Alphabets}
All \emph{languages} are represented by script capital letters, while all \emph{alphabets} are represented by blackboard bold capital letters. For instance, the Filipino language is given by, \(\mathcal{F}\) while the Filipino alphabet is given by \(\mathbb{F}\).

\paragraph{String Operations}
The dot operator (\(\cdot\)) denotes a \emph{concatenation} of two strings, the plus operator as a superscript (\(+\)) denotes the \emph{Kleene Plus}, the star operator as a superscript denotes the \emph{Kleene Star}, and the question mark as a superscript denotes the \emph{Kleene Question Mark} or the optional quantifier.

\begin{itemize}
    \item $s\cdot k$ is the concatenation of $s$ and $k$.
    \item $s^+$ is the Kleene star of $s$.
    \item $s^+$ is the Kleene plus of $s$.
    \item $s^?$ is the Kleene question mark of $s$.
\end{itemize}

\paragraph{Regular Expressions}
Regular expressions will use \emph{formal notation} and to say that some string \(s\) matches the regular expression modeled by, \(\mathcal{R}\) then we denote this by \(s \in \mathcal{R}\).

Although the notation for regular expressions uses formal notation, for simplicity, syntax from the IEEE POSIX is occasionally used.

\paragraph{Special Characters in Regular Expressions} The following are special symbols that are used frequently in the paper:

\begin{center}
\begin{tabular}{l l l}
        \textbf{Symbol} & \textbf{Representation} & \textbf{Notes}                    \\
        $\psi$          & " "                     & the blank space symbol $|\psi|=1$ \\
        $\lambda$       & ""                      & the empty string $|\lambda| = 0$
    \end{tabular}
\end{center}

\paragraph{Remarks on IEEE POSIX notation} For cases where wildcard characters are used, such as \texttt{.} or the $\backslash$\texttt{w} will accept only valid characters in the alphabet $\mathbb{F}$, $\mathbb{B}$, or both. For instance, the regular expression \texttt{.} on the language $\mathcal{B}$ will accept $\forall s\in \mathbb{B}$ and will not accept $\forall j\notin \mathbb{B}$ like \texttt{=, +, $\backslash$, @}. Similarly, the regular expression \texttt{.$^*$} on $\mathcal{B}$ will accept all strings from $\mathbb{B}^*$ \cite{posix_standard}.

\paragraph{Remarks on Case Sensitivity} The regular expressions for this paper are case-insensitive, excluding Chapter \ref{proper_nouns_chapter}. That means, when we have the regular expression, $\mathbb{B}^*$ this will accept "kumain", "KUMAIN", and other strings of the same letters but different cases. Furthermore, we say that "KUMAIN" and "kumain" are the same strings.

\section{Filipino}
\subsection{The Filipino Alphabet}
Let \(\mathcal{F}\) be the Filipino language and \(\mathbb{F}\) be the Alphabet of \(\mathcal{F}\), this alphabet is
composed of 56 scripts and 11 punctuation marks \cite{OOP}. The 56 scripts are divided into
two, the first half being the capital letters of the modern Latin script with
the addition of \texttt{Ñ} and \texttt{Ng}; while the other half is the lowercase variants
of each letter.

The 11 punctuation marks in the Filipino language are the: \textit{tuldok} (.),
\textit{tandang pananong} (?), \textit{tandang padamdam} (!), \textit{kuwit} (,),
\textit{kudlit} ('), \textit{\textit{gitling}} (-), \textit{tutuldok} (:),
\textit{tuldok-kuwit} (;), \textit{panipi} ("), \textit{pambukas na panaklong}
((), \textit{pampasarang panaklong} ()), at ang \textit{tutuldok-tuldok} (...)

In mathematical notation, we can represent \(\mathbb{F}\) as the set:
\[
    \mathbb{F} = \{\texttt{a},\texttt{b},\dots,\texttt{z},\texttt{ñ},\texttt{ng},   \
    \texttt{A},\texttt{B},\texttt{C},\dots,\texttt{Z},\texttt{Ñ},\texttt{Ng}\}         \
    \cup \{\texttt{.},\texttt{?},\texttt{!},\texttt{,},\texttt{'},\texttt{-},\texttt{:}, \
    \texttt{;},\texttt{"},\texttt{(},\texttt{)}, \texttt{...}\}
\]

and the size of \(\mathbb{F}\), \(|\mathbb{F}| = 67\) and the set of all possible Filipino words is $\mathcal{F}$. Furthermore, from this, we can introduce the following sets:
\begin{enumerate}
    \item \(\mathbb{M}\) = \{\texttt{.}, \texttt{?}, \texttt{!}, \texttt{,}, \texttt{,}, \texttt{'}, \texttt{-}, \texttt{:}, \texttt{;}, \texttt{"}, \texttt{(}, \texttt{)}, \texttt{...}\}, the set of punctuation marks.
    \item \(\mathbb{V}_\mathbb{F}\) = \{\texttt{a}, \texttt{e}, \texttt{i}, \texttt{o}, \texttt{u}, \texttt{A}, \texttt{E}, \texttt{I}, \texttt{O}, \texttt{U}\}, the set of upper and lower case vowels.
    \item \(\mathbb{C}_\mathbb{F}\) = \(\mathbb{F} - (\mathbb{M} \cup \mathbb{V}_\mathbb{F})\),
          the set of upper and lower case consonants.
    \item \(\Sigma_\mathbb{F}\) = \(\mathbb{F} - \mathbb{M}_\mathbb{F}\), the set of consonants and vowels.
    \item The subscript “lower” on \(\mathbb{C}_\mathbb{F}, \mathbb{V}_\mathbb{F}, \Sigma_\mathbb{F}\) denotes the set of lowercase letters.
    \item The subscript “upper” on \(\mathbb{C}_\mathbb{F}, \mathbb{V}_\mathbb{F}, \Sigma_\mathbb{F}\) denotes the set of uppercase letters.
\end{enumerate}

\paragraph{Remark on the Digraph: \texttt{ng}/\texttt{Ng} or "\textit{en dyi}"}
\label{digraph_remark}

Although the letter "Ng" or "ng" is a concatenation of two separate graphemes or
symbols in \(\mathbb{F}\) (since \(\texttt{Ng} = \texttt{N}\cdot\texttt{g}\) and
\(\texttt{ng} = \texttt{n}\cdot\texttt{g}\)), the letter \texttt{Ng} is officially recognized
as a grapheme in \(\mathbb{F}\) since it represents a distinct Filipino sound.
In particular, it represents the \textit{voiced velar nasal sound}, or in the International
Phonetic Alphabet (IPA), the \textipa{N} sound \cite{Malabonga_2009}.

For instance, the word "hangin" has 5 letters namely: \texttt{h}, \texttt{a}, \texttt{g}, \texttt{i}, \texttt{n},
since \texttt{ng} is pronounced as a velar nasal sound, not as two separate sounds
\textipa{n-g}. So, "hangin" is pronounced as \textipa{haNin} ("ha-ngin").
Take for instance the English  word "manger" where "ng" is a substring
but is not pronounced as the velar nasal sound. Instead, its pronunciation is
\textipa{\textprimstress meI ndZ @r} ("meyn-jer"); not
\textipa{\textprimstress m\ae N Z @r} ("mang-jer"),
\textipa{\textprimstress m\ae N @r} ("mang-er") or
\textipa{\textprimstress m\ae N@r} ("manger").

With this in mind, the digraph \texttt{ng} will be treated as a single letter if the word pronunciation is the voiced velar nasal sound; otherwise it is the two separate characters \texttt{n} and \texttt{g}.

\section{Bikol Language}
\subsection{The Bikol Language and Alphabet}
Let $\mathcal{B}$ be the Bikol language and $\mathbb{B}$ be the alphabet of $\mathcal{B}$. The 64 scripts, the Filipino characters excluding the character \texttt{ñ} and \texttt{Ñ}, and including the vowel sounds with the glottal stop \texttt{'a}, \texttt{'e}, \texttt{'i}, \texttt{'o},  \texttt{'u}, and their upper-case equivalents. The punctuation marks of $\mathcal{F}$ are the same in $\mathcal{B}$.

Therefore, $|\mathbb{B}|=75$ and we can introduce the following subsets:
\begin{enumerate}
    \item \(\mathbb{V}_\mathbb{B}\) = \{\texttt{a}, \texttt{e}, \texttt{i}, \texttt{o}, \texttt{u}, \texttt{A}, \texttt{E}, \texttt{I}, \texttt{O}, \texttt{U}\}, the set of upper and lower case vowels.
    \item \(\mathbb{C}_\mathbb{B}\) = \(\mathbb{B} - (\mathbb{M} \cup \mathbb{V}_\mathbb{B})\),
          the set of upper and lower case consonants.
    \item \(\Sigma_\mathbb{B}\) = \(\mathbb{B} - \mathbb{M}_\mathbb{B}\), the set of consonants and vowels.
    \item The subscript “lower” on \(\mathbb{C}_\mathbb{B}, \mathbb{V}_\mathbb{B}, \Sigma_\mathbb{B}\) denotes the set of lowercase letters.
    \item The subscript “upper” on \(\mathbb{C}_\mathbb{B}, \mathbb{V}_\mathbb{B}, \Sigma_\mathbb{B}\) denotes the set of uppercase letters.
\end{enumerate}

For clarity, words that contain a vowel with a glottal stop and words that contain the same vowel sound without a glottal stop are not the same:

\[
    \forall v\in\mathbb{V}_\mathbb{B}, v^\prime\notin v
\]

And,

\[
    \Sigma_\mathbb{B}^*v\Sigma_\mathbb{B}^*\notin \Sigma_\mathbb{B}^*v^\prime\Sigma_\mathbb{B}^*
\]

Furthermore, since the digraph \texttt{ng} and \texttt{Ng} is in the Bikol language, the same rules discussed \ref{digraph_remark}.

\section{Parts of Speech}

The following are the parts of speech in Filipino and their English counterparts \cite{OOP}.

\begin{center}
    \begin{tabular}{ll}
         \textbf{Filipino} & \textbf{English Equivalent}  \\
         Pangngalan & Noun \\
         Panghalip & Pronouns & 
         Pang-uri & Adjective \\
         Pandiwa & Verb  \\
         Pang-abay & Adverb \\
         Pang-ukol & Preposition \\
         Pang-ugnay & Conjunction \\ 
         Padamdam & Interjection
    \end{tabular}
\end{center}

While the following are the parts of speech in the Bikol language and their English equivalent \cite{bikol_dictionary}.

\begin{center} \begin{tabular}{ll} 
    \textbf{Bikol} & \textbf{English Equivalent} 
    \\ Pangngaran & Noun 
    \\ Pronouns* & Pronouns 
    \\ Adjective* & Adjective 
    \\ Berbo & Verb 
    \\ Adberto & Adverb 
    \\ Preposition* & Preposition 
    \\ Konhusion & Conjunction 
    \\ Interjection* & Interjection 
    \end{tabular} 
    \\
    * indicate that there is no Bikol language translation according to \cite{bikol_dictionary}.
\end{center}
\chapter{Selected Grammatical Errors}
This chapter discusses selected grammatical errors, the regular expression that can check for each error and sample sentences.

% NOTES
% COMMENT THIS OUT, once the chapter is done
{\color{blue}
This section will contain the preliminaries for the case study:

\begin{itemize}
    \item Errors in Tagalog.
    \begin{itemize}
        \item R-D Allophones
        \item Ng v. Nang
        \item Gitling Usage
        \begin{itemize}
            \item "Alas-"
            \item "Di-'
            \item Foreign Words and Proper Nouns
        \end{itemize}
    \end{itemize}
    \item Errors in the Bicol Language.
    \begin{itemize}
        \item Object-Focused Future Tense
        \item U-O Distribution
        \begin{itemize}
            \item Simple Case
            \item Verbs in Object-Focused Future Tense
            \item Reduplications
        \end{itemize}
    \end{itemize}

    The format will be used for \textbf{each error}.
    \begin{itemize}
        \item Discussion of error
        \item Regular Expression of error
        \item Sample correct sentences
        \item Sample incorrect sentences
    \end{itemize}
\end{itemize}
}

\section{Selected Errors in Tagalog}
\subsection{R-D Allophones}
\subsection{Ng v. Nang}

\subsection{Gitling Usage}

\subsubsection{"Alas-"}
Let \(h\) be an hour in the 12-hour clock (numeric or non-numeric).
\[
      k = (\text{"alas"}) \cdot "\text{-}" \cdot h
\]
\begin{example}
\end{example}
\begin{enumerate}
      \item \textbf{ala-1} ng tanghali. (correct) \\
            \textit{ala1} ng tanghali. (incorrect)
      \item \textbf{ala-una} ng umaga. (correct) \\
            \textit{ala una} ng umaga. (incorrect)
      \item \textbf{ala-una} ng tanghali. (correct) \\
            \textit{alas-una} ng tanghali. (incorrect)
      \item \textbf{alas-tres} ng tanghali. (correct) \\
            \textit{alas tres} ng tanghali. (incorrect)
      \item \textbf{alas-2} ng tanghali. (correct) \\
            \textit{alas2} ng tanghali. (incorrect)
\end{enumerate}

\subsubsection{"Di-"}

Let \(w \in \mathbb{F}*\) be a Filipino root word.
\[
      k = (\text{"di"}) \cdot "\text{-}" \cdot w
\]
The syllable "di" is a shortened format for the word "hindi", and it acts as a negation to the root word.
\begin{example}
\end{example}
\begin{enumerate}
      \item \textbf{di-kumain} (correct) \\
            \textit{dikumain} (incorrect)
      \item \textbf{di-umawit} (correct) \\
            \textit{hindi-umawit} (incorrect)
      \item \textbf{di-ako} (correct) \\
            \textit{di ako} (incorrect)
      \item \textbf{di-tulog} (correct) \\
            \textit{ditulog} (incorrect)
      \item \textbf{di-magnanakaw} (correct) \\
            \textit{hindi-Magnanakaw} (incorrect)
\end{enumerate}

\subsubsection{Proper Nouns}

Let \(j \in \mathcal{F} \wedge \mathbb{PN}\), where \(\mathbb{PN}\) is the set of proper nouns.
\[
      k = s \cdot "\text{-}" \cdot j
\]
\begin{example}
\end{example}
\begin{enumerate}
      \item \textbf{taga-Tondo} (correct) \\
            \textit{taga Tondo} (incorrect)
      \item \textbf{maka-Rizal} (correct) \\
            \textit{makaRizal} (incorrect)
      \item \textbf{pa-Palawan} (correct) \\
            \textit{papunta-palawan} (incorrect)
      \item \textbf{pa-China} (correct) \\
            \textit{pa-china} (incorrect)
      \item \textbf{ka-Davao} (correct) \\
            \textit{kaDavao} (incorrect)
\end{enumerate}

\subsubsection{Foreign Words}
Let \(w \in \mathbb{FW}\), where \(\mathbb{FW}\) is the set of foreign words.
\[
      k = s \cdot "\text{-}" \cdot w
\]
\begin{example}
\end{example}
\begin{enumerate}
      \item \textbf{pa-message} (correct) \\
            \textit{pa message} (incorrect)
      \item \textbf{pa-call} (correct) \\
            \textit{pa-Call} (incorrect)
      \item \textbf{maki-computer} (correct) \\
            \textit{makicomputer} (incorrect)
      \item \textbf{maki-ride} (correct) \\
            \textit{makiRide} (incorrect)
      \item \textbf{ipa-scan} (correct) \\
            \textit{ipa scan} (incorrect)
\end{enumerate}

\section{Selected Errors in the Bicol Language}
\subsection{Object-Focused Future Tense}
Any verb in the Bicol Language can be transformed into its Object-Focused Future Tense by adding the suffix:

\[
\texttt{h}^?\texttt{on}
\]

In particular, the Kleene question mark is 1 if and only if the root word $r$ follows:

\[
    \backslash\texttt{b.}^*\texttt{[aeiou]}\backslash\texttt{b}
\]

\begin{example}
\end{example}

\begin{enumerate}
    \item \textbf{gibohon} (correct) \\
    \textit{bihoon} (incorrect)
    \item \textbf{babasahon} (correct) \\
    \textit{babasaon} (incorrect)
    \item \textbf{masasaboton} (correct) \\
    \textit{masasabothon} (incorrect)
    \item \textbf{DO THIS} (correct) \\
    \textit{DO THIS} (incorrect)
    \item \textbf{DO THIS} (correct) \\
    \textit{DO THIS} (incorrect)
\end{enumerate}

\subsection{U-O Distributiion}

\subsubsection{Simple Case}
Any word in the Bicolano language that has an \texttt{o} sound that is not the last syllable is pronounced as \texttt{u}. Only if the \texttt{o} sound is the last syllable of the word, that sound is pronounced as \texttt{o}. 

Formally, we can say that any Bicolano language word that contains an ‘o’ sound can be represented by the regular expression:

\[
    \left(\texttt{C}^?\texttt{uC}^?\right)^* \left(\texttt{C}^?\texttt{oC}^?\right)
\]

\begin{example}
\end{example}

\begin{enumerate}
    \item \textbf{Tukdo} (correct) \\
    \textit{Tokdo} or \textit{Tukdu} (incorrect) \\
    \item \textbf{Bulan} (correct) \\
    \textit{Bolan} (incorrect) \\
    \item \textbf{Ako} (correct) \\
    \textit{Aku} (incorrect) \\
    \item \textbf{Tukdo} (correct) \\
    \textit{Tokdo} or \textit{Tukdu} (incorrect) \\
    \item \textbf{Du'ot} (correct) \\
    \textit{Du'ut} or \textit{Do'ot} (incorrect) \\
\end{enumerate}

\subsubsection{Verbs in Object-Focused Future Tense}



\subsubsection{Reduplications}




\chapter{Error Checkers}
This chapter will discuss the detection of these errors using context-free grammars (CFGs), Greibach normal forms (CNF), and Chomsky normal forms (CNF).

% NOTES
% COMMENT THIS OUT, once the chapter is done
{\color{blue}
This section will contain the preliminaries for the case study:

\begin{itemize}
    \item Each error in Tagalog and their CFG, GNF, and CNF.
    \item Each error in the Bicol Language and their CFG, GNF, and CNF.
\end{itemize}
}

\section{Grammars for Tagalog Errors}
\subsection{Tagalog Grammar: R-D Allophones}
\subsection{Tagalog Grammar: Ng v. Nang}

Nang Errors

\paragraph{First Case}
Example 1
\paragraph{CFG:}
\begin{equation*}
    \begin{aligned}
        \text{Start}   & \rightarrow \text{K "ng" M}   \\
        \text{K} & \rightarrow \text{"kumain"}   \\
        \text{M} & \rightarrow \text{"mabagal"}
    \end{aligned}
\end{equation*}

\paragraph{CNF:}
\begin{equation*}
    \begin{aligned}
        \text{Start}   & \rightarrow \text{KN M}   \\
        \text{K} & \rightarrow \text{"kumain"}   \\
        \text{M} & \rightarrow \text{"mabagal"} \\
        \text{N} & \rightarrow \text{"ng"} \\
        \text{KN} & \rightarrow \text{K N}
    \end{aligned}
\end{equation*}

\paragraph{GNF:}
\begin{equation*}
    \begin{aligned}
        \text{Z1}   & \rightarrow \text{"kumain" Z4 Z3}   \\
        \text{Z3} & \rightarrow \text{"mabagal"}   \\
        \text{Z4} & \rightarrow \text{"ng"}
    \end{aligned}
\end{equation*}

Example 2
\paragraph{CFG:}
\begin{equation*}
    \begin{aligned}
        \text{Start}   & \rightarrow \text{T S “ng” M}   \\
        \text{T} & \rightarrow \text{“tumakbo”}   \\
        \text{S} & \rightarrow \text{“siya”} \\
        \text{M} & \rightarrow \text{"mabilis"}
    \end{aligned}
\end{equation*}

\paragraph{CNF:}
\begin{equation*}
    \begin{aligned}
        \text{Start}   & \rightarrow \text{TS NM}   \\
        \text{T} & \rightarrow \text{“tumakbo”}   \\
        \text{S} & \rightarrow \text{“siya”} \\
        \text{M} & \rightarrow \text{"mabilis"} \\
        \text{N} & \rightarrow \text{"ng"} \\
        \text{TS} & \rightarrow \text{T S} \\
        \text{NM} & \rightarrow \text{N M}
    \end{aligned}
\end{equation*}

\paragraph{GNF:}
\begin{equation*}
    \begin{aligned}
        \text{Z1}   & \rightarrow \text{“tumakbo” Z3 Z7}   \\
        \text{Z3} & \rightarrow \text{“siya”}   \\
        \text{Z4} & \rightarrow \text{“mabilis”} \\
        \text{Z7} & \rightarrow \text{“ng” Z4}
    \end{aligned}
\end{equation*}

Example 3
\paragraph{CFG:}
\begin{equation*}
    \begin{aligned}
        \text{Start}   & \rightarrow \text{S “ng” M}   \\
        \text{s} & \rightarrow \text{“sumigaw”} \\
        \text{m} & \rightarrow \text{“malakas”}
    \end{aligned}
\end{equation*}

\paragraph{CNF:}
\begin{equation*}
    \begin{aligned}
        \text{Start}   & \rightarrow \text{SN M}   \\
        \text{S} & \rightarrow \text{“sumigaw”} \\
        \text{M} & \rightarrow \text{“malakas”} \\
        \text{N} & \rightarrow \text{“ng”} \\
        \text{SN} & \rightarrow \text{S N}
    \end{aligned}
\end{equation*}

\paragraph{GNF:}
\begin{equation*}
    \begin{aligned}
        \text{Z1}   & \rightarrow \text{“sumigaw” Z4 Z3}   \\
        \text{Z3} & \rightarrow \text{“malakas”} \\
        \text{Z4} & \rightarrow \text{“ng”}
    \end{aligned}
\end{equation*}


\paragraph{Second Case}
Example 1
\paragraph{CFG:}
\begin{equation*}
    \begin{aligned}
        \text{Start}   & \rightarrow \text{“ng” P S SA M}   \\
        \text{P} & \rightarrow \text{“pumunta”} \\
        \text{S} & \rightarrow \text{“siya”} \\
        \text{SA} & \rightarrow \text{“sa”} \\
        \text{M} & \rightarrow \text{“mall”}
    \end{aligned}
\end{equation*}

\paragraph{CNF:}
\begin{equation*}
    \begin{aligned}
        \text{Start}   & \rightarrow \text{NPS SAM} \\
        \text{P} & \rightarrow \text{“pumunta”} \\
        \text{S} & \rightarrow \text{“siya”} \\
        \text{SA} & \rightarrow \text{“sa”} \\
        \text{M} & \rightarrow \text{“mall”} \\
        \text{N} & \rightarrow \text{“ng”} \\
        \text{NP} & \rightarrow \text{N P} \\
        \text{SAM} & \rightarrow \text{SA M} \\
        \text{NPS} & \rightarrow \text{NP S}
    \end{aligned}
\end{equation*}

\paragraph{GNF:}
\begin{equation*}
    \begin{aligned}
        \text{Z1}   & \rightarrow \text{“ng” Z2 Z3 Z8} \\
        \text{Z22} & \rightarrow \text{“pumunta”} \\
        \text{Z3} & \rightarrow \text{“siya”} \\
        \text{Z5} & \rightarrow \text{“mall”} \\
        \text{Z8} & \rightarrow \text{“sa” Z5}
    \end{aligned}
\end{equation*}

Example 2
\paragraph{CFG:}
\begin{equation*}
    \begin{aligned}
        \text{Start}   & \rightarrow \text{“ng” M NA A}   \\
        \text{M} & \rightarrow \text{“maaraw”} \\
        \text{NA} & \rightarrow \text{“na”} \\
        \text{A} & \rightarrow \text{“araw”}
    \end{aligned}
\end{equation*}

\paragraph{CNF:}
\begin{equation*}
    \begin{aligned}
        \text{Start}   & \rightarrow \text{NM NAA}   \\
        \text{M} & \rightarrow \text{“maaraw”} \\
        \text{NA} & \rightarrow \text{“na”} \\
        \text{A} & \rightarrow \text{“araw”}
        \text{N} & \rightarrow \text{“ng”} \\
        \text{NM} & \rightarrow \text{N M} \\
        \text{NAA} & \rightarrow \text{NA A} \\
    \end{aligned}
\end{equation*}

\paragraph{GNF:}
\begin{equation*}
    \begin{aligned}
        \text{Z1}   & \rightarrow \text{“ng” Z2 Z7}   \\
        \text{Z2} & \rightarrow \text{“maaraw”} \\
        \text{Z4} & \rightarrow \text{“araw”} \\
        \text{Z7} & \rightarrow \text{“na” Z4}
    \end{aligned}
\end{equation*}

Example 3
\paragraph{CFG:}
\begin{equation*}
    \begin{aligned}
        \text{Start}   & \rightarrow \text{“ng” U S “ng” T}   \\
        \text{U} & \rightarrow \text{“uminom”} \\
        \text{S} & \rightarrow \text{“siya”} \\
        \text{T} & \rightarrow \text{“tubig”}
    \end{aligned}
\end{equation*}

\paragraph{CNF:}
\begin{equation*}
    \begin{aligned}
        \text{Start}   & \rightarrow \text{NU SNT}   \\
        \text{U} & \rightarrow \text{“uminom”} \\
        \text{S} & \rightarrow \text{“siya”} \\
        \text{T} & \rightarrow \text{“tubig”}
        \text{N} & \rightarrow \text{“ng”} \\
        \text{NU} & \rightarrow \text{N U} \\
        \text{NT} & \rightarrow \text{N T} \\
        \text{SNT} & \rightarrow \text{S NT}
    \end{aligned}
\end{equation*}

\paragraph{GNF:}
\begin{equation*}
    \begin{aligned}
        \text{Z1}   & \rightarrow \text{“ng” Z2 Z8}   \\
        \text{Z2} & \rightarrow \text{“uminom”} \\
        \text{Z4} & \rightarrow \text{“siya”} \\
        \text{Z7} & \rightarrow \text{“ng” Z4}
        \text{Z8} & \rightarrow \text{“siya” Z7}
    \end{aligned}
\end{equation*}


\paragraph{Third Case}
Example 1
\paragraph{CFG:}
\begin{equation*}
    \begin{aligned}
        \text{Start}   & \rightarrow \text{A M “ng” D}   \\
        \text{A} & \rightarrow \text{“ang”} \\
        \text{M} & \rightarrow \text{“mamatay”} \\
        \text{D} & \rightarrow \text{“dahil”}
    \end{aligned}
\end{equation*}

\paragraph{CNF:}
\begin{equation*}
    \begin{aligned}
        \text{Start}   & \rightarrow \text{AM ND}   \\
        \text{A} & \rightarrow \text{“ang”} \\
        \text{M} & \rightarrow \text{“mamatay”} \\
        \text{D} & \rightarrow \text{“dahil”}
        \text{N} & \rightarrow \text{“ng”} \\
        \text{AM} & \rightarrow \text{A M} \\
        \text{ND} & \rightarrow \text{N D}
    \end{aligned}
\end{equation*}

\paragraph{GNF:}
\begin{equation*}
    \begin{aligned}
        \text{Z1}   & \rightarrow \text{“ang” Z3 Z7}   \\
        \text{Z3} & \rightarrow \text{“mamatay”} \\
        \text{Z4} & \rightarrow \text{“dahil”} \\
        \text{Z7} & \rightarrow \text{“ng” Z4}
    \end{aligned}
\end{equation*}

Example 2
\paragraph{CFG:}
\begin{equation*}
    \begin{aligned}
        \text{Start}   & \rightarrow \text{A M “ng” B A}   \\
        \text{A} & \rightarrow \text{“ang”} \\
        \text{M} & \rightarrow \text{“magtulungan”} \\
        \text{B} & \rightarrow \text{“buo”}
    \end{aligned}
\end{equation*}

\paragraph{CNF:}
\begin{equation*}
    \begin{aligned}
        \text{Start}   & \rightarrow \text{AM NBA}   \\
        \text{A} & \rightarrow \text{“ang”} \\
        \text{M} & \rightarrow \text{“magtulungan”} \\
        \text{B} & \rightarrow \text{“buo”}
        \text{} & \rightarrow \text{“ng”} \\
        \text{} & \rightarrow \text{A M} \\
        \text{} & \rightarrow \text{B A} \\
        \text{} & \rightarrow \text{N BA}
    \end{aligned}
\end{equation*}

\paragraph{GNF:}
\begin{equation*}
    \begin{aligned}
        \text{Z1}   & \rightarrow \text{“ang” Z3 Z8}   \\
        \text{Z2} & \rightarrow \text{“ang”} \\
        \text{Z3} & \rightarrow \text{“magtulungan”} \\
        \text{Z7} & \rightarrow \text{“buo” Z2}
        \text{Z8} & \rightarrow \text{“ng” Z7}
    \end{aligned}
\end{equation*}

Example 3
\paragraph{CFG:}
\begin{equation*}
    \begin{aligned}
        \text{Start}   & \rightarrow \text{A M “ng” MT AY}   \\
        \text{A} & \rightarrow \text{“ang”} \\
        \text{M} & \rightarrow \text{“magkasama”} \\
        \text{MT} & \rightarrow \text{“matagal”} \\
        \text{AY} & \rightarrow \text{“ay”} \\
    \end{aligned}
\end{equation*}

\paragraph{CNF:}
\begin{equation*}
    \begin{aligned}
        \text{Start}   & \rightarrow \text{AMN MTAY}   \\
        \text{A} & \rightarrow \text{“ang”} \\
        \text{M} & \rightarrow \text{“magkasama”} \\
        \text{MT} & \rightarrow \text{“matagal”} \\
        \text{AY} & \rightarrow \text{“ay”} \\
        \text{N} & \rightarrow \text{“ng”} \\
        \text{AM} & \rightarrow \text{A M} \\
        \text{MTAY} & \rightarrow \text{MT AY} \\
        \text{AMN} & \rightarrow \text{AM N}
    \end{aligned}
\end{equation*}

\paragraph{GNF:}
\begin{equation*}
    \begin{aligned}
        \text{Z1}   & \rightarrow \text{“ang” Z3 Z6 Z8}   \\
        \text{Z3} & \rightarrow \text{“magkasama”} \\
        \text{Z5} & \rightarrow \text{“ay”} \\
        \text{Z6} & \rightarrow \text{“ng”} \\
        \text{Z8} & \rightarrow \text{“matagal” Z5}
    \end{aligned}
\end{equation*}


\paragraph{Fourth Case}
Example 1
\paragraph{CFG:}
\begin{equation*}
    \begin{aligned}
        \text{Start}   & \rightarrow \text{K B S R “ng” M}   \\
        \text{K} & \rightarrow \text{“kailangang”} \\
        \text{B} & \rightarrow \text{“bitayin”} \\
        \text{S} & \rightarrow \text{“si”} \\
        \text{R} & \rightarrow \text{“Rizal”} \\
        \text{M} & \rightarrow \text{“matakot”}
    \end{aligned}
\end{equation*}

\paragraph{CNF:}
\begin{equation*}
    \begin{aligned}
        \text{Start}   & \rightarrow \text{KBS RNM}   \\
        \text{K} & \rightarrow \text{“kailangang”} \\
        \text{B} & \rightarrow \text{“bitayin”} \\
        \text{S} & \rightarrow \text{“si”} \\
        \text{R} & \rightarrow \text{“Rizal”} \\
        \text{M} & \rightarrow \text{“matakot”} \\
        \text{N} & \rightarrow \text{“ng”} \\
        \text{KB} & \rightarrow \text{K B} \\
        \text{NM} & \rightarrow \text{N M} \\
        \text{KBS} & \rightarrow \text{KB S} \\
        \text{RNM} & \rightarrow \text{R NM}
    \end{aligned}
\end{equation*}

\paragraph{GNF:}
\begin{equation*}
    \begin{aligned}
        \text{Z1}   & \rightarrow \text{“kailangang” Z3 Z4 Z11}   \\
        \text{Z3} & \rightarrow \text{“bitayin”} \\
        \text{Z4} & \rightarrow \text{“si”} \\
        \text{Z6} & \rightarrow \text{“matakot”} \\
        \text{Z9} & \rightarrow \text{“ng” Z6} \\
        \text{Z1} & \rightarrow \text{“Rizal” Z9}
    \end{aligned}
\end{equation*}

Example 2
\paragraph{CFG:}
\begin{equation*}
    \begin{aligned}
        \text{Start}   & \rightarrow \text{D S P SA O “ng” M}   \\
        \text{D} & \rightarrow \text{“dinala”} \\
        \text{S} & \rightarrow \text{“si”} \\
        \text{P} & \rightarrow \text{“Pedro”} \\
        \text{SA} & \rightarrow \text{“sa”} \\
        \text{O} & \rightarrow \text{“ospital”} \\
        \text{M} & \rightarrow \text{“magamot”}
    \end{aligned}
\end{equation*}

\paragraph{CNF:}
\begin{equation*}
    \begin{aligned}
        \text{Start}   & \rightarrow \text{DSPSA ONM}   \\
        \text{D} & \rightarrow \text{“dinala”} \\
        \text{S} & \rightarrow \text{“si”} \\
        \text{P} & \rightarrow \text{“Pedro”} \\
        \text{SA} & \rightarrow \text{“sa”} \\
        \text{O} & \rightarrow \text{“ospital”} \\
        \text{M} & \rightarrow \text{“magamot”} \\
        \text{N} & \rightarrow \text{“ng”} \\
        \text{DS} & \rightarrow \text{D S} \\
        \text{NM} & \rightarrow \text{N M} \\
        \text{DSP} & \rightarrow \text{DS P} \\
        \text{ONM} & \rightarrow \text{O NM} \\
        \text{DSPSA} & \rightarrow \text{DSP SA}
    \end{aligned}
\end{equation*}

\paragraph{GNF:}
\begin{equation*}
    \begin{aligned}
        \text{Z1}   & \rightarrow \text{“dinala” Z3 Z4 Z5 Z12}   \\
        \text{Z3} & \rightarrow \text{“si”} \\
        \text{Z4} & \rightarrow \text{“Pedro”} \\
        \text{Z5} & \rightarrow \text{“sa”} \\
        \text{Z7} & \rightarrow \text{“magamot”} \\
        \text{Z10} & \rightarrow \text{“ng” Z7} \\
        \text{Z12} & \rightarrow \text{“ospital” Z10}
    \end{aligned}
\end{equation*}

Example 3
\paragraph{CFG:}
\begin{equation*}
    \begin{aligned}
        \text{Start}   & \rightarrow \text{I S “ng” M}   \\
        \text{I} & \rightarrow \text{“nag-ipon”} \\
        \text{S} & \rightarrow \text{“siya”} \\
        \text{M} & \rightarrow \text{“makabili”}
    \end{aligned}
\end{equation*}

\paragraph{CNF:}
\begin{equation*}
    \begin{aligned}
        \text{Start}   & \rightarrow \text{IS NM}   \\
        \text{I} & \rightarrow \text{“nag-ipon”} \\
        \text{S} & \rightarrow \text{“siya”} \\
        \text{M} & \rightarrow \text{“makabili”} \\
        \text{N} & \rightarrow \text{“ng”} \\
        \text{IS} & \rightarrow \text{I S} \\
        \text{NM} & \rightarrow \text{N M}
    \end{aligned}
\end{equation*}

\paragraph{GNF:}
\begin{equation*}
    \begin{aligned}
        \text{Z1}   & \rightarrow \text{“nag-ipon” Z3 Z7}   \\
        \text{Z3} & \rightarrow \text{“siya”} \\
        \text{Z4} & \rightarrow \text{“makabili”} \\
        \text{Z7} & \rightarrow \text{“ng” Z4}
    \end{aligned}
\end{equation*}


\paragraph{Fifth Case}
Example 1
\paragraph{CFG:}
\begin{equation*}
    \begin{aligned}
        \text{Start}   & \rightarrow \text{K “ng” K}   \\
        \text{K} & \rightarrow \text{“kain”}
    \end{aligned}
\end{equation*}

\paragraph{CNF:}
\begin{equation*}
    \begin{aligned}
        \text{Start}   & \rightarrow \text{KN K}   \\
        \text{K} & \rightarrow \text{“kain”} \\
        \text{N} & \rightarrow \text{“ng”} \\
        \text{KN} & \rightarrow \text{K N}
    \end{aligned}
\end{equation*}

\paragraph{GNF:}
\begin{equation*}
    \begin{aligned}
        \text{Z1}   & \rightarrow \text{“kain” Z3 Z2}   \\
        \text{Z2} & \rightarrow \text{“kain”} \\
        \text{Z3} & \rightarrow \text{“ng”}
    \end{aligned}
\end{equation*}

Example 2
\paragraph{CFG:}
\begin{equation*}
    \begin{aligned}
        \text{Start}   & \rightarrow \text{T “ng” T}   \\
        \text{T} & \rightarrow \text{“takbo”}
    \end{aligned}
\end{equation*}

\paragraph{CNF:}
\begin{equation*}
    \begin{aligned}
        \text{Start}   & \rightarrow \text{TN T}   \\
        \text{T} & \rightarrow \text{“takbo”} \\
        \text{N} & \rightarrow \text{“ng”} \\
        \text{TN} & \rightarrow \text{T N}
    \end{aligned}
\end{equation*}

\paragraph{GNF:}
\begin{equation*}
    \begin{aligned}
        \text{Z1}   & \rightarrow \text{“takbo” Z3 Z2}   \\
        \text{Z2} & \rightarrow \text{“takbo”} \\
        \text{Z3} & \rightarrow \text{“ng”}
    \end{aligned}
\end{equation*}

Example 3
\paragraph{CFG:}
\begin{equation*}
    \begin{aligned}
        \text{Start}   & \rightarrow \text{S “ng” S}   \\
        \text{S} & \rightarrow \text{“salita”}
    \end{aligned}
\end{equation*}

\paragraph{CNF:}
\begin{equation*}
    \begin{aligned}
        \text{Start}   & \rightarrow \text{SN S}   \\
        \text{S} & \rightarrow \text{“salita”} \\
        \text{N} & \rightarrow \text{“ng”} \\
        \text{SN} & \rightarrow \text{S N}
    \end{aligned}
\end{equation*}

\paragraph{GNF:}
\begin{equation*}
    \begin{aligned}
        \text{Z1}   & \rightarrow \text{“salita” Z3 Z2}   \\
        \text{Z2} & \rightarrow \text{“salita”} \\
        \text{Z3} & \rightarrow \text{“ng”}
    \end{aligned}
\end{equation*}

Ng Errors

\paragraph{Noun}
Example 1
\paragraph{CFG:}
\begin{equation*}
    \begin{aligned}
        \text{Start}   & \rightarrow \text{BG “nang” B}   \\
        \text{BG} & \rightarrow \text{“bag”} \\
        \text{B} & \rightarrow \text{“bata”}
    \end{aligned}
\end{equation*}

\paragraph{CNF:}
\begin{equation*}
    \begin{aligned}
        \text{Start}   & \rightarrow \text{BG NB}   \\
        \text{BG} & \rightarrow \text{“bag”} \\
        \text{B} & \rightarrow \text{“bata”} \\
        \text{N} & \rightarrow \text{“nang”} \\
        \text{NB} & \rightarrow \text{N B}
    \end{aligned}
\end{equation*}

\paragraph{GNF:}
\begin{equation*}
    \begin{aligned}
        \text{Z1}   & \rightarrow \text{“bag” Z5}   \\
        \text{Z3} & \rightarrow \text{“bata”} \\
        \text{Z5} & \rightarrow \text{“nang” Z3}
    \end{aligned}
\end{equation*}

Example 2
\paragraph{CFG:}
\begin{equation*}
    \begin{aligned}
        \text{Start}   & \rightarrow \text{P “nang” B}   \\
        \text{P} & \rightarrow \text{“pinto”} \\
        \text{B} & \rightarrow \text{“bahay”}
    \end{aligned}
\end{equation*}

\paragraph{CNF:}
\begin{equation*}
    \begin{aligned}
        \text{Start}   & \rightarrow \text{PN B}   \\
        \text{P} & \rightarrow \text{“pinto”} \\
        \text{B} & \rightarrow \text{“bahay”} \\
        \text{N} & \rightarrow \text{“nang”} \\
        \text{PN} & \rightarrow \text{P N}
    \end{aligned}
\end{equation*}

\paragraph{GNF:}
\begin{equation*}
    \begin{aligned}
        \text{Z1}   & \rightarrow \text{“pinto” Z4 Z3}   \\
        \text{Z3} & \rightarrow \text{“bahay”} \\
        \text{Z4} & \rightarrow \text{“nang”}
    \end{aligned}
\end{equation*}

Example 3
\paragraph{CFG:}
\begin{equation*}
    \begin{aligned}
        \text{Start}   & \rightarrow \text{H “nang” P}   \\
        \text{H} & \rightarrow \text{“hawakan”} \\
        \text{P} & \rightarrow \text{“pinto”}
    \end{aligned}
\end{equation*}

\paragraph{CNF:}
\begin{equation*}
    \begin{aligned}
        \text{Start}   & \rightarrow \text{HN P}   \\
        \text{H} & \rightarrow \text{“hawakan”} \\
        \text{P} & \rightarrow \text{“pinto”} \\
        \text{N} & \rightarrow \text{“nang”} \\
        \text{HN} & \rightarrow \text{H N} \\
    \end{aligned}
\end{equation*}

\paragraph{GNF:}
\begin{equation*}
    \begin{aligned}
        \text{Z1}   & \rightarrow \text{“hawakan” Z4 Z3}   \\
        \text{Z3} & \rightarrow \text{“pinto”} \\
        \text{Z4} & \rightarrow \text{“nang”}
    \end{aligned}
\end{equation*}


\paragraph{Verb}
Example 1
\paragraph{CFG:}
\begin{equation*}
    \begin{aligned}
        \text{Start}   & \rightarrow \text{BIN “nang” B A S}   \\
        \text{BIN} & \rightarrow \text{“binalat”} \\
        \text{B} & \rightarrow \text{“bata”} \\
        \text{A} & \rightarrow \text{“ang”} \\
        \text{S} & \rightarrow \text{“saging”}
    \end{aligned}
\end{equation*}

\paragraph{CNF:}
\begin{equation*}
    \begin{aligned}
        \text{Start}   & \rightarrow \text{BIN NBAS}   \\
        \text{BIN} & \rightarrow \text{“binalat”} \\
        \text{B} & \rightarrow \text{“bata”} \\
        \text{A} & \rightarrow \text{“ang”} \\
        \text{S} & \rightarrow \text{“saging”} \\
        \text{N} & \rightarrow \text{“nang”} \\
        \text{NB} & \rightarrow \text{N B} \\
        \text{AS} & \rightarrow \text{A S} \\
        \text{NBAS} & \rightarrow \text{NB AS}
    \end{aligned}
\end{equation*}

\paragraph{GNF:}
\begin{equation*}
    \begin{aligned}
        \text{Z1}   & \rightarrow \text{“binalat” Z9}   \\
        \text{Z3} & \rightarrow \text{“bata”} \\
        \text{Z5} & \rightarrow \text{“saging”} \\
        \text{Z8} & \rightarrow \text{“ang” Z5} \\
        \text{Z9} & \rightarrow \text{“nang” Z3 Z8}
    \end{aligned}
\end{equation*}

Example 2
\paragraph{CFG:}
\begin{equation*}
    \begin{aligned}
        \text{Start}   & \rightarrow \text{T “nang” B}   \\
        \text{T} & \rightarrow \text{“tinapon”} \\
        \text{B} & \rightarrow \text{“basurero”}
    \end{aligned}
\end{equation*}

\paragraph{CNF:}
\begin{equation*}
    \begin{aligned}
        \text{Start}   & \rightarrow \text{TN B}   \\
        \text{T} & \rightarrow \text{“tinapon”} \\
        \text{B} & \rightarrow \text{“basurero”} \\
        \text{N} & \rightarrow \text{“nang”} \\
        \text{TN} & \rightarrow \text{T N}
    \end{aligned}
\end{equation*}

\paragraph{GNF:}
\begin{equation*}
    \begin{aligned}
        \text{Z1}   & \rightarrow \text{“tinapon” Z4 Z3}   \\
        \text{Z3} & \rightarrow \text{“basurero”} \\
        \text{Z4} & \rightarrow \text{“nang”}
    \end{aligned}
\end{equation*}

Example 3
\paragraph{CFG:}
\begin{equation*}
    \begin{aligned}
        \text{Start}   & \rightarrow \text{I “nang” L}   \\
        \text{I} & \rightarrow \text{“inalis”} \\
        \text{L} & \rightarrow \text{“lola”}
    \end{aligned}
\end{equation*}

\paragraph{CNF:}
\begin{equation*}
    \begin{aligned}
        \text{Start}   & \rightarrow \text{IN L}   \\
        \text{I} & \rightarrow \text{“inalis”} \\
        \text{L} & \rightarrow \text{“lola”} \\
        \text{N} & \rightarrow \text{“nang”} \\
        \text{IN} & \rightarrow \text{I N}
    \end{aligned}
\end{equation*}

\paragraph{GNF:}
\begin{equation*}
    \begin{aligned}
        \text{Z1}   & \rightarrow \text{“inalis” Z4 Z3}   \\
        \text{Z3} & \rightarrow \text{“lola”} \\
        \text{Z4} & \rightarrow \text{“nang”}
    \end{aligned}
\end{equation*}

\subsection{Tagalog Grammar: Gitling Usage for "Alas-"}
\paragraph{CFG:}

\begin{equation*}
    \begin{aligned}
        \text{Start}   & \rightarrow \text{Error\_1 {\textpipe} Error\_2 {\textpipe} Error\_3}   \\
        \text{O} & \rightarrow \text{1 {\textpipe} "una"}   \\
        \text{H} & \rightarrow \text{2 {\textpipe} "dos"} \\
        \text{H} & \rightarrow \text{3 {\textpipe} "tres"} \\
        \text{H} & \rightarrow \text{4 {\textpipe} "kwatro"} \\
        \text{H} & \rightarrow \text{5 {\textpipe} "singko"} \\
        \text{H} & \rightarrow \text{6 {\textpipe} "sais"} \\
        \text{H} & \rightarrow \text{7 {\textpipe} "syete"} \\
        \text{H} & \rightarrow \text{8 {\textpipe} "otso"} \\
        \text{H} & \rightarrow \text{9 {\textpipe} "nwebe"} \\
        \text{H} & \rightarrow \text{10 {\textpipe} "dyes"} \\
        \text{H} & \rightarrow \text{11 {\textpipe} "onse"} \\
        \text{H} & \rightarrow \text{12 {\textpipe} "dose"} \\
        \text{P} & \rightarrow \text{" "} \\
        \text{G} & \rightarrow \text{"-"} \\
        \text{Error\_1}   & \rightarrow \text{"alas" O {\textpipe} "alas" H}   \\
        \text{Error\_1}   & \rightarrow \text{"ala" H {\textpipe} "ala" O}   \\
        \text{Error\_2}   & \rightarrow \text{"alas" P O {\textpipe} "alas" P H}   \\
        \text{Error\_2}   & \rightarrow \text{"ala" P H {\textpipe} "ala" P O }   \\
        \text{Error\_3}   & \rightarrow \text{"alas" G O {\textpipe} "ala"G H}
    \end{aligned}
\end{equation*}
\text{For ease, the non-terminal derived from H is "h" representing the hours following the first hour.}

\paragraph{CNF (Chomsky Normal Form):}

\begin{equation*}
    \begin{aligned}
        \text{Start}   & \rightarrow \text{S O {\textpipe} S H} \\
        \text{Start}   & \rightarrow \text{A O {\textpipe} A H} \\
        \text{Start}   & \rightarrow \text{S PO {\textpipe} S PH} \\
        \text{Start}   & \rightarrow \text{A PO {\textpipe} A PH} \\
        \text{Start}   & \rightarrow \text{S GO {\textpipe} A GH} \\
        \text{S}    & \rightarrow \text{"alas"}\\
        \text{A}    & \rightarrow \text{"ala"}\\
        \text{O}   & \rightarrow \text{"1" {\textpipe} "una"}\\
        \text{H}   & \rightarrow \text{h}\\
        \text{P}   & \rightarrow \text{" "}\\
        \text{G} & \rightarrow \text{"-"}\\
        \text{PO}      & \rightarrow \text{P O}\\
        \text{PH}      & \rightarrow \text{P H}\\
        \text{GO}      & \rightarrow \text{G O}\\
        \text{GH}      & \rightarrow \text{G H}
    \end{aligned}
\end{equation*}

\paragraph{GNF (Greibach Normal Form):}

\begin{equation*}
    \begin{aligned}
        \text{Start}   & \rightarrow \text{"alas" O {\textpipe} "alas" H} \\
        \text{Start}   & \rightarrow \text{"ala" O {\textpipe} "ala" H} \\
        \text{Start}   & \rightarrow \text{"alas" P O {\textpipe} "alas" P H} \\
        \text{Start}   & \rightarrow \text{"ala" P O {\textpipe} "ala" P H} \\
        \text{Start}   & \rightarrow \text{"alas" G O {\textpipe} "alas" G H} \\
        \text{O}   & \rightarrow \text{"1" {\textpipe} "una"}\\
        \text{H}   & \rightarrow \text{h}\\
        \text{P}   & \rightarrow \text{" "}\\
        \text{G} & \rightarrow \text{"-"}
    \end{aligned}
\end{equation*}

\subsection{Tagalog Grammar: Gitling Usage for "Di-"}

\paragraph{CFG:}

\begin{equation*}
    \begin{aligned}
        \text{Start}  & \rightarrow \text{Error\_1 {\textpipe} Error\_2 {\textpipe} Error\_3}   \\
        \text{P}   & \rightarrow \text{" "}\\
        \text{G} & \rightarrow \text{"-"}\\        
        \text{W} & \rightarrow \text{doktor {\textpipe} lalaki {\textpipe} mabait {\textpipe} tulog}   \\
        \text{C} & \rightarrow \text{Doktor {\textpipe} Lalaki {\textpipe} Mabait {\textpipe} Tulog}   \\
        \text{Error\_1}  & \rightarrow \text{"di" W {\textpipe} "di" P W}   \\
        \text{Error\_1}  & \rightarrow \text{"di" C {\textpipe} "di" P C}   \\
        \text{Error\_2}  & \rightarrow \text{"hindi" G W {\textpipe} "hindi" G C}   \\
        \text{Error\_3}  & \rightarrow \text{"di" G C}   
    \end{aligned}
\end{equation*}

\paragraph{CNF (Chomsky Normal Form):}

\begin{equation*}
    \begin{aligned}
        \text{Start}   & \rightarrow \text{S W {\textpipe} S PW} \\
        \text{Start}   & \rightarrow \text{S C {\textpipe} S PC} \\
        \text{Start}   & \rightarrow \text{H GW {\textpipe} H GC} \\
        \text{Start}   & \rightarrow \text{S GC} \\
        \text{S}    & \rightarrow \text{"di"}\\
        \text{H}    & \rightarrow \text{"hindi"}\\
        \text{W} & \rightarrow \text{doktor {\textpipe} lalaki {\textpipe} mabait {\textpipe} tulog}   \\
        \text{C} & \rightarrow \text{Doktor {\textpipe} Lalaki {\textpipe} Mabait {\textpipe} Tulog}   \\
        \text{P}   & \rightarrow \text{" "}\\
        \text{G} & \rightarrow \text{"-"}\\
        \text{PW}      & \rightarrow \text{P W}\\
        \text{PC}      & \rightarrow \text{P C}\\
        \text{GW}      & \rightarrow \text{G W}\\
        \text{GC}      & \rightarrow \text{G C}
    \end{aligned}
\end{equation*}

\paragraph{GNF (Greibach Normal Form):}

\begin{equation*}
    \begin{aligned}
        \text{Start}   & \rightarrow \text{"di" W {\textpipe} "di" P W} \\
        \text{Start}   & \rightarrow \text{"di" C {\textpipe} "di" PC} \\
        \text{Start}   & \rightarrow \text{"hindi" GW {\textpipe} "hindi" GC} \\
        \text{Start}   & \rightarrow \text{"di" GC} \\
        \text{W} & \rightarrow \text{doktor {\textpipe} lalaki {\textpipe} mabait {\textpipe} tulog}   \\
        \text{C} & \rightarrow \text{Doktor {\textpipe} Lalaki {\textpipe} Mabait {\textpipe} Tulog}   \\
        \text{P}   & \rightarrow \text{" "}\\
        \text{G} & \rightarrow \text{"-"}
    \end{aligned}
\end{equation*}

\subsection{Tagalog Grammar: Gitling Usage for Proper Nouns}

\paragraph{CFG:}

\begin{equation*}
    \begin{aligned}
        \text{Start}  & \rightarrow \text{Error\_1 {\textpipe} Error\_2}   \\
        \text{P}   & \rightarrow \text{" "}\\
        \text{G} & \rightarrow \text{"-"}\\        
        \text{S}    & \rightarrow \text{"taga" {\textpipe} "pa" {\textpipe} "maka"}\\
        \text{PN} & \rightarrow \text{"Tondo" {\textpipe} "Davao" {\textpipe} "Rizal" {\textpipe} "DLSU"}   \\
        \text{PV} & \rightarrow \text{"tondo" {\textpipe} "davao" {\textpipe} "rizal" {\textpipe} "dlsu"}   \\
        \text{Error\_1}  & \rightarrow \text{S PN {\textpipe} S P PN}   \\
        \text{Error\_1}  & \rightarrow \text{S PV {\textpipe} S P PV}   \\
        \text{Error\_2}  & \rightarrow \text{S G PV}   
    \end{aligned}
\end{equation*}

\paragraph{CNF (Chomsky Normal Form):}

\begin{equation*}
    \begin{aligned}
        \text{Start}  & \rightarrow \text{S PN {\textpipe} S PPN}   \\
        \text{Start}  & \rightarrow \text{S PV {\textpipe} S PPV}   \\
        \text{Start}  & \rightarrow \text{S GPV}   \\
        \text{P}   & \rightarrow \text{" "}\\
        \text{G} & \rightarrow \text{"-"}\\        
        \text{S}    & \rightarrow \text{"taga" {\textpipe} "pa" {\textpipe} "maka"}\\
        \text{PN} & \rightarrow \text{"Tondo" {\textpipe} "Davao" {\textpipe} "Rizal" {\textpipe} "DLSU"}   \\
        \text{PV} & \rightarrow \text{"tondo" {\textpipe} "davao" {\textpipe} "rizal" {\textpipe} "dlsu"}   \\
        \text{PPN} & \rightarrow \text{P PN}   \\
        \text{PPV} & \rightarrow \text{P PV}  \\
        \text{GPV} & \rightarrow \text{G PV}  
    \end{aligned}
\end{equation*}

\paragraph{GNF (Greibach Normal Form):}

\begin{equation*}
    \begin{aligned}
        \text{Start}  & \rightarrow \text{"taga" PN {\textpipe} "taga" P PN}   \\
        \text{Start}  & \rightarrow \text{"taga" PV {\textpipe} "taga" P PV}   \\
        \text{Start}  & \rightarrow \text{"taga" G PV}   \\
        \text{Start}  & \rightarrow \text{"pa" PN {\textpipe} "pa" P PN}   \\
        \text{Start}  & \rightarrow \text{"pa" PV {\textpipe} "pa" P PV}   \\
        \text{Start}  & \rightarrow \text{"pa" GPV}   \\
        \text{Start}  & \rightarrow \text{"maka" PN {\textpipe} "maka" P PN}   \\
        \text{Start}  & \rightarrow \text{"maka" PV {\textpipe} "maka" P PV}   \\
        \text{Start}  & \rightarrow \text{"maka" G PV}   \\
        \text{P}   & \rightarrow \text{" "}\\
        \text{G} & \rightarrow \text{"-"}\\        
        \text{PN} & \rightarrow \text{"Tondo" {\textpipe} "Davao" {\textpipe} "Rizal" {\textpipe} "DLSU"}   \\
        \text{PV} & \rightarrow \text{"tondo" {\textpipe} "davao" {\textpipe} "rizal" {\textpipe} "dlsu"}   
    \end{aligned}
\end{equation*}

\subsection{Tagalog Grammar: Gitling Usage for Foreign Words}

\paragraph{CFG:}

\begin{equation*}
    \begin{aligned}
        \text{Start}  & \rightarrow \text{Error\_1 {\textpipe} Error\_2 {\textpipe} Edge\_case}   \\
        \text{P}   & \rightarrow \text{" "}\\
        \text{G} & \rightarrow \text{"-"}\\        
        \text{S}    & \rightarrow \text{"pa" {\textpipe} "ipa" {\textpipe} "maki"}\\        
        \text{W} & \rightarrow \text{"message" {\textpipe} "anime" {\textpipe} "game"}\\
        \text{C} & \rightarrow \text{"Message" {\textpipe} "Anime" {\textpipe} "Game"}\\
        \text{Z} & \rightarrow \text{"Kpop"}\\        
        \text{E} & \rightarrow \text{"kpop" {\textpipe} "K-pop" {\textpipe} "k-pop"}\\        
        \text{Error\_1}  & \rightarrow \text{S W {\textpipe} S P W}   \\
        \text{Error\_1}  & \rightarrow \text{S C {\textpipe} S P C}   \\
        \text{Error\_2}  & \rightarrow \text{S G C}   \\
        \text{Edge\_case}  & \rightarrow \text{S G E {\textpipe} S P E {\textpipe} S E} \\   
        \text{Edge\_case}  & \rightarrow \text{S P Z {\textpipe} S Z}         
    \end{aligned}
\end{equation*}

\paragraph{CNF (Chomsky Normal Form):}

\begin{equation*}
    \begin{aligned}
        \text{Start}  & \rightarrow \text{S W {\textpipe} S PW}   \\
        \text{Start}  & \rightarrow \text{S C {\textpipe} S PC}   \\
        \text{Start}  & \rightarrow \text{S GC}   \\
        \text{Start}  & \rightarrow \text{S GE {\textpipe} S PE {\textpipe} S E} \\   
        \text{Start}  & \rightarrow \text{S PZ {\textpipe} S Z}      \\
        \text{P}   & \rightarrow \text{" "}\\
        \text{G} & \rightarrow \text{"-"}\\        
        \text{S}    & \rightarrow \text{"pa" {\textpipe} "ipa" {\textpipe} "maki"}\\        
        \text{W} & \rightarrow \text{"message" {\textpipe} "anime" {\textpipe} "game"}\\
        \text{C} & \rightarrow \text{"Message" {\textpipe} "Anime" {\textpipe} "Game"}\\
        \text{Z} & \rightarrow \text{"Kpop"}\\        
        \text{E} & \rightarrow \text{"kpop" {\textpipe} "K-pop" {\textpipe} "k-pop"}\\
        \text{PW} & \rightarrow \text{P W}\\    
        \text{PC} & \rightarrow \text{P C}\\        
        \text{GC} & \rightarrow \text{G C}\\        
        \text{PE} & \rightarrow \text{P E}\\     
        \text{GE} & \rightarrow \text{G E}\\
        \text{PZ} & \rightarrow \text{P Z}
    \end{aligned}
\end{equation*}

\paragraph{GNF (Greibach Normal Form):}

\begin{equation*}
    \begin{aligned}
        \text{Start}  & \rightarrow \text{"pa" W {\textpipe} "pa" P W}   \\
        \text{Start}  & \rightarrow \text{"pa" C {\textpipe} "pa" P C}   \\
        \text{Start}  & \rightarrow \text{"pa" G C}   \\
        \text{Start}  & \rightarrow \text{"pa" G E {\textpipe} "pa" P E {\textpipe} "pa" E} \\   
        \text{Start}  & \rightarrow \text{"pa" P Z {\textpipe} "pa" Z}      \\
        \text{Start}  & \rightarrow \text{"ipa" W {\textpipe} "ipa" P W}   \\
        \text{Start}  & \rightarrow \text{"ipa" C {\textpipe} "ipa" P C}   \\
        \text{Start}  & \rightarrow \text{"ipa" G C}   \\
        \text{Start}  & \rightarrow \text{"ipa" G E {\textpipe} "ipa" P E {\textpipe} "ipa" E} \\   
        \text{Start}  & \rightarrow \text{"ipa" P Z {\textpipe} "ipa" Z}      \\
        \text{Start}  & \rightarrow \text{"maki" W {\textpipe} "maki" P W}   \\
        \text{Start}  & \rightarrow \text{"maki" C {\textpipe} "maki" P C}   \\
        \text{Start}  & \rightarrow \text{"maki" G C}   \\
        \text{Start}  & \rightarrow \text{"maki" G E {\textpipe} "maki" P E {\textpipe} "maki" E} \\   
        \text{Start}  & \rightarrow \text{"maki" P Z {\textpipe} "maki" Z}      \\
        \text{P}   & \rightarrow \text{" "}\\
        \text{G} & \rightarrow \text{"-"}\\        
        \text{W} & \rightarrow \text{"message" {\textpipe} "anime" {\textpipe} "game"}\\
        \text{C} & \rightarrow \text{"Message" {\textpipe} "Anime" {\textpipe} "Game"}\\
        \text{Z} & \rightarrow \text{"Kpop"}\\        
        \text{E} & \rightarrow \text{"kpop" {\textpipe} "K-pop" {\textpipe} "k-pop"}
    \end{aligned}
\end{equation*}

\section{Grammars for Bicol Language Errors}
\subsection{Bicol Language Grammar: Object-Focused Future Tense}
\subsection{Bicol Language Grammar: U-O Distribution for the Simple Case}
\subsection{Bicol Language Grammar: U-O Distribution for Verbs in Object-Focused Future Tense}
\subsection{Bicol Language Grammar: U-O Distribution for Reduplications}

\chapter{Exploring Literature with Errors}

This chapter will discuss Tagalog and Bicol literature and websites that have the errors discussed in the previous chapters.

\section{Tagalog Literature}

\section{Bicol Literature}


\section{Tagalog Literature}
\subsection{Errors in R-D Allophones}
\subsection{Errors in Ng v. Nang}
\subsection{Errors in Gitling Usage for "Alas-"}
\subsection{Errors in Gitling Usage for "Di-"}
\subsection{Errors in Gitling Usage for Foreign Words and Proper Nouns}


\section{Bicol Language Literatore}
\subsection{Errors in Object-Focused Future Tense}
\subsection{Errors in U-O Distribution for the Simple Case}
\subsection{Errors in U-O Distribution for Verbs in Object-Focused Future Tense}
\subsection{Errors in U-O Distribution for Reduplications}
\chapter{LanguageTool}
\label{language_tool}
\section{Introduction to LanguageTool}

LanguageTool is an open-source grammar and spell checker, originally by Daniel Naber \url{https://github.com/danielnaber}, that utilizes a combination of XML-based pattern rules, regular expressions, and Java code to detect and correct linguistic errors across multiple languages \cite{about_languagetool}. The core of its functionality lies in the \texttt{grammar.xml} file, where most rules are defined. These XML rules consist of patterns that match specific sequences of words or part-of-speech tags, allowing LanguageTool to identify common grammatical mistakes. Regular expressions enhance this capability by enabling the creation of flexible and complex matching patterns within the XML rules. For instance, a rule might use a regular expression to match variations of a verb to ensure subject-verb agreement. When certain grammatical issues cannot be effectively captured through XML patterns and regular expressions, LanguageTool allows for the implementation of custom Java rules \cite{LanguageToolGitHub}. 

Developers can extend LanguageTool's \texttt{Rule} class and implement matching function of that class, the \texttt{match(AnalyzedSentence)} method. Doing so allows the user to create more sophisticated checks that are not possible using regular expressions alone. This multi-faceted approach enables LanguageTool to provide comprehensive grammar checking by leveraging the strengths of regular expressions for flexibility, and Java for complex rule implementation \cite{LanguageToolGitHub}.

The developer documentation of LanguageTool discusses in detail the implementation of rules \cite{LanguageToolDevDocs}. In summary, the grammar checking of LanguageTool works as follows:

\paragraph{Boilerplate Structure of a Simple Rule} Each rule is defined by a \texttt{<rule>} element containing a \texttt{<patter>} to match the text and a \texttt{<message>} to display when the pattern is found.

\begin{lstlisting}[language=XML, caption=Boilerplate Code for a Simple Rule]
<rule id="RULE_ID" name="Rule Name">
    <pattern>
        <!-- Tokens and regular expressions go here -->
    </pattern>
    <message>Your message explaining the issue.</message>
    <example correction="corrected text">
        Incorrect text to be highlighted.
    </example>
</rule>    
\end{lstlisting}

\paragraph{Defining a Token} Each \texttt{<token>} represents a word or punctuation mark in the text, and we can restrict the matching for the \texttt{token} by using regular expressions or the \texttt{regexp} attribute.

For instance, we want to create a grammar rule in \texttt{grammar.xml} that defines and checks for the "alot" and "a lot" error.

\begin{lstlisting}[language= XML, caption=Example Grammar Rule for "alot" v. "a lot"]
<rule id="ALOT_RULE" name="Alot Correction">
    <pattern>
        <token regexp="yes">[aA]lot</token>
    </pattern>
    <message>
        'Alot' is a common misspelling. 
        Consider using 'a lot' instead.
    </message>
    <example correction="a lot">
        I have <marker>alot</marker> of work to do.
    </example>
</rule>
\end{lstlisting}

A grammar rule has the following parts. First, we have the \texttt{<rule>} tag which initializes the rule; it also allows you to indicate an \texttt{id} and \texttt{name} for that rule. Then, we specify the \texttt{<patter>} tag which uses pattern matching to check if the defined pattern is found in the input text. Furthermore, we can use regular expressions for the pattern matching by using the \texttt{<token regexp="yes">} tag. Then, we have the \texttt{<message>} tag that displays a specific message when the error pattern is found. Finally, we can also include the \texttt{<example>} tag to include examples in the error message.

As for creating sophisticated grammar and style rules, these are discussed at length in Chapter \ref{future_work}.

\section{LanguageTool Implementation}

The complete source code for the LanguageTool implementation for the errors discussed in the previous chapters is available in the preface of the paper. The sections below will take a closer look at one of the grammars implemented for Tagalog and Bikol language.


\subsection{LanguageTool Implementation for R-D Alternation}

This section will discuss how the rule for R-D Alternation discussed in Chapter \ref{rd_alternation} is implemented as a grammar in LanguageTool.

% Maximum source code width                                             |
\begin{lstlisting}[language= XML, caption=Grammar for R-D Alternation]
<category id="d-r_r-d" name="D-R R-D Change">
    <rule id="daw_raw" name="change daw -> raw">
        <pattern case_sensitive="no" mark_from="1">
            <token postag="N.*|AD.*|V.*|AV.*" postag_regexp="yes" 
                regexp="yes">
                [A-Z][a-z]*[aeiou]
            </token>
            <token>
                daw
            </token>
        </pattern>
        <message>
            Use <suggestion>\1 raw</suggestion>; if the noun, adjective, 
            verb, or adverb ends in a vowel, you have to use "raw".
        </message>
        <short>
            Daw to raw if word ends in a vowel
        </short>
    </rule>

    <rule id="raw_daw" name="change raw -> daw">
        <pattern case_sensitive="no" mark_from="1">
            <token postag="N.*|AD.*|V.*|AV.*" 
                postag_regexp="yes" regexp="yes">
                [A-Z][a-z]*[^aeiou]
            </token>
            <token>
                raw
            </token>
        </pattern>
        <message>
            Use <suggestion>\1 daw</suggestion>; if the noun, adjective, 
            verb, or adverb ends in a consonant, you have to use "daw".
        </message>
        <short>
            Raw to daw if word ends in a consonant
        </short>
    </rule>

    <rule id="din_rin" name="change din -> rin">
        <pattern case_sensitive="no" mark_from="1">
            <token postag="N.*|AD.*|V.*|AV.*" postag_regexp="yes" 
                regexp="yes">
                [A-Z][a-z]*[aeiou]
            </token>
            <token>
                din
            </token>
        </pattern>
        <message>
            Use <suggestion>\1 rin</suggestion>; if the noun, adjective,
            verb, or adverb ends in a vowel, you have to use "rin".
        </message>
        <short>
            Din to rin if word ends in a vowel
        </short>
    </rule>

    <rule id="rin_din" name="change rin -> din">
        <pattern case_sensitive="no" mark_from="1">
            <token postag="N.*|AD.*|V.*|AV.*" postag_regexp="yes" 
                regexp="yes">
                [A-Z][a-z]*[^aeiou]
            </token>
            <token>
                rin
            </token>
        </pattern>
        <message>
            Use <suggestion>\1 din</suggestion>; if the noun, adjective, 
            verb, or adverb ends in a consonant, you have to use "din".
        </message>
        <short>
            Rin to din if word ends in a consonant
        </short>
    </rule>
</category>
\end{lstlisting}

Here we see that we have one \textit{rule category} that groups similar rules. In this case, the category is \texttt{d-r\textunderscore r-d} which has four rules as a part of it. Namely, each rule is for the \textbf{Raw to Daw}, \textbf{Daw to Raw}, \textbf{Rin to Din}, and \textbf{Din to Rin}.

\subsection{LanguageTool Implementation for U-O Distribution}

This section will discuss how the rule for U-O Distribution discussed in Chapter \ref{uo_distribution} is implemented as a grammar in LanguageTool.

% Maximum source code width                                             |
\begin{lstlisting}[language= XML, caption=Grammar for U-O Distribution]
<category id="u-o_distribution" name ="U-O Distribution">
    <rule id="simple_case1" name="Detect incorrect 'u' and 'o' usage">
        <pattern case_sensitive="no"> 
            <token regex="yes">
                \b([^aeiou]?[aeio][^aeiou])+
                ([^aeiou]?[aeiu][^aeiou]?)\b
            </token>
        </pattern>
        <message>
            'o' must be 'u' if it is not the last syllable; 'u' 
            must be 'o' if it is the last syllable.
        </message>
        <short>
            'o' must be 'u' if it is not the last syllable; 'u' 
            must be 'o' if it is the last syllable.
        </short>
    </rule>

    <rule id="simple_case2" name="Change every 'o' except the last 
        'o' syllable to 'u'.">
        <pattern case_sensitive="no"> 
            <token regex="yes">
                \b[^aeiou]*o([^aeiou]*o)+[^aeiou]*\b
            </token>
        </pattern>
        <message>
            Change every 'o' except the last 'o' syllable to 'u'.
        </message>
        <short>
            Change every 'o' except the last 'o' syllable to 'u'.
        </short>
    </rule>

    <rule id="simple_case3" name="Change last 'u' syllable to 'o'.">
        <pattern case_sensitive="no"> 
            <token regex="yes">\
                b[^aeiou]*u([^aeiou]*u)+[^aeiou]*\b
            </token>
        </pattern>
        <message>
            Change last 'u' syllable to 'o'.
        </message>
        <short>
            Change last 'u' syllable to 'o'.
        </short>
    </rule>

    <rule id="verbs_object_focus" name="Change 'u' to 'o'.">
        <pattern case_sensitive="no"> 
            <token regex="yes">
                \b[a-zA-Z']*u[a-zA-Z']*h?on\b(?<!\b[a-zA-Z']*
                o[a-zA-Z']*h?on\b)
            </token>
        </pattern>
        <message>
            Replace second-to-last suffix 'u' with 'o' because 
            verbs at their object-focused future form may have 
            an 'o' sound. 
        </message>
        <short>
            Change second-to-last suffix 'u' with 'o'.
        </short>
    </rule>

    <rule id="u-o_reduplication1" name="Change 'o' to 'u'.">
        <pattern case_sensitive="no"> 
            <token regex="yes">
                (?:^|(?<=\s))
                ([b-df-hj-np-tv-z']*u[b-df-hj-np-tv-z']*)
                (?!\1)[b-df-hj-np-tv-z']*o[b-df-hj-np-tv-z']*
                (?:$|(?=\s))
            </token>
        </pattern>
        <message>
            Replace the 'o' with 'u' because the reduplicated 
            syllable uses 'u'. 
        </message>
        <short>
            Change 'o' in the 2nd part to 'u'.
        </short>
    </rule>
</category>
\end{lstlisting}

Here we see that we have one \textit{rule category} that groups similar rules. In this case, the category is \texttt{u-o\textunderscore distribution} which has five rules as a part of it. Namely, three sub-cases of the \textbf{Simple Case}, \textbf{Object-Focused Future Verbs}, and \textbf{Reduplications} for the U-O Distribution in the Bikol language.


\chapter{Future Work}
\label{future_work}

This chapter discusses possible recommendations for future work on using formal language theory and the theory of computation in Filipino and Bikol Language grammar checkers.

\section{Context Sensitivity of Philippine Languages}

As evident in the common errors selected in Chapter \ref{errors}, these errors are relatively simple in nature. In other words, these are either specific cases of errors, or the errors selected are simple themselves. For instance, we can add the \textit{gitling} misuse for onomatopoeia in Chapter \ref{gitling_usage}, but this error is not included due to the complexity of the error.

 For that specific case, we need to take into account if the word is \textit{indeed} an onomatopoeia in $\mathcal{F}$, the syllables of that word, the vowel alterations, and other specifications.

 Another example is the CFG representation of the intensifying reduplication, like the phrase "kain nang kain". We can construct the CFG:

 \begin{center}
     \begin{tabular}{rcl}
         S & $\to$ &  V "nang" V \\ 
         V & $\to$ & "kain"
     \end{tabular}
 \end{center}

Having more than one non-terminal for the terminal V would result in the CFG:


 \begin{center}
     \begin{tabular}{rcl}
          S & $\to$ & V "nang" V \\
          V & $\to$ & "kain" | tulog
     \end{tabular}
 \end{center}

However, notice that this CFG will accept "kain nang kain" and "tulog nang tulog", but also "kain nang tulog" and "tulog nang kain" which are incorrect.

A bandage solution to this issue, and to allow us to create a CFG for this specific rule, we can say that we will have a CFG that has the non-terminal symbols S, V$_1$, V$_2$, $\dots$ V$_n$ where $n$ is the size of the set of verbs of $\mathcal{F}$ \cite{Seki_Matsumura_Fujii_Kasami_1991}. Or,

\begin{center}
    \begin{tabular}{rcl}
         S & $\to$ & V$_1$ "nang" V$_1$ $|$ V$_2$ "nang" V$_2$ $|$ $\dots$ $|$ V$_n$ "nang" V$_n$  \\
         V$_1$ & $\to$ & "kain" \\
         V$_2$ & $\to$ & "tulog" \\
         $\vdots$ & $\vdots$ & $\vdots$ \\
         V$_n$ & $\to$ & "takbo"
    \end{tabular}
\end{center}

Doing so will allow us to create a CFG that will accept the sentence structure $\omega \text{ nang } \omega$ where $\omega \in \mathcal{F}_\text{verbs}$. However, this is not very efficient as we will have $|\mathcal{F}_\text{verbs}| + 1$ the number of non-terminals and $2\times|\mathcal{F}_\text{verbs}|$ terminals.

Therefore, we need to add a constraint to a CFG, essentially creating an \textit{Attribute Grammar} \cite{kalita_kumar_roy_2022}. Doing so, we get

\begin{center}
    \begin{tabular}{rcll}
         S &$\to$& V$_1$ "nang" V$_2$ & : V$_1$, V$_2$ $\in$ V and V$_1$ = V$_2$  \\
         V &$\to$& "kain" $|$ "tulog" $| \cdots |$ "takbo" $|$ &
    \end{tabular}
\end{center}

Furthermore, introducing \textit{Context-Sensitive Grammars} will allow for the minimization of terminal and non-terminal symbols, making the grammar efficient \cite{Stabler_2004, Rawski_Dolatian_Heinz_Raimy_2023}. 

\section{Turing Machines, Turing Completeness, and Java}

It is the case that we need to extend the capability of CFGs or use automata that are strictly more powerful than CFGs. LanguageTool accomplished checking for errors that are at least as complex as the rule defined above through Java code \cite{LanguageToolDevDocs}.

It follows that, if a grammar rule cannot be expressed as a CFG, then expressing it as Java code is necessary since Java is \textit{Turing complete}. Thus, it is strictly more powerful than a CFG \cite{church_turing}.

Below are examples of grammar rules that are impossible or inefficient to implement as a CFG.

\paragraph{Kump and Kumb Words} If \(s\in \mathcal{S}\), and the prefix of \(s\) is given by the regular expression \((\texttt{C}\mid\texttt{c})\texttt{on}(\texttt{f}\mid\texttt{v})\). Then, the translation of $s$ to $\mathcal{F}$ or \(\mathcal{F}(s)\) is prefixed with \((\texttt{C}\mid\texttt{c})\texttt{um}(\texttt{p}\mid\texttt{b})\).

In particular:
\begin{itemize}
      \item \(\left(s \in (\texttt{conf})\mathbb{S}^*\right) \Longrightarrow \left(\mathcal{F}(s) \in (\texttt{kump})\Sigma^*\right)\)
      \item \(\left(s \in (\texttt{conv})\mathbb{S}^*\right) \Longrightarrow \left(\mathcal{F}(s) \in (\texttt{kumb})\Sigma^*\right)\)
\end{itemize}

\paragraph{Es- and Is- Words} Given the Spanish and English roots of Filipino, some \textit{loan} words have rules for Filipino spelling. Let \(s\) be any string, the English language \(\mathcal{E}\), the Spanish language \(\mathcal{S}\), and \(\mathcal{F}(s)\) be the translation of \(s\) in \(\mathcal{F}\).

\begin{enumerate}
      \item \(\left(\forall s\in \mathcal{S},s \in \texttt{es}\mathbb{S}^*\right) \Longrightarrow \left(\mathcal{F}(s) \in \texttt{es}\Sigma_\mathcal{F}^*\right) \)
      \item \(\left(\forall s\in \mathcal{E},s \in \texttt{s}\mathbb{E}^*\right) \Longrightarrow  \left(\mathcal{F}(s) \in \texttt{is} \Sigma_\mathcal{F}^* \right)\)
\end{enumerate}

Rule (1) denotes that if \(s\) is a Spanish word, translating \(s\) to a Filipino word would use \texttt{es} as the prefix to the word to denote that \(\mathcal{F}(s)\) is a word of Spanish origin. On the other hand, for rule (2), if \(s\) is an English word, then \(\mathcal{F}(s)\) would use \texttt{is} as the prefix to the word to denote that it is of English origin.

\paragraph{Onomatopeia} Let \( k \in \mathcal{F} \) be an onomatopoeic word given by:
\[
      k = s_0\texttt{-}s_1\texttt{-}\cdots\texttt{-}s_n 
\]
where \(\forall n, s_n \in \mathbb{O} \) represents an \textit{onomatopoeic syllable}, \( \mathbb{O} \) is the set of all such syllables \( n \geq 1 \), and each \( s_n \) is a single-syllable onomatopoeic unit. The word consists of at least two such syllables, separated by a \textit{gitling}.

\paragraph{Fifth Case of \texttt{Nang}} Used in between repeated verbs.

\[
      S_5 = v_1 \cdot p \cdot \text{"nang"} \cdot p \cdot v_1
\]

\paragraph{Bikol Language: U-O Distribution for Reduplications}
Words that are a reduplication, or a joining of two syllables, can have the \texttt{u} sound at the end if and only if the reduplicated syllable does indeed have the \texttt{u} sound \cite{bikol_dictionary}.

Formally, any word given by the IEEE Posix regular expression:
\[
      \left( \Sigma_\mathbb{B}^*\texttt{u}\Sigma_\mathbb{B}^* \right)\backslash1 \in \mathcal{B}
\]

Below is an example implementation of the Verb-Nang-verb grammar rule in LanguageTool using Java.

% Maximum source code width                                             |
\begin{lstlisting}[language=C, caption=Implementation of Verb-Nang-Verb Grammar Rule in Java]
import org.languagetool.AnalyzedSentence;
import org.languagetool.AnalyzedTokenReadings;
import org.languagetool.rules.Rule;
import org.languagetool.rules.RuleMatch;

import java.io.IOException; 
import java.util.ArrayList;
import java.util.List;

public class VerbNangVerbRule extends Rule {

    private static final String[] VERBS = {
        "mangyari", "gumawa", "sumulat", "tumakbo", 
        "kumain", "maglaro", "magbasa"  
        // Verbs in Filipino
    };

    public VerbNangVerbRule(Language language) {
        List<Pattern> patterns = new ArrayList<>();
    }

    @Override
    public String getId() {
        return "VERB-NANG-VERB_RULE";
    }

    @Override
    public String getDescription() {
        return "A rule that detects errors in the phrase 
        structure VERB-Nang-VERB.";
    }

    @Override
    public RuleMatch[] match(AnalyzedSentence sentence) 
        throws IOException {
        List<RuleMatch> ruleMatches = new ArrayList<>();
        AnalyzedTokenReadings[] tokens = 
            sentence.getTokensWithoutWhitespace();

        for (int i = 1; i < words.length - 1; i++) {
            String prevWord = words[i - 1];
            String nextWord = words[i + 1];

            if ("nang".equals(words[i])) {
                // Check if the words surrounding "nang" 
                // are in the verbs array and are the same
                if (isVerb(prevWord) && 
                    isVerb(nextWord) && 
                    prevWord.equals(nextWord)) {
                    
                    RuleMatch ruleMatch = 
                        new RuleMatch(this, i, i + 2);
                    
                    ruleMatches.add(ruleMatch);

                    RuleMatch ruleMatch = new RuleMatch(
                            this,
                            sentence,
                            i - 1,
                            i + 1,
                            this.getMessage();
                    );
                    ruleMatches.add(ruleMatch);
                }
            }
        }

        return ruleMatches;
    }

    private boolean isVerb(String word) {
        for (String verb : VERBS) {
            if (verb.equals(word)) {
                return true;
            }
        }
        return false;
    }

    private String getMessage(){
        return "Verb-Nang-Verb misuse; the verbs 
                surrounding 'nang' must be the same.";
    }
}
\end{lstlisting}

In conclusion, this paper has explored the challenges and opportunities in using formal language theory, specifically Context-Free Grammars (CFGs) and regular expressions, for error checking in the grammar of Filipino and the Bikol language. We have discussed several common errors and provided various solutions, ranging from simple CFG representations, their CNF and GNF equivalents, to more advanced constructs like Attribute Grammars and Context-Sensitive Grammars, addressing the intricacies of the Philippine languages. Furthermore, we have highlighted the limitations of CFGs when confronted with complex grammatical rules and demonstrated how the power of Turing-complete programming languages, such as Java, can be leveraged to handle errors beyond the capabilities of CFGs.

As we move forward, integrating more sophisticated models for computation will be essential for further refining grammar checkers for Philippine languages. By combining formal language theory with machine learning techniques and natural language processing, future work can significantly enhance the accuracy and efficiency of Filipino grammar checkers, enabling a more nuanced understanding of language use and error detection.

\printbibliography

\end{document}